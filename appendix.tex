\chapter{EM-CCD camera calibration}
\section{Andor Basic code listing for automatic image acquisition}
\label{sec:basic-acquisition}
\definecolor{light-gray}{gray}{0.95}

\lstdefinestyle{myframe}{
  basicstyle=\LSTfont,
  % basicstyle=\footnotesize\ttfamily,
  rulesepcolor=\color{gray} ,
  rulecolor = \color{black},
  frame = single,
  % framerule = 0pt,
 % backgroundcolor =\color{light-gray}, 
  fontadjust=true,
  breaklines = true,
  showstringspaces=false,
  commentstyle=\itshape,
}
\lstdefinestyle{mymaxima}{
  language=C,
  title={Maxima},
  style=myframe
}
\lstdefinestyle{myclang}{
  language=C,
  title={C language},
  style=myframe
}
\lstdefinestyle{myfortran}{
  language=Fortran,
  style=myframe
}
\lstdefinestyle{mymatlab}{
  language=Matlab,
  title={Matlab},
  style=myframe
}
\lstdefinestyle{mylisp}{
  language=Lisp,
  title={Common Lisp},
  style=myframe
}
\lstdefinestyle{mybasic}{
  title={Andor Basic},
  language=[Visual]Basic,
  style=myframe
}
\lstdefinestyle{mypython}{
  language=Python,
  title={Python},
  showstringspaces=false,
  tabsize=4,
  basicstyle=\ttfamily,
  morekeywords={models, lambda, forms},
  frame = single,
  breaklines = true,
  style=myframe
}

The following code allows to record data for calibration of a Andor
EM-CCD with as little user interaction as possible. More detail is
given in the main text in section \ref{sec:ccd-intro} on page
\pageref{sec:ccd-intro}.

The program is written in a dialect of Basic and automates the Andor Solis software.
Before use, a picture resembling
\figref{fig:shot-noise} on page \pageref{fig:shot-noise} should be
imaged onto the sensor, e.g.\ a defocused fluorescent sample. The
acquired data can later be processed using either the Matlab/DIPimage
function \verb!cal_readnoise! or, equivalently, by using the Python
script from the next section.

A camera calibration allows to convert image data into the device
independent unit of effective photoelectrons but this only works as
long as data acquisition occurs at the same camera settings (pre-amp
gain, EM-gain, sensor temperature, vertical shift speed, readout rate,
...) as those that were used for the calibration. This Basic program
measures data for a wide range of EM-CCD amplification settings, but
it could easily be adopted to analyze other parameters as well.

The camera that I have used in the development of this program
comprises an internal mechanical shutter. This is useful because it is
important that for each parameter setting at least one dark image is
acquired (but it is better to acquire two dark images and use the
second). Ultimately, dark images are necessary to quantify the readout
noise.

The code listings in this section are supposed to be in this sequence
in a single source file.

For this program I assume that the camera is illuminated with a
continuous flux of photons. The program is designed to acquire images
with a wide range of EM gains. Since the photon flux remains constant
but amplification varies greatly, I acquire one image with a short
$\unit[10]{\mu s}$ exposure prior to each measurement. The following
function searches for the maximum value in this image and calculates
an appropriate exposure time so that a maximum of \unit[10000]{ADU}
will be obtained in the preceeding acquisitions.
\begin{lstlisting}[style=mybasic]
function ~GetSaturatingExposure()
        SetKineticNumber(1)
        exp=.01
        SetExposureTime(exp)
        run()
        m=maximum(#0,1,512)
        GetSaturatingExposure=exp*10000/(m-100) % 100 is the background in ADU
        CloseWindow(#0)
return
\end{lstlisting}
The following code listing selects the conventional readout register
of the sensor (a register without EM multiplication), acquires 20
images without and then with light by calling the function
\verb!run()! and stores the data in a TIF image file.
\begin{lstlisting}[style=mybasic]
name$ = "C:\Users\work\Desktop\martin\20111111\scan-em3\ixon_"
print("start")

SetOutputAmp(1)
print("conv_start")
exp= ~GetSaturatingExposure()
print(exp)
SetExposureTime(exp)
SetKineticNumber(20)
SetShutter(0,1)
run()
save(#0,name$ + "conv1_dark.sif")
ExportTiff(#0, name$ + "conv1_dark.tif", 1, 1, 0, 0)
CloseWindow(#0)
CloseWindow(#1)

SetShutter(1,1)
run()
save(#0,name$ + "conv1_bright.sif")
ExportTiff(#0, name$ + "conv1_bright.tif", 1, 1, 0, 0)
CloseWindow(#0)
CloseWindow(#1)
\end{lstlisting}
\comment{ $ } The loop in the next listing makes similar acquisitions
for a range of EM gains.  Evaluating the data revealed that a settling
time of a few seconds should be allowed for after calling
\verb!SetGain!. After all, there are relatively high voltages. I
inserted a comment in the corresponding line.
\begin{lstlisting}[style=mybasic]
SetOutputAmp(0)
SetShutter(1,1)
for i = 40 to 300 step 10
        SetGain(i)
        % here should be a 3s wait
        exp=~GetSaturatingExposure()
        print(exp)
        SetExposureTime(exp)
        SetKineticNumber(20)
        SetShutter(0,1)
        run()
        save(#0,name$ + str$(i) + "_dark.sif")
        ExportTiff(#0, name$ + str$(i) + "_dark.tif", 1, 1, 0, 0)
        CloseWindow(#0)
        CloseWindow(#1)
        SetShutter(1,1)
        run()
        save(#0,name$ + str$(i) + "_bright.sif")
        ExportTiff(#0, name$ + str$(i) + "_bright.tif", 1, 1, 0, 0)
        CloseWindow(#0)
        CloseWindow(#1)
next
\end{lstlisting}
Finally I acquire a last measurement with the conventional readout
register: \begin{lstlisting}[style=mybasic]
SetOutputAmp(1)
print("conv_end")
exp= ~GetSaturatingExposure()
print(exp)
SetExposureTime(exp)
SetKineticNumber(20)
SetShutter(0,1)
run()
save(#0,name$ + "conv2_dark.sif")
ExportTiff(#0, name$ + "conv2_dark.tif", 1, 1, 0, 0)
CloseWindow(#0)
CloseWindow(#1)
        
SetShutter(1,1)
run()
save(#0,name$ + "conv2_bright.sif")
ExportTiff(#0, name$ + "conv2 _bright.tif", 1, 1, 0, 0)
CloseWindow(#0)
CloseWindow(#1)
\end{lstlisting}
Table \ref{tab:ixon-table} summarizes results of calibration
measurements that were acquired using this software and evaluated
using the Python code from the next section.

Unfortunately, the data in the first and last line (conv1 and conv2
with $M=1$) show a disparity in the pre-amplifier gain
$M_\textrm{pre}$. An improvement would be to use an LED light source
(which doesn't bleach) or intersperse conventional readouts between
measurements with EM gain in order to compensate for variations due to
bleaching.

% 1.3165/1.593*15496 = 12806 < 14923

\begin{table}[!htbp]
  \centering
%  \begin{tabular}{|l|l|l|l|l|l|l|l|}
  \begin{tabular}{r l l r  l r l}
\toprule
$\textsf{gain}_\textrm{software}$ & $1/(M\cdot M_\textrm{pre})$ & $N_r$ & $N_{(M)}/(W\times H)$ &  \textsf{exposure} & $N_{(M)}'/(W\times H)$ & $1/F_n$ \\
 & [$e^-/$ADU] & [$e^-/$px] & [$e^-/$px] & [ADU] & [s] & [$e^-/$(px s)]  \\
\midrule
conv1 & 1.3165 & 7.189 & 3008.66      & 0.2016 & 14923 & 0.981 \\
50 & 0.1160 & 0.486 & 260.05 & 0.0289 & 8995 & 0.591 \\
60 & 0.0984 & 0.406 & 225.46 & 0.0249 & 9054 & 0.595 \\
70 & 0.0841 & 0.349 & 190.52 & 0.0212 & 8983 & 0.591 \\
80 & 0.0729 & 0.305 & 165.24 & 0.0186 & 8907 & 0.586 \\
90 & 0.0680 & 0.288 & 150.54 & 0.0161 & 9368 & 0.616 \\
100 & 0.0611 & 0.262 & 128.47 & 0.0136 & 9427 & 0.620 \\
110 & 0.0550 & 0.241 & 121.11 & 0.0129 & 9409 & 0.619 \\
120 & 0.0510 & 0.228 & 113.71 & 0.0120 & 9498 & 0.624 \\
130 & 0.0465 & 0.211 & 106.66 & 0.0112 & 9541 & 0.627 \\
140 & 0.0433 & 0.201 & 96.95 & 0.0101 & 9564 & 0.629 \\
150 & 0.0405 & 0.192 & 89.68 & 0.0093 & 9671 & 0.636 \\
160 & 0.0380 & 0.183 & 87.24 & 0.0090 & 9656 & 0.635 \\
170 & 0.0359 & 0.175 & 81.56 & 0.0084 & 9739 & 0.640 \\
180 & 0.0339 & 0.169 & 79.80 & 0.0081 & 9863 & 0.648 \\
190 & 0.0321 & 0.163 & 74.00 & 0.0075 & 9806 & 0.645 \\
200 & 0.0305 & 0.158 & 72.57 & 0.0073 & 9878 & 0.649 \\
210 & 0.0292 & 0.155 & 69.44 & 0.0070 & 9944 & 0.654 \\
220 & 0.0280 & 0.150 & 67.69 & 0.0068 & 9971 & 0.656 \\
230 & 0.0268 & 0.147 & 65.63 & 0.0065 & 10057 & 0.661 \\
240 & 0.0257 & 0.188 & 63.90 & 0.0063 & 10131 & 0.666 \\
250 & 0.0244 & 0.140 & 62.52 & 0.0062 & 10026 & 0.659 \\
260 & 0.0237 & 0.137 & 62.86 & 0.0062 & 10078 & 0.663 \\
270 & 0.0229 & 0.135 & 63.17 & 0.0062 & 10130 & 0.666 \\
280 & 0.0221 & 0.133 & 63.64 & 0.0062 & 10204 & 0.671 \\
290 & 0.0214 & 0.130 & 63.38 & 0.0062 & 10162 & 0.668 \\
300 & 0.0205 & 0.128 & 63.20 & 0.0062 & 10133 & 0.666 \\
conv2 & 1.5953 & 8.768 & 8198.86 & 0.5291 & 15496 & 1.019 \\
\bottomrule
\end{tabular}
%  \includegraphics[width=12cm]{../app_cam/ixon3}
\caption{Comparison of read noise for different EM-gain settings
  (first column) of the Andor IXon3. $W$ and $H$ are the size of the sensor (in pixels). The value $N_{(M)}'$
  estimates the number of photoelectrons the detector would have
  seen with \unit[1]{s} integration time and is used to calculate
  the excess noise factor in the last column. In EM-mode the fastest
  readout speed was used (\unit[10]{MHz}) with the default vertical shift speed of
  \unit[1.7]{$\mu$s}.}
  \label{tab:ixon-table}
\end{table}

\newpage

\section{Python code listing for the read noise evaluation}
\label{sec:python-readnoise-eval}
Here I present a Python implementation for evaluating data that has
been recorded using the program from the previous
section. \figref{fig:ixon} shows two evaluations for data with and
without EM-gain using an Andor IXon3 EM-CCD camera.

\begin{figure}[htbp]
  \centering
  \pdfinput{17cm}{ixon_conv1}
  \pdfinput{17cm}{ixon_300}
  \caption{Readout noise evaluation using the Python code {\bf top:}
    Conventional readout of an Andor iXon3 camera. {\bf bottom:}
    readout with an EM-gain setting of 300 on the same camera with
    identical sample. {\bf left:} 2D histogram of per pixel variances
    against binned intensities. red data points: measuremnts, blue
    line: linear fit {\bf middle:} variance of 20 dark images. {\bf
      right:} mean of 20 dark images.}
  \label{fig:ixon}
\end{figure}
The following code loads several Python packages. Essentially, I use
\verb!numpy! by loading \verb!pylab! \citep{Jones} for the data
analysis tasks and the package \verb!matplotlib! \citep{Hunter:2007}
for visualizing the results.
\begin{lstlisting}[style=mypython]
#!/usr/bin/env python
# usage:   ti.py DIRECTORY CAMERA_NAME EM_GAIN
# example: ti.py /media/backup/andor-ultra-ixon/martin/20111111/scan-em3/ ultra 2700
import sys
import os
import matplotlib
matplotlib.use('Agg')
from pylab import *
from libtiff import TIFFfile, TIFFimage
from scipy import stats
seterr(divide='ignore')
\end{lstlisting}
When starting the program I specify which data should be loaded using
command line parameters. The following code will read in the measured
image data with and without illumination.
\begin{lstlisting}[style=mypython]
folder = sys.argv[1]
cam = sys.argv[2]
gain = sys.argv[3]

def readpics(gain,cam='ixon_',isdark=False):
    print 'loading ', os.path.join(folder,cam) + '_' + gain + '_bright.tif'
    fg=TIFFfile(os.path.join(folder,cam) + '_' + gain + '_bright.tif')
    bright,bright_names=fg.get_samples()
    bg=TIFFfile(os.path.join(folder,cam) + '_' + gain + '_dark.tif')    
    dark,dark_names=bg.get_samples()
    return (bright[0],dark[0])

(f,b) = readpics(gain=gain,cam=cam)
\end{lstlisting}
The next code listing creates a two-dimensional histogram with 64 bins
for variances and 128 bins for intensities.
\begin{lstlisting}[style=mypython]
ny,nx=64,128
H,y,x=histogram2d(v.flatten(),i.flatten(),bins=[ny,nx],
                  range=[[0,v.max()],[0,i.max()]])
extent = [x[0], x[-1], y[0], y[-1]] 

fig=figure(figsize=(24, 8),dpi=300)
hold(False)
title('bal')
subplot(1,3,1)
imshow(log(H), extent=extent,
           aspect='auto', interpolation='none',origin='lower')
hold(True)
\end{lstlisting}
The histogram is not strictly necessary for the analysis but gives an
overview of the measured data at a quick glance, i.e.\ if there were
enough measurements for all intensities and whether the sensor was
over-exposed.

The most important part of the evaluation is performed in the
following code segment. Here I accumulate data in the variables
\verb!acc! and \verb!accn! that is later used to determine the average
values of the variances for all intensities.
\begin{lstlisting}[style=mypython]
acc=zeros(x.shape,dtype=float64)
accn=zeros(x.shape,dtype=int64)
s=nx/i.max()
for ii,vv in nditer([i,v]):
    p=round(ii*s)
    acc[p]+=vv
    accn[p]+=1   
\end{lstlisting}
Subsequently, I determine the parameters for a linear fit of the
variances vs.\ intensities, utilizing only the first 60\% of the
intensities so that any non-linearities that might occur at higher
intensities do not affect the slope.
\begin{lstlisting}[style=mypython]
ax=x[nonzero(accn)]
ay=acc/accn
ay=ay[nonzero(accn)]
l=round(.6*len(ax))
bx=ax[0:l]
by=ay[0:l]
plot(ax,ay,'r+')
slope,intercept,rval,pval,stderr=stats.linregress(bx,by)
\end{lstlisting}
From the slope of the line I determine the conversion factor to change
the ADU units into the device-independent unit of photoelectrons. With
this I convert the variance of the dark images into readout noise in
terms of photoelectrons per pixel.
\begin{lstlisting}[style=mypython]
plot(ax,polyval([slope,intercept],ax))
xlabel('intensity/ADU')
ylabel(r'variance/ADU$^2$')
real_gain=1/slope # unit electrons/ADU
read_noise=sqrt(var(b))*real_gain # electrons RMS per pixel
mean_elecs=(mean(f)-mean(b))*real_gain # photoelectrons electrons per pixel
print gain,cam,real_gain,read_noise,mean_elecs,mean(b),rval,pval,stderr
tit='EM-gain: %s, cam: %s, real gain: %.2f e/ADU\n
read noise: %.2f e RMS/pixel, mean: %.2f e/pixel, offset: %.2f'
% (gain,cam,real_gain,read_noise,mean_elecs,mean(b))
title(tit)
subplot(1,3,2)
imshow(var(b,axis=0))
title('variance of darkimages')
colorbar()
subplot(1,3,3)
imshow(mean(b,axis=0))
title('mean of darkimages')
colorbar()
show()
fig.savefig(cam+'_'+gain+'.png')
\end{lstlisting}


\chapter{Optical sectioning by structured illumination using the HiLo method}
\label{sec:sectioning}
\begin{summary}
  In this chapter I compare three methods to calculate optical
  sections from fluorescence microscope images with structured
  illumination. I employ the wave-optical model of image formation
  (see section \ref{sec:wave-image-formation}) to simulate focal
  stacks with structured illumination of a three-dimensional
  fluorophore distribution. 

  The code in this chapter permits to simulate aberration-free imaging
  and to compare the performance of sectioning algorithms at different
  light levels, i.e.\ in the presence of photon shot noise.
\end{summary}

\comment{
\jpginput{}{m_phase}{}
}

First, I will construct a three-dimensional fluorophore
distribution. Then I will calculate the three-dimensional point spread
function $h(\r)$ of a high-aperture immersion objective with
$\textrm{NA}=1.4$ and $n=1.52$ using equations (\ref{eq:psf}) and
(\ref{eq:mccutchen}) on page \pageref{eq:mccutchen}. Using this
three-dimensional array (denoted by \verb!psf(:,:,:)! in the code),
the three-dimensional light distribution in the sample, as well as
images on the camera can be calculated by a convolution
operation\footnote{I use the same point spread function for excitation
  and detection. More realistic results can be obtained when the point
  spread functions are scaled according to excitation and emission
  wavelengths.}.

For the numerical simulation I use the DIPimage Toolbox for Matlab
\citep{dipimage} as it permits to express the required image
processing algorithms with a relatively elegant syntax.

The \cma{syntax of DIPimage} following code listing fills a
three-dimensional array with the geometric primitives line, rectangle
and a sphere shell. The function \verb!newim(GX,GX,GX)! creates a new
DIPimage data object. In general, DIPimage functions can receive their
arguments either explicitly, as in this case, or implicitly by calling
with another DIPimage data object, e.g.\ when calling \verb!rr(g2)!
for constructing the hollow sphere. The function \verb!rr()! returns
an array where each element contains the distance to the central
pixel. The comparison operation results in a Boolean data type which I
implicitly turn back into a floating point number by adding a zero.

To save computation time when applying the section algorithms, I cut
\verb!S! to the smallest dimensions, so that it still contains the
complete object. I implement all convolution operations with Fourier
transforms. Therefore the data should be padded with zeros to prevent
overlap of the image information at the borders. I used padding for
all images that I show here.

The figure next to the code shows the geometry of the
three-dimensional fluorophore distribution and three simulated
widefield images of the planar objects.

\begin{lstlisting}[style=mymatlab]
global GX; GX = 64; g2 = newim(GX,GX,GX); % make empty 3d image
line_z=floor(size(g2,3)/2)-3;
lineseg = drawline(g2,floor([.3*GX 0 line_z]),floor([.8*GX .9*GX line_z]),1);

rect_x0 = floor(.2*GX); rect_x1 = floor(.9*GX);
rect_y0 = floor(.2*GX); rect_y1 = floor(.5*GX);
rect_z1 = floor(GX/2)+3;
rectangle1 = newim(g2);
rectangle1(rect_x0:rect_x1,rect_y0:rect_y1,rect_z1)=1;

rect_x0 = floor(.6*GX); rect_x1 = floor(.8*GX);
rect_y0 = floor(.0*GX); rect_y1 = floor(.9*GX);
rect_z2 = floor(GX/2)+9;
rectangle2 = newim(g2);
rectangle2(rect_x0:rect_x1,rect_y0:rect_y1,rect_z2)=1;

hollow_sphere = 0.0 + (.3<rr(g2,'freq') & rr(g2,'freq')<.4); 

S = 12 * lineseg + 4 * (rectangle1 +rectangle2) + hollow_sphere;
[unused1,unused2,S] = bbox(S>0,S); clear g2;
\end{lstlisting}  %3.33cm hoch
\vspace{-6.33cm}\hspace{9cm} \svginput{1}{hilo-sec-S}

\vspace{3cm} With the following code I determine the generalized
McCutchen aperture for a lens with a given numerical aperture
\verb!NA!. For this I calculate a spherical shell that touches all
sides of the auxiliary array \verb!a(:,:,:)! \citep{Vembu1961}.

The size of the array \verb!calotte(:,:,:)! is calculated using the
numerical aperture, so that it contains only the McCutchen Aperture
(the amplitude point spread function $a(\vnu)$ in section
\ref{sec:wave-image-formation}). In order to enable convolutions
without aliasing artifacts when using fast Fourier transforms, I
choose the size of the array \verb!otf(:,:,:)! for the optical
transfer function twice as big as \verb!calotte(:,:,:)!. If the object
in \verb!S! is bigger, then I increase the size of \verb!otf(:,:,:)!
to encompass this also.  According to equation (\ref{eq:resolution})
on page \pageref{eq:resolution} the $z-$sampling is
\begin{align}
  \delta z &= \frac{\lambda_0}{n(1-\cos\alpha)}\kappa
\end{align}
with $\kappa=$ \verb!2*size(calotte,3)/size(psf,3)!. Figure a) shows a
$xz-$cross section through the three-dimensional point spread
function, and Figure b) is the cross section through the
three-dimensional optical transfer function.
\begin{lstlisting}[style=mymatlab]
n = 1.52; % refractive index
NA = 1.4; % numerical aperture of the lens
alpha = asin(NA/n);  % acceptance half-angle of lens
X = 37; g = newim(X,X,X); % X should be odd, so that g has a center pixel
a=ft(sinc(rr(g)*pi)); % draw sphere that fills g exactly
zpos = floor(X/2) + round(.5*X*cos(alpha));
xpos1 = floor(X/2) - round(.5*X*sin(alpha));
xpos2 = floor(X/2) + round(.5*X*sin(alpha))
% cut out part of the sphere depending on lenses aperture:
calotte = a(xpos1:xpos2,xpos1:xpos2,zpos:end); 
clear g a;

center_ref = @(a) a(floor(size(a,1)/2),floor(size(a,2)/2),floor(size(a,3)/2));
otf=extract(calotte,max(size(S),size(calotte)*2));
psf = ift(otf);
psf = psf*conj(psf);
otf = real(ft(psf));
otf = otf / center_ref(otf);
psf = psf / center_ref(otf); 

psf2d = abs(ft(extract(calotte,size(calotte)*2)))^2;
psf2d = psf2d(:,:,floor(size(psf2d,3)/2));
otf2d = real(ft(psf2d));

otf2dcorr = DampEdge(rr(otf2d,'freq')<.49,.03,2,0);
otf2dcorr = otf2dcorr/otf2d;
otf2dcorr = otf2dcorr / center_ref(otf2dcorr);
% ensure the array the same 2d extent as the 3d otf
otf2dcorr = squeeze(extract(otf2dcorr,[size(psf,1) size(psf,2)]));
\end{lstlisting}  %2.43cm hoch
\vspace{-6.43cm}\hspace{12cm} \svginput{1}{hilo-sec-psfotf} 
\vspace{2.5cm}

Images on the camera are obtained according to equation
(\ref{eq:image-convolution}) on page \pageref{eq:image-convolution} by
a convolution of the three-dimensional fluorophore distribution
\verb!S! with the point spread function \verb!psf(:,:,:)!. In the case
of uniform widefield illumination this is possible using just two
three-dimensional Fourier transforms.
\begin{lstlisting}[style=mymatlab]
WF = real(ift(ft(extract(S,size(psf))) * otf));
phases = 4; % must be even, so that pi is sampled for HiLo
struc = newim(size(otf,1),size(otf,2),size(S,3),phases);
for slice = 0:size(S,3)-1
    struc(:,:,slice,:) = create_structured_slice(S,otf,phases,slice);
end
\end{lstlisting}
For structured illumination, four three-dimensional Fourier transforms
are necessary for each slice and phase of the projected grating. The
following function calculates images of one slice for all grating
phases. I use \verb!dip_fouriertransform! to exclude the array
dimension containing the phase from the transformation.  With a
$70\times 70\times 51$ sized point spread function, a single call to
the following function takes \unit[1.4]{s} per slice on one Intel
i5-2430M core.

Figure a) shows the sinusoidal grating before imaging by the
lens. Figure b) shows the simulated in-focus light distribution for
incoherent illumination. According to this, the contrast in the image
is less than in the original grating. The $xz-$cross section in Figure
c) shows that the modulation occurs only in a narrow range around the
focal plane. The scalebars correspond to a wavelength
of $\lambda_0=\unit[500]{nm}$.
\begin{lstlisting}[style=mymatlab]
function Struc = create_structured_slice(S,otf,phases,slice)
  % generate 'phases' different sine gratings 
  % I use DampEdge to ensure a smooth transition to zero at the edges
  G=newim([size(S) phases]); 
  tilt = 2*pi*xx(size(S,1),size(S,2))/64*12;
  G(:,:,slice,:) = DampEdge(.5*(1+sin(repmat(tilt,[1 1 1 phases])+ramp([size(tilt),1,phases],4)*2*pi/phases)),.18,2);
  
  % generate the three-dimensional light distribution in the sample
  WF_z = floor(size(otf,3)/2)-floor(size(S,3)/2)+slice;
  kG = dip_fouriertransform(extract(G,size(otf)),'forward',[1 1 1 0]);
  kG = kG*repmat(otf,[1 1 1 phases]);
  Illum = real(dip_fouriertransform(kG,'inverse',[1 1 1 0]));
  
  % multiply fluorophore distribution S with the light distribution
  kG = Illum *  repmat(extract(S,size(otf)),[1 1 1 phases]);
  kG = dip_fouriertransform(kG,'forward',[1 1 1 0]);
  kG = kG * repmat(otf,[1 1 1 phases]);
  kG = dip_fouriertransform(kG,'inverse',[1 1 1 0]);
  
  % return a 2D slice for each grating phase
  Struc = real(kG(:,:,WF_z,:));
\end{lstlisting}

\vspace{-6.5cm}\hspace{12cm} \svginput{1}{hilo-sec-Illum}

\vspace{1.5cm} The following code adds photon noise to the image data,
so that the brightest pixel corresponds to a signal of 60000 photons,
and reconstructs optical sections with three different algorithms. The
code to implement equations (\ref{eq:Iminmax}) and
(\ref{eq:Ihomodyne}) on page \pageref{eq:Iminmax} fits into a single
line. Reconstructed sections are shown in
\figref{fig:hilo-sec-comparison} on page
\pageref{fig:hilo-sec-comparison} and cross sections along z in
\figref{fig:cross-section-z} in this section.
\begin{lstlisting}[style=mymatlab]
section_max_min = @(a) squeeze(max(a,[],4)-min(a,[],4))
section_homodyne = @(a) squeeze(abs(a(:,:,:,0)+a(:,:,:,1)*exp(i*2*pi*1/4)+a(:,:,:,2)*exp(i*2*pi*2/4)+a(:,:,:,3)*exp(i*2*pi*3/4)))

noise_struc=noise(struc/max(struc)*60000,'poisson'); % reconstruct from noisy images
maxmin=section_max_min(noise_struc);
homody=section_homodyne(noise_struc);
hilo=section_hilo(noise_struc,.13,otf2d,otf2dcorr);
\end{lstlisting}

\comment{
\pdfinput{}{cross-section-z}
}
\begin{figure}[htbp]
  \centering
  \includegraphics{cross-section-z_vector}
  \caption{Cross sections along z through the reconstructed focal
    stack for different sectioning algorithms. The violet line depicts
    the fluorophore distribution in the sample and the cyan colored
    line is a section through the widefield stack.}
  \label{fig:cross-section-z}
\end{figure}
The following function implements the HiLo algorithm as described on
page \pageref{fig:hilo-method-description}. Variables with the digit '3'
at the end are two-dimensional array that have been extended to three
dimensions by copying.
\begin{lstlisting}[style=mymatlab]
function [sec uni nonuni] = section_hilo(struc,filter_fwhm,otf2d,otf2dcorr)
  otf2dcorr3 = repmat(otf2dcorr,[1 1 size(struc,3)]);
  uni = otf2dcorr3*squeeze(dip_fouriertransform(struc(:,:,:,0)+struc(:,:,:,2),'forward',[1 1 0 0]));
  nonuni_unshifted = otf2dcorr3*squeeze(dip_fouriertransform(struc(:,:,:,0)-struc(:,:,:,2),'forward',[1 1 0 0]));
  tiltbig = 2*pi*xx(size(uni,1),size(uni,2))/GX*12;
  tilt3=repmat(exp(-i*tiltbig),[1 1 size(struc,3)])
  nonuni = dip_fouriertransform(tilt3*dip_fouriertransform(nonuni_unshifted,'inverse',[1 1 0]),'forward',[1 1 0]);

  filter_sigma = filter_fwhm/sqrt(log(2));
  lowpass = extract(DampEdge(exp(-rr(otf2d)^2/(filter_sigma*size(otf2d,1))^2),.2,2,0),[size(uni,1) size(uni,2)]);
  hipass = 1-lowpass;

  ring = rr(otf2d) < filter_fwhm*size(otf2d,1);
  ring = extract(bdilation(ring)-ring,[size(uni,1) size(uni,2)]);

  ring3 = repmat(ring,[1 1 size(uni,3)]);
  lowpass3 = repmat(lowpass,[1 1 size(uni,3)]);
  hipass3 = repmat(hipass,[1 1 size(uni,3)]);

  ringhi = mean(ring3*abs(hipass3*uni)^2,[],[1 2]);
  ringlo = mean(ring3*abs(lowpass3*nonuni)^2,[],[1 2]);
  eta = ringhi/ringlo; % one value per slice
  eta_median=median(eta);
      
  image_lo = imag(dip_fouriertransform(lowpass3 * nonuni,'inverse',[1 1 0]));
  image_hi = real(dip_fouriertransform(hipass3 * uni,'inverse',[1 1 0]));
  sec = image_lo+image_hi/eta_median;
\end{lstlisting}
\chapter{Mapping between focal plane SLM and camera}
\label{sec:app_map}
\section{Rigid coordinate transformation}
\label{sec:map_maxima}
The equation for the least squares problem in equation
\ref{eq:rigid-sum} on \pageref{eq:rigid-sum} can be expressed
componentwise:
\begin{align}
  \mathcal{Q}=\sum_i^n&
  \abs{s(\cos\phi r^c_{ix}+q\sin\phi r^c_{iy})+t_x-r^d_{ix}}^2
  +
  \abs{s(-\sin\phi r^c_{ix}+q\cos\phi r^c_{iy})+t_y-r^d_{iy}}^2
\end{align}
The following Maxima code will find the solution to the least squares
problem:
\cma{Maxima}
\begin{lstlisting}[style=mymaxima]
load(minpack)$
q:-1;
g(s,p,tx,ty):=[s*( cos(p)*<cx>+q*sin(p)*<cy>)+tx-<dx>,
               s*(-sin(p)*<cx>+q*cos(p)*<cy>)+ty-<dy>, ... ]$
minpack_lsquares(g(s,p,tx,ty), [s,p,tx,ty], [0.88,-3.1,1200,-20]);
\end{lstlisting}
Where \verb!g! is a vector function, which contains two entries for
each pair of points of the focal plane SLM and camera coordinates.  I
computationally construct the lines according to the given pattern,
replacing \verb!<cx>!, \verb!<cy>!  with measured camera coordinates
and \verb!<dx>!, \verb!<dy>! with display coordinates (see the Matlab
code in the next section).

The function \verb!minpack_lsquares! calls the subroutine \verb!lmder!
which was originally developed for the Fortran package \verb!minpack!.
\cma{Fortran}
\begin{lstlisting}[style=myfortran]
c     subroutine lmder (http://www.netlib.org/minpack/lmder.f)
c     the purpose of lmder is to minimize the sum of the squares of
c     m nonlinear functions in n variables by a modification of
c     the levenberg-marquardt algorithm. the user must provide a
c     subroutine which calculates the functions and the jacobian.
c     the subroutine statement is
c       subroutine lmder(fcn,m,n,x,fvec,fjac,ldfjac,ftol,xtol,gtol,
c                        maxfev,diag,mode,factor,nprint,info,nfev,
c                        njev,ipvt,qtf,wa1,wa2,wa3,wa4)
\end{lstlisting}
Maxima automatically calculates the symbolic Jacobian and thereby
removes an error-prone part of the programmer's work for such
optimization problems. This code could easily be modified for more
complicated transformations, e.g.\ including distortion. Because of
this flexibility, I decided to use Maxima.


\subsection{Application of the rigid transform in OpenGL}
\label{sec:map_opengl}
The results of the parameter optimization can then be used to adjust
the displayed SLM patterns to object positions on earlier camera
images. In particular, it is straight forward, to implement the rigid
transform using OpenGL's transform primitives (OpenGL is the graphics
library, that I usually use).

There are two possibilities of applying the transform: On the one
hand, geometrical primitives might be displayed on the focal plane
SLM, so that particular areas on the camera are illuminated.  On the
other hand, a camera image which was acquired earlier can be displayed
as texture on the focal plane SLM.

Uploading camera images to the focal plane SLM is too slow in our
final system to use the latter method\footnote{However, I obtained
  interesting results with \unit[30]{Hz} frame rate of the camera and
  fast feedback to an LCoS controller that was directly connected to a
  graphics card.}

For this reason, I have mainly used the first method and transform the
geometric primitives before I display them on the focal plane SLM.

Here is the corresponding Common Lisp code to initialize the OpenGL
modelview matrix with the rigid transform, so that drawn objects will
appear at the given positions on the camera.
\cma{Common Lisp}
\begin{lstlisting}[style=mylisp]
(defun load-cam-to-lcos-matrix (&optional (x 0s0) (y 0s0))
  (let* ((s 0.828333873909549) (sx  s)        (sy  (- s))
         (phi -3.102)          (sp (sin phi)) (cp (cos phi))
         (tx 608.433)          (ty 168.918)
         (a (make-array (list 4 4) :element-type 'single-float
             :initial-contents
             (list (list (* sx cp)    (* sy sp)  .0   (+ x tx))
                   (list (* -1 sx sp) (* sy cp)  .0   (+ y ty))
                   (list .0           .0        1.0   .0)
                   (list .0           .0         .0  1.0)))))
    (gl:load-transpose-matrix (sb-ext:array-storage-vector a))))    
\end{lstlisting}  
Alternatively, here is the equivalent code in C (with different
parameters):
\cma{C language}
\begin{lstlisting}[style=myclang]
float m[4*4]; // OpenGL Modelview Matrix
float s=-.8749328910202312,
      sx=s,sy=-s,phi=-.8052030670943575,
      cp=cos(phi),sp=sin(phi),
      tx=1456.71806436377,
      ty=910.4787738693659;
  m[0]=   sx*cp;   m[4]=sy*sp;   m[8] =0;    m[12]=tx; 
  m[1]=-1*sx*sp;   m[5]=sy*cp;   m[9] =0;    m[13]=ty; 
  m[2]=0;          m[6]=0.;      m[10]=1;    m[14]=0;  
  m[3]=0;          m[7]=0.;      m[11]=0;    m[15]=1;  
glMatrixMode(GL_MODELVIEW);
glLoadMatrixf(m);
\end{lstlisting}


\section{Image processing: Localizing bright spots on the camera}
\label{sec:matlab-spots}
Here I show Matlab/DIPimage code \citep{dipimage} to localize
individual spots on the camera images, prepare the input for Maxima,
read back the fitted parameter values and superimpose the transformed
coordinate system of the camera pixels on the the grid of the focal
plane SLM in order to estimate the quality of the fit.

The software development kit of the Andor cameras provides functions
to store image data along with acquisition parameters in the FITS file
format. This can be loaded into Matlab using the function
\verb!readim! from the DIPimage toolbox.

% cd /mnt/scan 

\begin{lstlisting}[style=mymatlab]
%% load the files
% 0 .. 99 spot images
% only 10..99 usable because the first are on border and not illuminated
a = newim(1392,1040,103);
for i=0:102
% Andor's FITS format isn't read correctly correct this by adding 2^15
  a(:,:,i) = 2^15 + readim(sprintf('o%03d.fits',i));
end
\end{lstlisting}
Unfortunately, DIPimage's \verb!readim! function seems to have a bug
and loads the data as negative values. I manually correct this.

In this particular experiment, the first ten spots on the focal plane
were not illuminated. Therefore, these images are excluded from the
following processing. Notable is image 102 because it contains an
image with uniform illumination. From its histogram I estimated a
threshold value of \unit[800]{ADU} to find a mask that corresponds to
the illuminated region.

The uniformly illuminated image is displayed in
\figref{fig:rigid-pics}~left on page \pageref{fig:rigid-pics}. I
utilize this image for normalization, so that spots in each image have
the same values in each image \verb!corr! regardless of the number of
layers of beads in the spots position. I also subtract a background
value of \unit[510]{ADU}, which I have derived from non-illuminated
pixels of the images.
\begin{lstlisting}[style=mymatlab]
bg = 510; 
bright = squeeze(a(:,:,102)); 
mask = gaussf(bright,8) > 800; % create mask with illuminated area

posmax = newim(100,2);
for i = 10:99
  corr = (squeeze(a(:,:,i)) - bg) / bright * mask; % correct for sample non-uniformity
  [coords,vals] = findmaxima(gaussf(corr,32));     % find coordinates of maximum
  [valss,valsind] = sort(vals);    % sort coordinates by intensity
  tmp = coords(valsind,:);         % collect the maximum with highest
  posmax(i,:) = tmp(end,:);        % intensity into result
end
\end{lstlisting}
The DIPimage toolbox provides the function \verb!findmaxima!, that
locates all local maxima in an image with subpixel accuracy. I sort
the result by grey value and only use the largest.  The measured 90
coordinates in \verb!posmax! correspond to $\r^c_i$ in equation
\ref{eq:rigid-sum}.
 

The following Matlab code generates and executes code in Maxima to
determine the parameters of the transformation, as I explained in the
previous section.

The programmatically generated Maxima batch program is stored in the
file \verb!fit.max!. After a successful run Maxima returns the
transform parameters in the file \verb!max.out!.
\begin{lstlisting}[style=mymatlab]
c = double(posmax)';
cmd = ''; % collect equations in maxima format
for i=10:99 
  dx = num2str(400+50*mod(i,10));
  dy = num2str(500+50*floor(i./10));
  cx = num2str(c(i+1,1));
  cy = num2str(c(i+1,2));
  cmd=[cmd ' s*( cos(p)*' cx '+q*sin(p)*' cy')+tx-' dx ', ...
             s*(-sin(p)*'cx '+q*cos(p)*' cy ')+ty-' dy ','];
end
cmd(:,end) = []; % delete last comma

% load the fitting package and start defining the merit function g
pre = 'load(minpack)$ q:-1; g(s,p,tx,ty):=[';
% now put cmd between
% call the fitting function and store the parameters into max.out
cod = [']$ fit:minpack_lsquares(g(s,p,x,y),[s,p,x,y],[.88,-1.3,1200,-20]);'...
       'write_data(fit[1],"max.out");']
fid = fopen('fit.max','w'); % write maxima commands into file
fwrite(fid,[pre cmd cod]);
fclose(fid);
[max_status,max_result]=system('maxima -b fit.max'); % execute maxima
\end{lstlisting}
I load the transformation parameters back into Matlab and create the a
diagram (shown in \figref{fig:rigid-compare} in the main text on page
\pageref{fig:rigid-compare}) to visualize, how well the transform
matches camera and display coordinates.
\begin{lstlisting} [style=mymatlab]
% load rigid transformation parameters from the file into matlab
params = load('max.out')';
scale  = params(1);
phi    = params(2);
tx     = params(3);
ty     = params(4);
mirr   = -1;
R      = [ cos(phi), mirr*sin(phi); ...
          -sin(phi), mirr*cos(phi)];
T      = [tx ty]';

%% plot the two grids on top of each other
mapped = zeros(100,2);
for i=11:100 % camera coordinates into display coordinates
  mapped(i,:) = (scale*R*q(i,:)'+T)';
end

dpos = zeros(100,2);
for i=0:99 % calculate display points
  dpos(i+1,1) = 400+50*mod(i,10);
  dpos(i+1,2) = 500+50*floor(i./10);
end

hold off; plot(dpos(:,1),dpos(:,2),'.'); hold on;
plot(mapped(11:end,1),mapped(11:end,2),'r+');
\end{lstlisting}

% print -depsc2 /home/martin/thesis/kielhorn/rigid/rigid-compare

%%% Local Variables: 
%%% mode: latex
%%% TeX-master: "kielhorn_memi"
%%% eval: (reftex-mode)
%%% eval: (flyspell-mode)
%%% End: 
