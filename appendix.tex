\chapter{EM-CCD camera calibration}
\section{Andor Basic code listing for automatic image acquisition}
\label{sec:basic-acquisition}
\definecolor{light-gray}{gray}{0.95}

\lstdefinestyle{myframe}{
  basicstyle=\LSTfont,
  % basicstyle=\footnotesize\ttfamily,
  rulesepcolor=\color{gray} ,
  rulecolor = \color{black},
  frame = single,
  % framerule = 0pt,
  backgroundcolor =\color{light-gray}, 
  fontadjust=true,
  breaklines = true,
  showstringspaces=false,
}
\lstdefinestyle{mybasic}{
  language=[Visual]Basic,
  commentstyle=\itshape,
  style=myframe
}
\lstdefinestyle{mypython}{
  language=Python,
  showstringspaces=false,
  tabsize=4,
  commentstyle=\itshape,
  basicstyle=\ttfamily,
  morekeywords={models, lambda, forms},
  frame = single,
  breaklines = true,
  style=myframe
}


\begin{lstlisting}[style=mybasic]
' This is code for the Basic interpreter in Andor Solis
function ~GetSaturatingExposure()
        SetKineticNumber(1)
        exp=.01
        SetExposureTime(exp)
        run()
        m=maximum(#0,1,512)
        GetSaturatingExposure=exp*10000/(m-100)
        CloseWindow(#0)
return
name$ = "C:\Users\work\Desktop\martin\20111111\scan-em3\ixon_"
print("start")

SetOutputAmp(1)
print("conv_start")
exp= ~GetSaturatingExposure()
print(exp)
SetExposureTime(exp)
SetKineticNumber(20)
SetShutter(0,1)
run()
save(#0,name$ + "conv1_dark.sif")
ExportTiff(#0, name$ + "conv1_dark.tif", 1, 1, 0, 0)
CloseWindow(#0)
CloseWindow(#1)
        
SetShutter(1,1)
run()
save(#0,name$ + "conv1_bright.sif")
ExportTiff(#0, name$ + "conv1 _bright.tif", 1, 1, 0, 0)
CloseWindow(#0)
CloseWindow(#1)

SetOutputAmp(0)
SetShutter(1,1)
for i = 40 to 300 step 10
        SetGain(i)
        exp=~GetSaturatingExposure()
        print(exp)
        SetExposureTime(exp)
        SetKineticNumber(20)
        SetShutter(0,1)
        run()
        save(#0,name$ + str$(i) + "_dark.sif")
        ExportTiff(#0, name$ + str$(i) + "_dark.tif", 1, 1, 0, 0)
        CloseWindow(#0)
        CloseWindow(#1)
        SetShutter(1,1)
        run()
        save(#0,name$ + str$(i) + "_bright.sif")
        ExportTiff(#0, name$ + str$(i) + "_bright.tif", 1, 1, 0, 0)
        CloseWindow(#0)
        CloseWindow(#1)
next

SetOutputAmp(1)
print("conv_end")
exp= ~GetSaturatingExposure()
print(exp)
SetExposureTime(exp)
SetKineticNumber(20)
SetShutter(0,1)
run()
save(#0,name$ + "conv2_dark.sif")
ExportTiff(#0, name$ + "conv2_dark.tif", 1, 1, 0, 0)
CloseWindow(#0)
CloseWindow(#1)
        
SetShutter(1,1)
run()
save(#0,name$ + "conv2_bright.sif")
ExportTiff(#0, name$ + "conv2 _bright.tif", 1, 1, 0, 0)
CloseWindow(#0)
CloseWindow(#1)
\end{lstlisting}

\section{Python code listing for the read noise evaluation}
\label{sec:python-readnoise-eval}

\begin{figure}
  \centering
  \pdfinput{17cm}{ixon_conv1}
  \pdfinput{17cm}{ixon_300}
  \caption{{\bf top:} Conventional readout of an Andor IXon3
    camera. {\bf bottom:} readout with an EM-gain setting of 300 on
    the same camera with identical sample. {\bf left:} 2D histogram of
    per pixel variances against binned intensities. {\bf middle:}
    variance of 20 dark images. {\bf right:} mean of 20 dark images.}
  \label{fig:ixon}
\end{figure}
  


\begin{lstlisting}[style=mypython]
#!/usr/bin/env python
# ./ti.py /media/backup/andor-ultra-ixon/martin/20111111/scan-em3/ ultra 2700
import sys
import os

import matplotlib
matplotlib.use('Agg')

from pylab import *
from libtiff import TIFFfile, TIFFimage
from scipy import stats

seterr(divide='ignore')

folder = sys.argv[1]
cam = sys.argv[2]
gain = sys.argv[3]

def readpics(gain,cam='ixon_',isdark=False):
    print 'loading ', os.path.join(folder,cam) + '_' + gain + '_bright.tif'
    fg=TIFFfile(os.path.join(folder,cam) + '_' + gain + '_bright.tif')
    bright,bright_names=fg.get_samples()
    bg=TIFFfile(os.path.join(folder,cam) + '_' + gain + '_dark.tif')    
    dark,dark_names=bg.get_samples()
    return (bright[0],dark[0])

(f,b) = readpics(gain=gain,cam=cam)

bg=mean(b,axis=0)
v=var(f,axis=0)
i=mean(f,axis=0)

ny,nx=64,128
H,y,x=histogram2d(v.flatten(),i.flatten(),bins=[ny,nx],
                  range=[[0,v.max()],[0,i.max()]])
extent = [x[0], x[-1], y[0], y[-1]] 
acc=zeros(x.shape,dtype=float64)
accn=zeros(x.shape,dtype=int64)
s=nx/i.max()
for ii,vv in nditer([i,v]):
    p=round(ii*s)
    acc[p]+=vv
    accn[p]+=1   

fig=figure(figsize=(24, 8),dpi=300)
hold(False)
title('bal')
subplot(1,3,1)
imshow(log(H), extent=extent,
           aspect='auto', interpolation='none',origin='lower')
hold(True)
ax=x[nonzero(accn)]
ay=acc/accn
ay=ay[nonzero(accn)]
l=round(.6*len(ax))
bx=ax[0:l]
by=ay[0:l]
plot(ax,ay,'r+')
slope,intercept,rval,pval,stderr=stats.linregress(bx,by)
plot(ax,polyval([slope,intercept],ax))
xlabel('intensity/ADU')
ylabel(r'variance/ADU$^2$')
real_gain=1/slope # unit electrons/ADU
read_noise=sqrt(var(b))*real_gain # electrons RMS per pixel
mean_elecs=(mean(f)-mean(b))*real_gain # photoelectrons electrons per pixel
print gain,cam,real_gain,read_noise,mean_elecs,mean(b),rval,pval,stderr
tit='EM-gain: %s, cam: %s, real gain: %.2f e/ADU\n
read noise: %.2f e RMS/pixel, mean: %.2f e/pixel, offset: %.2f'
% (gain,cam,real_gain,read_noise,mean_elecs,mean(b))
title(tit)
subplot(1,3,2)
imshow(var(b,axis=0))
title('variance of darkimages')
colorbar()
subplot(1,3,3)
imshow(mean(b,axis=0))
title('mean of darkimages')
colorbar()
show()
fig.savefig(cam+'_'+gain+'.png')
\end{lstlisting}
