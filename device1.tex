
\chapter{Device 1: prototype for spatio-angular illumination}
\begin{summary}
   - Im vorhergehende Kapitel haben wir das dem spatio-angularen
     Mikroskop zugrundliegende Konzept dargestellt. Hier gehen wir auf
     zusaetzliche Details ein, die fuer die praktische Implementierung
     wichtig sind. Unter anderem die Eigenschaften der beiden
     verwendeten Displays, elektronische Synchronisation der
     verschiedenen Komponenten und einem Algorithmus, um das             % /Hier mehr spezifische Probleme/
     Koordinatensystem der Kamerapixel und der Pixel des focal plane
     SLM ineinander zu transformieren.

   - Das pupil plane SLM wurde durch unseren Partner Fraunhofer IPMS
     waehrend des Projekts neu entwickelt.  Daher widmen wir uns diesem   % /MMA kommt spaeter extra/
     Subsystem im Kapitel (FIXME) naeher.
\end{summary}
\section{Description of the optical components}
So far we have only shown the beam path for transmissive displays (in
\figref{fig:memi-simple}). Such SLM only have a very low transmission
in practice. Therefore we use reflective displays in our prototype.

In \figref{fig:memi-real} I adjusted the beam path accordingly. This
schematic also depicts the optics we use to adapt light from the laser
to fill the etendue of our system. The light source enters the system
from the bottom left. The optic components are color corrected and
have anti-reflex coating for wavelengths in the range from
\unit[400]{nm} to \unit[700]{nm}.

The system successively illuminates the pupil plane SLM---a grayscale
micromirror array developed by our project partner Fraunhofer IPMS
Dresden---and the focal plane SLM, a commercial binary liquid crystal
on silicon display.
 
I gathered some of the following details from the documents that were
created during the development of our prototype and are classified as
confidential. I have summarized the key decisions here and the
relevant project partners have agreed to the publication (FIXME not
finished).


\begin{figure}[!htbp]
  \centering
  \svginput{2}{memi-real}
  \caption{Schematic of the light path through our microscope. Laser
    light enters from the lower left, is scrambled and homogenized to
    illuminate the pupil plane SLM in $\textrm{P}''$ and the focal
    plane SLM in F'. $F$ is the field plane in the sample and its
    primed versions are conjugated planes. $P$ is the pupil of the
    objective. $B_0$ and $B_1$ are adjustable circular apertures. PBS
    is a polarizing beam splitter. DBS is a dichromatic beam splitter.
    The red boxes deliminate subsystems of the illumination system:
    {\bf A:} light scrambling and homogenization, {\bf B:}
    Fourier-optical filter to provide intensity modulating pupil plane
    SLM. {\bf C:} Polarization based intensity modulator as focal
    plane SLM. {\bf D:} Wide-field fluorescence microscope with
    detection path. (FIXME finish diagram, don't use B twice)}
  \label{fig:memi-real}
\end{figure}

\subsection{Ensuring homogeneous illumination}
A quantitative evaluation of our experiments (FIXME ref sec:results)
with different illumination patterns is simplified when both pupil
plane SLM and focal plane SLM are uniformly illuminated.

We use either a laser\footnote{Lasever LSR473H, diode-pumped solid
  state laser, output power 600mW, $\lambda=\unit[473]{nm}$} or an
light emitting diode (LED) \nomenclature{LED}{light emitting diode} as
the light source in our experiments. Below we discuss optical measures
that attain homogeneity of the illumination of both displays.

The LED\footnote{Huey Jann HPB8-48KBD, wavelength
  $\unit[(463\pm1)]{nm}$, brightness \unit[35]{lm}, view angle
  $120{}^\circ$, FIXME TODO: Flaeche messen} we use has a large active
area. Due to etendue mismatch a relatively large amount of its
produced light will never reach the sample. But it is easy to achieve
a homogeneous illumination. Moreover, the LED can be quickly switched
on and off electronically \footnote{The DPSS Laser doesn't allow fast
  direct electronic switching at full power. We have to use an
  acousto-optic modulator connected with the additional expense of its
  optical alignment (FIXME siehe spaetere ref section).}.

Unlike an LED, a laser delivers light of considerably higher spectral
radiance ($\unit[]{W/(sr\, m^2 m)}$). Thus it is in principle possible
to use the laser as a highly efficient light source for our
system. Unfortunately, the high spectral and spatial coherence of a
laser often lead to high-contrast fluctuations of the irradiance and
we have to compensate for this by time averaging.

When using the Laser, we send its parallel Gaussian beam into a
bundle\footnote{Fiber bundle with circular cross-section (Loptek,
  Berlin, DE), \unit[1.1]{mm} diameter and \unit[2]{m} length. The
  beam broadening is $3{}^\circ$ and increases, when the bundle is
  bent \citep{D8.4}.}  of randomly distributed fibers. This randomizes
the light distribution at the bundle output and also broadens the
illumination angles.

A relay system (A1) images the circular output of the fiber bundle
onto the entrance of a light pipe. This relay system contains a
rotating microlens array\footnote{Array of cross-oriented cylindrical
  lenses on both sides with a pitch of \unit[0.5]{mm} resulting in an
  effective focal length of \unit[6.9]{mm} (LIMO, Dortmund, DE).}. It
is driven by a motor with the axis of rotation being diplaced from the
optical axis. This time-varying element allows to reduce speckle.

Both, the fiber bundle and microlens array, increase the etendue of
the laser illumination to the optimum value, which is given by one of
our SLM as discussed below in \ref{sec:etendue} (FIXME ref). 

The light  pipe is  a hollow  mirror-integrator tunnel  with quadratic
cross-section and depicted in \figref{fig:integrator-rod}. The mixing
effect of the  tunnel can be understood by  considering the irradiance
in the plane of the tunnel output as it would occur without tunnel.

Drawing the outline of the square cross-section into this irradiance
map selects the light that directly reaches this plane.  Surrounding
this outline with the four squares that touch its edges selects the
light that will reach the output plane after one reflection. The
irradiance maps from neighbouring squares are mirrored and added to
the direct illumination. Depending on the numerical aperture of the
input light, more reflections may occur --- resulting in the addition
of irradiance from next-nearest-neighbours and so forth.

This improves the uniformity of the light distribution in the output
plane without altering the numerical aperture of the light.  The more
subregions are superimposed, the better will be the uniformity.
Assuming $N$ subregions were overlaid and their contributions were
statistically independent, then according to the central limit theorem
the standard deviation of the irradiance is proportional to
$1/\sqrt{N}$ \citep{Koshel2012}.

However, we also align the source distribution to be rotationally
symmetric about the optical axis and obtain an even more uniform
output than this prediction because positive and negative slopes from
different subregions compensate in the superposition (also
\cite{Koshel2012}).

In our system the side length of the cross-section of the tunnel is
\unit[2.5]{mm} and its length of \unit[250]{mm} ensures enough
reflections for homogeneous illumination. A relay system magnifies the
tunnel output to $\unit[4]{mm}\times\unit[4]{mm}$ in the plane
$\textrm{F}'''$.

We thought about using the output of the tunnel directly, without the
additional relais system, but then the length of the tunnel would have
been prohibitively long.

\jpginput{8cm}{integrator-rod}{Hollow mirror-integrator tunnel with a
  quadratic cross section of \unit[2.5]{mm} side length and
  \unit[250]{mm} length.}

Regarding the two relay systems the optical designer at In-Vision
commented (FIXME ref D8.9) that these have not been optimized for
prefect imaging but for the transport of the homogeneous light
distribution. The system A1 at the tunnel entrance has to transfer the
illumination from the round fiber end to the square tunnel
entrance. The engineer designed a good quality system with only two
lenses (and the microlens array). At the other end (A2 in
\figref{fig:memi-real}) five elements carry the light from the tunnel
exit into $\textrm{F}'''$.

During the planning phase we also considered a homogenization design
based on a fly's eye condensor (two consecutive microlens
arrays). According to simulations performed by In-Vision, this,
however, would have been more difficult to adjust than the tunnel. In
particular the system would have been more dependent on illumination
wavelength.

In summary the following points are important in order to achieve
homogeneous illumination of focal and pupil plane with the tunnel:
\begin{itemize}
\item The image of the end of the bundle should properly cover the
  tunnel entrance. Especially the corners of the tunnel should not be
  darker than the center. Inhomogeneous illumination at the tunnel
  entrance leads to inhomogeneous illumination of the pupil plane SLM.
\item The end of the fibre bundle must be adjusted in four axes
  (centering and angle).
\item The focal length of the microlenses should be chosen shorter
  than predicted by pure etendue calculation. In order to compensate
  inevitably occuring microchipping on the edges of the cemented glass
  mirrors.
\end{itemize}

\subsection{ Fourier-optischer Filter zur Kontrasterzeugung am pupil plane SLM}
  - Der micro-mirror array, den wir als pupil plane SLM einsetzen,        % /MMA torsion spiegel/
    besteht aus Torsionsspiegeln, die die Phase des Lichts modulieren
    (fuer eine genauere Beschreibung siehe spaeteres Kapitel               
    FIXME). Um damit eine Intensitaetsmodulation zu bewirken, nutzen
    wir den in Fig 4.2 B gezeigten Fourier filter. 

  - Die Linse L1 hat zwei Aufgaben: Zum einen bildet sie die Feldmaske   % /Schlierenoptiklinse/
    B0 in den Feldstopp B1 ab. Zum anderen wird die Ebene P'' mit dem
    SLM nach unendlich abgebildet.

  - Bei ungekippten Spiegeln, wird somit F''' nach F'' abgebildet und    % /MMA Kontrasterzeugung/
    gleichzeitig gibt es ein scharfes Bild von P'' nach P'. Beide
    Ebenen F'' und P' sind dann homogen ausgeleuchtet.

  - Werden die Spiegel auf der linken Haelfte in P'' gekippt, dann
    lenken sie das Licht entlang der gestrichelten Linie (in Fig 4.1)
    ab. Dieses Licht wird von der Apertur B1 absorbiert und steht dann
    nicht in P' zur verfuegung. D.h. die rechte Seite in P' ist
    dunkel. Der gesamte radiant flux ($\unit[]{W}$) durch die Apertur in
    F'' nimmt ab, die irradiance ($\unit[]{W/m^2}$) ueber die Apertur
    bleibt aber homogen.

  - Im realen System besteht die Linse L1 aus 4 Elementen. Aufgrund
    der Symmetrie weist sie keinen axialen Farbfehler auf. Es bleibt
    jedoch ein kleiner lateraler Farbfehler (FIXME genauer ergruenden
    was das bedeutet).
 

\subsection{ Relais-System zwischen pupil plane und focal plane SLM}
  - Die Linsen L2 und L3 bilden ein doppelt telezentrisches             % /Relais-System/
    Relais-System mit Vergroesserung 2 und bilden F'' auf der Ebene
    des focal plane SLM in F' ab. Gleichzeitig bildet dieses
    Relais-System den pupil plane SLM von P'' nach unendlich ab.
 
  - Prinzipiell koennte man auch den focal plane SLM in F'' an Stelle
    der Apertur B1 platzieren. In unserem Prototypen haben wir uns
    jedoch fuer dieses zusaetzliche Relais-System entschieden, um den
    Kontrast beider SLM voneinander zu entkoppeln.

   - TODO warum haben wir das relay system? 
     - vermutlich weil wir den mma kontrast vom lcos entkoppeln wollen
     - es ist natuerlich fuer sammelnde system, dass axial color sich
       aufaddiert und nicht kompensiert wird


\subsection{ Polarisationsbasierte Kontrasterzeugung am focal plane SLM}
  - Der von uns verwendete focal plane SLM ist ein liquid crystal on
    silicon Geraet, dass die Polarisation des reflektierten Lichts
    entweder um 90 grad dreht oder konstant laesst.
 
  - Ein Polarisationsstrahlteiler erzeugt daraus einen binaeren
    Intensitaetskontrast (siehe Fig 4.1 C).

  - Wir haben uns fuer einen wire-grid Polarizer (Moxtek PBF02C, Orem,
    UT, US) entschieden, weil die Platte weniger Rueckreflexe
    verursacht als ein Strahlteilerwuerfel.

  - Die s-Polarisation des eingehenden Lichts wird in Richtung des SLM
    reflektiert. Aktive Pixel des SLM rotieren die Polarisation des
    Lichts um 90 Grad und passiert dann den Strahlteiler als
    p-Polarisation in Tranmission in richtung Mikroskop. Dort befindet
    sich ein zusaetzlicher Cleanup-Analysator im Strahlengang.
 
  - Es waere auch denkbar, SLM und Strahlteiler anders anzuordnen, so
    dass das vom SLM kommende Licht in das Mikroskop
    \emph{reflektiert} wird. In diesem Fall verschlechtert jedoch eine
    ungewollte Oberflaechendurchbiegung des Strahlteilers die
    Abbildungsqualitaet vom focal plane SLM. Deshalb nutzen wir den
    Strahlteiler in Tranmission.

  - Die duenne Platte (<2mm) des Strahlteilers macht das System leicht
    asymmetrisch und fuehrt damit hauptsaechlich zu Astigmatismus und
    lateral color (ref D8.9 FIXME), das Optikdesign bleibt aber
    beugungsbegrenzt.


\jpginput{14cm}{setup-photo-blueprint}{The wide field epi-fluorescence
  microscope with attached illumination head. The positions of the two
  spatial light modulators (Micro mirror array (MMA) and liquid
  crystal on silicon display (LCoS)) are indicated. Drawing by Josef
  Wenisch (In-Vision, Austria).}


\jpginput{14cm}{memi-setup-only-lenses}{only lenses7}

\subsection{ Variables Teleskop als Tubuslinse}
  - Die groesse der Pupille von Mikroskopobjektiven haengt von deren
    Bildfeld und numerischer Apertur ab. Die letzt Linse
    TL${}_\textrm{ill}$ in unserem Beleuchtungssystem ist daher so
    konzipiert, dass sie P'' mit variabler Vergroesserung nach P
    abbildet. 

  - Die Linse besteht aus drei beweglichen Gruppen und kann somit
    garantieren, dass der pupil plane SLM bei Vergroesserungsaenderung
    stationaer auf der pupil plane des Objektivs abgebildet bleibt und
    gleichfalls der focal plane SLM immer im unendlichen abgebildet
    bleibt (FIXME gibt es ein paper mit begruendung?).





\begin{figure}[!htbp]
   \centering
   \svginput{2}{memi-sketch}
   \caption{Schematic of the lenses in the MEMI system and their focal
     lengths. The focal length $f_\textrm{TL}$ of the tube lens can be
     varied. This allows to scale the second intermediate image
     $r''_\textrm{MMA}$ of the micro mirror array to fit the back
     focal plane of different objectives. Dimensions in mm.}
   \label{fig:memi-sketch}
 \end{figure}




% \imagw{14cm}{mma}{{\bf left:} Scanning electron microscope image of
%   the micro-mirror array (MMA).  The pixel pitch of the device is
%   \unit[0.016]{mm}. The hinges for the tilt movement and the
%   electrodes are clearly visible. {\bf middle:} Optical reflective
%   microscope image of the MMA. {\bf right:} exaggerated rendering of
%   how a 8x8 checker board pattern would be displayed on the
%   device. Electron and optical micrograph by Fraunhofer IPMS Dresden
%   (Germany)}

% \begin{figure}[!hbt]
%   \centering
%   \includegraphics[width=7cm]{mma-plain}
%   \includegraphics[width=7cm]{mma-ill}
%   \caption{{\bf left:} Micro mirror array chip during installation of
%     the optics. {\bf right:}~Illuminated micro mirror array in the
%     aligned system.}
%   \label{fig:mma-closeup}
% \end{figure}
