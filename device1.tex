
\chapter{Device 1: prototype for spatio-angular illumination}
\begin{summary}
   - Im vorhergehende Kapitel haben wir das dem spatio-angularen
     Mikroskop zugrundliegende Konzept dargestellt. Hier gehen wir auf
     zusaetzliche Details ein, die fuer die praktische Implementierung
     wichtig sind. Unter anderem die Eigenschaften der beiden
     verwendeten Displays, elektronische Synchronisation der
     verschiedenen Komponenten und einem Algorithmus, um das             % /Hier mehr spezifische Probleme/
     Koordinatensystem der Kamerapixel und der Pixel des focal plane
     SLM ineinander zu transformieren.

   - Das pupil plane SLM wurde durch unseren Partner Fraunhofer IPMS
     waehrend des Projekts neu entwickelt.  Daher widmen wir uns diesem   % /MMA kommt spaeter extra/
     Subsystem im Kapitel (FIXME) naeher.
\end{summary}
\section{Description of the optical components}
So far we have only shown the beam path for transmissive displays (in
\figref{fig:memi-simple}). Such SLM only have a very low transmission
in practice. Therefore we use reflective displays in our prototype.

In \figref{fig:memi-real} I adjusted the beam path accordingly. This
schematic also depicts the optics we use to adapt light from the laser
to fill the etendue of our system. The light source enters the system
from the bottom left. The optic components are color corrected and
have anti-reflex coating for wavelengths in the range from
\unit[400]{nm} to \unit[700]{nm}.

The system successively illuminates the pupil plane SLM---a grayscale
micromirror array developed by our project partner Fraunhofer IPMS
Dresden---and the focal plane SLM, a commercial binary liquid crystal
on silicon display.
 
I gathered some of the following details from the documents that were
created during the development of our prototype and are classified as
confidential. I have summarized the key decisions here and the
relevant project partners have agreed to the publication (FIXME not
finished).


\begin{figure}[!htbp]
  \centering
  \svginput{2}{memi-real}
  \caption{Schematic of the light path through our microscope. Laser
    light enters from the lower left, is scrambled and homogenized to
    illuminate the pupil plane SLM in $\textrm{P}''$ and the focal
    plane SLM in F'. $F$ is the field plane in the sample and its
    primed versions are conjugated planes. $P$ is the pupil of the
    objective. $B_0$ and $B_1$ are adjustable circular apertures. PBS
    is a polarizing beam splitter. DBS is a dichromatic beam splitter.
    The red boxes deliminate subsystems of the illumination system:
    {\bf A:} light scrambling and homogenization, {\bf B:}
    Fourier-optical filter to provide intensity modulating pupil plane
    SLM. {\bf C:} Polarization based intensity modulator as focal
    plane SLM. {\bf D:} Wide-field fluorescence microscope with
    detection path. (FIXME finish diagram, don't use B twice)}
  \label{fig:memi-real}
\end{figure}

\subsection{Ensuring homogeneous illumination}
A quantitative evaluation of our experiments (FIXME ref sec:results)
with different illumination patterns is simplified when both pupil
plane SLM and focal plane SLM are uniformly illuminated.

We use either a laser\footnote{Lasever LSR473H, diode-pumped solid
  state laser, output power 600mW, $\lambda=\unit[473]{nm}$} or an
light emitting diode (LED) \nomenclature{LED}{light emitting diode} as
the light source in our experiments. Below we discuss optical measures
that attain homogeneity of the illumination of both displays.

The LED\footnote{Huey Jann HPB8-48KBD, wavelength
  $\unit[(463\pm1)]{nm}$, brightness \unit[35]{lm}, view angle
  $120{}^\circ$, FIXME TODO: Flaeche messen} we use has a large active
area. Due to etendue mismatch a relatively large amount of its
produced light will never reach the sample. But it is easy to achieve
a homogeneous illumination. Moreover, the LED can be quickly switched
on and off electronically \footnote{The DPSS Laser doesn't allow fast
  direct electronic switching at full power. We have to use an
  acousto-optic modulator connected with the additional expense of its
  optical alignment (FIXME siehe spaetere ref section).}.

Unlike an LED, a laser delivers light of considerably higher spectral
radiance ($\unit[]{W/(sr\, m^2 m)}$). Thus it is in principle possible
to use the laser as a highly efficient light source for our
system. Unfortunately, the high spectral and spatial coherence of a
laser often lead to high-contrast fluctuations of the irradiance and
we have to compensate for this by time averaging.

When using the Laser, we send its parallel Gaussian beam into a
bundle\footnote{Fiber bundle with circular cross-section (Loptek,
  Berlin, DE), \unit[1.1]{mm} diameter and \unit[2]{m} length. The
  beam broadening is $3{}^\circ$ and increases, when the bundle is
  bent \citep{D8.4}.}  of randomly distributed fibers. This randomizes
the light distribution at the bundle output and also broadens the
illumination angles.

A relay system (A1) images the circular output of the fiber bundle
onto the entrance of a light pipe. This relay system contains a
rotating microlens array\footnote{Array of cross-oriented cylindrical
  lenses on both sides with a pitch of \unit[0.5]{mm} resulting in an
  effective focal length of \unit[6.9]{mm} (LIMO, Dortmund, DE).}. It
is driven by a motor with the axis of rotation being diplaced from the
optical axis. This time-varying element allows to reduce speckle.

Both, the fiber bundle and microlens array, increase the etendue of
the laser illumination to the optimum value, which is given by one of
our SLM as discussed below in \ref{sec:etendue} (FIXME ref). 

The light  pipe is  a hollow  mirror-integrator tunnel  with quadratic
cross-section and depicted in \figref{fig:integrator-rod}. The mixing
effect of the  tunnel can be understood by  considering the irradiance
in the plane of the tunnel output as it would occur without tunnel.

Drawing the outline of the square cross-section into this irradiance
map selects the light that directly reaches this plane.  Surrounding
this outline with the four squares that touch its edges selects the
light that will reach the output plane after one reflection. The
irradiance maps from neighbouring squares are mirrored and added to
the direct illumination. Depending on the numerical aperture of the
input light, more reflections may occur --- resulting in the addition
of irradiance from next-nearest-neighbours and so forth.

This improves the uniformity of the light distribution in the output
plane without altering the numerical aperture of the light.  The more
subregions are superimposed, the better will be the uniformity.
Assuming $N$ subregions were overlaid and their contributions were
statistically independent, then according to the central limit theorem
the standard deviation of the irradiance is proportional to
$1/\sqrt{N}$ \citep{Koshel2012}.

However, we also align the source distribution to be rotationally
symmetric about the optical axis and obtain an even more uniform
output than this prediction because positive and negative slopes from
different subregions compensate in the superposition (also
\cite{Koshel2012}).

In our system the side length of the cross-section of the tunnel is
\unit[2.5]{mm} and its length of \unit[250]{mm} ensures enough
reflections for homogeneous illumination. A relay system magnifies the
tunnel output to $\unit[4]{mm}\times\unit[4]{mm}$ in the plane
$\textrm{F}'''$.

We thought about using the output of the tunnel directly, without the
additional relais system, but then the length of the tunnel would have
been prohibitively long.

\jpginput{8cm}{integrator-rod}{Hollow mirror-integrator tunnel with a
  quadratic cross section of \unit[2.5]{mm} side length and
  \unit[250]{mm} length.}

Regarding the two relay systems the optical designer at In-Vision
commented (FIXME ref D8.9) that these have not been optimized for
prefect imaging but for the transport of the homogeneous light
distribution. The system A1 at the tunnel entrance has to transfer the
illumination from the round fiber end to the square tunnel
entrance. The engineer designed a good quality system with only two
lenses (and the microlens array). At the other end (A2 in
\figref{fig:memi-real}) five elements carry the light from the tunnel
exit into $\textrm{F}'''$.

During the planning phase we also considered a homogenization design
based on a fly's eye condensor (two consecutive microlens
arrays). According to simulations performed by In-Vision, this,
however, would have been more difficult to adjust than the tunnel. In
particular the system would have been more dependent on illumination
wavelength.

In summary the following points are important in order to achieve
homogeneous illumination of focal and pupil plane with the tunnel:
\begin{itemize}
\item The image of the end of the bundle should properly cover the
  tunnel entrance. Especially the corners of the tunnel should not be
  darker than the center. Inhomogeneous illumination at the tunnel
  entrance leads to inhomogeneous illumination of the pupil plane SLM.
\item The end of the fibre bundle must be adjusted in four axes
  (centering and angle).
\item The focal length of the microlenses should be chosen shorter
  than predicted by pure etendue calculation. In order to compensate
  inevitably occuring microchipping on the edges of the cemented glass
  mirrors.
\end{itemize}

\subsection{Fourier optical filter for contrast generation on pupil
  plane SLM}

The micromirror array, which we use as a pupil plane SLM consists of
torsion mirrors that modulate the phase of the light (for a more
detailed description see ref FIXME). In order to modulate the
intensity we use the Fourier filter shown in \figref{fig:memi-real}~B.

The lens L1 has two purposes: First, it images the field mask B0 into
the field stop B1. Second, the plane $\textrm{P}''$ with the phase SLM
is imaged to infinity.

With undeflected micromirrors, the SLM has no signifcant effect and
works like a plane mirror. Both planes $\textrm{F}''$ and
$\textrm{P}'$ are then homogeneously illuminated.

If the left half of the micromirrors are tilted, then they direct the
light along the dashed line in \figref{fig:memi-real}. This light is
absorbed by the field stop B1 and therefore missing in $\textrm{P}'$,
i.e.\ the right side in $\textrm{P}'$ is dark. The total radiant flux
($\unit[]{W}$) through the beam stop in $\textrm{F}''$ decreases while
the transmitted irradiance ($\unit[]{W/m^2}$) remains homogeneous.

In the real system, the lens L1 consists of four elements.

\subsection{Relay optics between pupil plane and focal plane SLM}
The lenses L2 and L3 form a double-telecentric relay system with
magnification 2 and image $\textrm{F}''$ onto the focal plane SLM in
$\textrm{F}'$. At the same time these lenses make sure that the pupil
plane SLM is imaged to infinity.
 
The relay system ensures that the pixels of the focal plane SLM are at
the resolution limit, while the pupil plane SLM fills the pupil.

In addition, the relay system enables a simpler mechanical realization
and good contrast. It would already be difficult to accommodate the
focal plane SLM and polarization beam splitter in $\textrm{F}''$ ---
including an adjustable aperture would probably not be feasible at
all.


\subsection{ Contrast generation on focal plane SLM using
  polarization}
The SLM we use to control the focal plane illumination is a
ferroelectric liquid crystal on silicon device (ForthDD WXGA R3,
UK). Depending on if a pixel is off or om, the returning light either
retains the polarization of the input light or rotates it by 90
degrees.

From this, a polarization beam splitter generates a binary intensity
contrast (see \figref{fig:memi-real}~C). 

We opted for a wire-grid polarization beam splitter (Moxtek PBF02C,
Orem, UT, US) because they ensure a high enough optical quality, good
contrast and the plate causess less back reflections than a beam
splitter cube.


- ignacio moreno 2009 operational modes of a ferroelectric lcos
(FIXME)

The s-polarized component of the incoming light is reflected towards
the SLM. Active pixels of the SLM rotate the polarization of light by
90 degrees and then passes through the beam splitter as p-polarization
in the direction of the microscope. There is a supplementary cleanup
analyzer in the beam path. 

It would also be conceivable to arrange SLM and beam splitter
differently, so that the light coming from the SLM is \emph{reflected}
towards microscope. In this case, however, unwanted bending of the
beam splitter's surface will deteriorate the image quality of the
focal plane SLM. Therefore, we use the beam splitter in transmission.


The beam splitter plate makes the overall optics slightly assymmetric
and thus induces mainly astigmatism and lateral color (ref FIXME
D8.9). The plate is thin enough (thickness \unit[0.7]{mm}), so that
the design remains diffraction limited.

 	
\jpginput{14cm}{setup-photo-blueprint}{The wide field epi-fluorescence
  microscope with attached illumination head. The positions of the two
  spatial light modulators (pupil plane SLM: micromirror array (MMA)
  and focal plane SLM: liquid crystal on silicon display (LCoS)) are
  indicated. Drawing by Josef Wenisch (In-Vision, Austria).}
	  	
\jpginput{14cm}{memi-setup-only-lenses}{only lenses7}


\subsection{ Variable telescope as tube lens}
Microscope objectives come with various pupil diameters. The last lens
$\textrm{TL}_\textrm{ill}$ in our illumination system has been
designed as a variable zoom objective, that maps the pupil plane SLM
from $\textrm{P}''$ with variable magnification to $\textrm{P}$.

Unlike conventional zoom telescopes we use three movable lens groups
to guarantee that the image of the pupil plane SLM remains stationary
and simultaneously focal plane SLM stays imaged into infinity while
changing the magnification.

 - email mit erhard ipp
  - Du gehst jetzt also mit linear polarisiertem Laser direkt in den Lichtmischtunnel?
  - Ja. Der Laser wird an zwei Metall-Spiegeln M1 und M2)
     reflektiert. Vor dem Mikrolinsenarray messe ich
   - (1.885+/-0.005) mW ohne Polarisator, (1.580+/-0.001) mW mit
     Polarisator in Maximalstellung und
   - (27.26+/-0.01) uW mit Polarisator in Minimumstellung.  Der
     Kontrast ist 1.8e-3/27e-6=70:1.


\section{Electronic control of the component}

Both spatial light modulators can run at most with $50\%$ duty
cycle. Therefore it is necessary to synchronize the displays. Their
controllers allow to upload several hundred frames of image data
before an experiment and keep them in local storage. Images can then
be selected by fast function calls over USB (LCoS) or Ethernet (MMA).

The camera (Clara, Andor PLC, Belfast, Northern Ireland) as the
slowest device is chosen as the master. The camera provides two TTL
outputs. The output ``fire'' is high while the camera is
integrating. The output ``shutter'' goes high \unit[1]{ms} before
``fire'' and provides enough time (\unit[$>850$]{$\mu$s}) for the MMA
controller to tilt and let the mirrors settle.

The LCoS controller can display its images only for certain discrete
times (\unit[20]{ms}, \unit[10]{ms}, \unit[5]{ms}, \unit[200]{$\mu$s})
and it is not straight forward to change this via USB
interface. Therefore we always work with a fixed LCoS display time of
\unit[20]{ms}. The ``fire'' output of the camera also switches the
laser on using an acousto-optic modulator (AOM).

When the z-stage is used, the camera is stopped until the stage has
reached its target position.

%\begin{figure}[H]
%  \centering
%  \svginput{memi-electronics}
%  \caption{The camera triggers both spatial light modulators with its
%    TTL outputs. The acousto-optic modulator sends light into the
%    system during camera integration.}
%  \label{fig:memi-electronics}
%\end{figure}








\begin{figure}[!htbp]
  \centering
  \svginput{2}{memi-sketch}
  \caption{Schematic of the lenses in the MEMI system and their focal
    lengths. The focal length $f_\textrm{TL}$ of the tube lens can be
    varied. This allows to scale the second intermediate image
    $r''_\textrm{MMA}$ of the micro mirror array to fit the back
    focal plane of different objectives. Dimensions in mm.}
  \label{fig:memi-sketch}
\end{figure}




% \imagw{14cm}{mma}{{\bf left:} Scanning electron microscope image of
%   the micro-mirror array (MMA).  The pixel pitch of the device is
%   \unit[0.016]{mm}. The hinges for the tilt movement and the
%   electrodes are clearly visible. {\bf middle:} Optical reflective
%   microscope image of the MMA. {\bf right:} exaggerated rendering of
%   how a 8x8 checker board pattern would be displayed on the
%   device. Electron and optical micrograph by Fraunhofer IPMS Dresden
%   (Germany)}

% \begin{figure}[!hbt]
%   \centering
%   \includegraphics[width=7cm]{mma-plain}
%   \includegraphics[width=7cm]{mma-ill}
%   \caption{{\bf left:} Micro mirror array chip during installation of
%     the optics. {\bf right:}~Illuminated micro mirror array in the
%     aligned system.}
%   \label{fig:mma-closeup}
% \end{figure}
