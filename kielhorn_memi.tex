
%\documentclass[DIV19]{scrartcl}
%\usepackage[paper size={90mm, 120mm},left=2mm,right=2mm,top=2mm,bottom=2mm,nohead]{geometry}
% FIXME try prettyref
\documentclass[oneside,a4paper,12pt,BCOR20mm,DIV14]{scrbook} % should be DIV14
%\documentclass{book}
% this gives a bit more than 2cm margin right and 4cm left
% koma-script.pdf: A4 is 210mmx297mm, BCOR is substraced before the page
% width is divided into DIV parts (HLU), a one sided leaves 1.5 HLU
% HLU*DIV=210-BCOR -> DIV=(210-BCOR)/HLU
% I want BCOR= 20mm 1.5 HLU = 20 mm 
% -> DIV=truncate(190*1.5/20) = truncate(14.25)=14
% I could use headinclude so that the header isn't printed into the margin

% Initially two softbound theses should be submitted to the
% Examinations Office for the examiners. Softbound theses should have
% the pages glued in.
% They don't need gold lettering on the spine.

%\includeonly{spatio-angular}



\usepackage[utf8]{inputenc}
\usepackage[T1]{fontenc}
\usepackage[usenames,dvipsnames]{color}
\usepackage[onehalfspacing]{setspace} 
\usepackage{graphicx}
\usepackage{longtable}
\usepackage{float}
\usepackage{wrapfig}
\usepackage{soul}
\usepackage{amssymb}
\usepackage{amsmath}

\usepackage{pdfcomment}

% choose font here: http://www.tug.dk/FontCatalogue/mathfonts.html
%\usepackage[math]{iwona}
\usepackage[math]{kurier}
%\usepackage{cmbright}
%\usepackage[T1]{fontenc}



\usepackage[pdftex]{hyperref} 

\pdfcompresslevel=0
%\newcont\pdfcompresslevel


%\usepackage[hypertex,breaklinks]{hyperref} 
% breaklinks only seems to work with dvipdfm,
% otherwise urls have no
% line breaks

\usepackage[showrefs,showcites,ignoreunlbld]{refcheck} % for draft, uncomment for final

\usepackage[disable]{todonotes} % for draft, disable for final

\usepackage{lineno}
\usepackage[refpage]{nomencl}
%\special{background Black}\special{color Green}
%\usepackage[utf8x]{inputenc} 
%\usepackage[T2A]{fontenc} % for the russian reference
\usepackage{wasysym} %diameter
% http://www.andy-roberts.net/misc/latex/latextutorial3.html

%\usepackage{url} % natbib.pdf p.11 break urls up, seems to be done
                 % with hyperref, though

\usepackage{natbib}

\usepackage{siunitx} %sudo apt-get install texlive-science
\usepackage{units}


% for app_hilo
\usepackage{listings}
\usepackage{color}
\usepackage{textcomp}
\usepackage{subfigure}


% \listfiles % show which files are loaded by tex

\bibpunct{(}{)}{;}{a}{}{,}
\makenomenclature
\newcommand{\vect}[1]{\mathbf{#1}}
\renewcommand{\r}{\vect r}
\renewcommand{\a}{\vect a}
\newcommand{\s}{\vect s}
\newcommand{\vnu}{\mbox{\boldmath{$\nu$}}}
\newcommand{\vtau}{\mbox{\boldmath{$\tau$}}}
\newcommand{\vrho}{\mbox{\boldmath{$\rho$}}}
\newcommand{\vvarrho}{\mbox{\boldmath{$\varrho$}}}
\newcommand{\supp}{\mathop{\mathrm{supp}}}
\newcommand{\diag}{\mathop{\mathrm{diag}}}
\def\k{\vect k}
\def\d{\vect d}
\def\e{\vect e}
\def\f{\vect f}
\def\c{\vect c}
\def\x{\vect x}
\def\y{\vect y}
\def\z{\vect z}
\def\q{\vect q}
\def\p{\vect p}
\def\l{\vect l}

\newcommand{\nvect}[1]{\vect{\hat{#1}}}
%\renewcommand{\i}{\nvect i}
\newcommand{\vi}{\nvect \i}
\def\hc{\nvect c}
\def\hs{\nvect s}
\def\hd{\nvect d}
\def\hx{\nvect x}
\def\hy{\nvect y}

\def\hz{\nvect z}
\def\n{\nvect n}
\def\t{\nvect t}
\def\m{\nvect m}
\def\vrho{\boldsymbol\rho}
\def\abs#1{\mathopen| #1 \mathclose|}


\renewcommand{\O}{\textsf{O}} % oxygen

% conclusions of paragraphs in the margin
%\usepackage[marginparwidth=2.5cm]{geometry}
\setlength{\marginparwidth}{2.5\marginparwidth}
\reversemarginpar
\newcommand{\cma}[1]{\marginpar{\small #1}}
% include eps or pdf file, that was generated by inkscape, depending
% on if pdflatex or latex processes this file. latex allows a faster
% development cycle but pdflatex generates a smaller and better final
% pdf output
\def\svgending{\ifx\pdfoutput\undefined% 
  .eps_tex% 
  \else%
  .pdf_tex%
  \fi}

% use \svginput{1}{bla} to include bla.svg, make sure you keep this in
% one line, so that make can automatically find the dependencies with
% sed
\newcommand{\svginput}[2]{{\def\svgscale{#1}\input{#2\svgending}}}

\def\pdfending{\ifx\pdfoutput\undefined% 
  .vector.eps% 
  \else%
  .vector%
  \fi}
\newcommand{\pdfinput}[2]{\includegraphics[width=#1]{#2\pdfending}}

% example call \imagw{8cm}{bla.jpg}{bla}{caption abl}
\newcommand{\imagw}[4]{
  \begin{figure}[!hbt]
    \centering
    \includegraphics[width=#1]{#2}
    \caption{#4}
    \label{fig:#3}
  \end{figure}
}

\def\jpgending{\ifx\pdfoutput\undefined% 
  .eps% 
  \else%
  %
  \fi}
% use like this \jpginput{8cm}{imagefile}{caption ...  make sure all
% characters until the opening brace for the caption are on one line
\newcommand{\jpginput}[3]{\imagw{#1}{#2\jpgending}{#2}{#3}}

% this is for plots that are generated by gnuplot
\newcommand{\gnuplotinput}[2]{\begin{figure}[!hbt]%
    \centering%
    \includegraphics{#1_gnuplot}%
    \caption{#2}%
    \label{fig:#1}%
  \end{figure}}

\newcommand{\celegans}{\emph{C.~elegans}}
\DeclareMathOperator{\sign}{sign}
\DeclareMathOperator*{\sinc}{sinc}

% reference to picture
\newcommand{\figref}[1]{\figurename~\ref{#1}}
\title{Spatio-angular microscope} % i don't call make title
\author{Martin Kielhorn}
% short summary at the beginning of a section
\newenvironment{summary}{\begin{quote}\small}{\end{quote}}

\begin{document}
\listoftodos
%\linenumbers
\begin{titlepage}
  
  \hspace{-4cm}
  \svginput{1}{objective-trace}



  \vspace{-5cm}
  
  \hspace{4cm}\textsf{\Huge Spatio--Angular Microscopy}
  
  \vspace{2cm}
  \hspace{6cm}\textsf{\huge PhD Thesis}


  \vspace{3cm}
  \hspace{4cm}\textsf{\Large Martin Kielhorn}
  
  \vspace{1cm}
  \hspace{4cm}\textsf{\Large July 2012}
\end{titlepage}
\newpage
\include{abstract}
\urlstyle{sf}
%\setcounter{tocdepth}{3}
\section*{Preface}
\cite{Matthae2003} %US6504653 (FIXME)
\begin{flushright}
  M.~K.
\end{flushright}

\noindent
Jena, Germany

\noindent
July 2012

\newpage
\tableofcontents
\printnomenclature

\chapter*{preface}
add herbert gross to acknowledgements

hoffentlich kommen bessere prototypen nach mir

die programmierung verschiedenster hardware war ein wesentlicher aspekt
ich habe mir ein beispiel an mikromanager genommen profitiert habe

ich hoffe dass diese arbeit einen grundstein legt.
und dass jemand der damit beginnen will leichter den einstieg fidnet
daher alle meine software auf github

die komplexitaet und stabilitaet wird zwar nicht erreicht
das war aber nicht das ziel

ich wollte schnelles prototyping


\chapter{Introduction}
\label{sec:intro}
\begin{summary}
  In this work I discuss a modification of a fluorescence microscope    \cma{my device}
  that minimizes the toxic effects of the excitation light.

  In the following introductory chapter I describe what phototoxicity   \cma{phototoxicity}
  is and how it comes about. Then I give an example of how it
  influences biological observations in a developing \celegans\ 
  embryo and describe how this particular biological system can be
  used to evaluate and compare the phototoxicity of different microscopes.

  Later in this chapter I give an overview of image formation   \cma{cameras}
  in the wide-field microscope and I describe its principle limitations
  regarding resolution and depth discrimination. Furthermore I
  discuss the two most important current image detector technologies
  --- electron multiplying charge-coupled devices (EMCCD) and
  scientifc complementary metal–oxide–semiconductor (sCMOS).
\end{summary}
Regardless of whether it is the picture of earth captured by an
orbiting satellite, the x-ray motion picture of a running dog or the
time-lapse recording of a blooming flower. Images capture our
imagination and they are a good starting point to develop new models
and theories.

This is particularly true for microscopy.  Only after people became
aware of microorganisms by direct observation, medieval quack could
finally be overcome and modern medicine based on the scientific
method flourished instead.

Even today --- with electron microscopes, magnetic resonance
tomography and sequencing machines --- optical microscopy still is an
indispensable tool for research of living organisms.

Fluorescence microscopy is of particular importance: It enables the     \cma{labelling, switching}
scientist to selectively label a particular type of molecule in living cells
and observe how they perform their biological function.

Besides localizing molecules it is possible to measure physical
quantities inside of the sample. There are, for example, fluorescent
labels that report membrane potentials or viscosity inside of cells.

Finally, it is even possible to exert a controlling function with the
excitation light: There are compounds that locally release chemicals
when illuminated and there are  genetically encoded ion channels that can be
switched by light \citep{Boyden2005}.

However, the excitation light introduces unnatural and potentially
deleterious energy into the specimen. If the exogenous light harms the
observed organism in any way, this effect is called phototoxicity.

% dennoch: from photophysical prospective of single- and multi-photon
% microcopy, probably the most disheartening reality is the occurrence
% of photobleaching and photodamage
% \citep{diaspro2009nanoscopy}

There are a number of techniques that can reduce phototoxicity: Two
photon excitation, controlled light exposure, selective plane
illumination, highly inclined and laminated optical sheet, and oblique
plane microscopy. I introduce them in
chapter \ref{sec:illum-patterns}. These techniques have different
pros and cons and not all are equally suited for a specific problem,
e.g.\ selective plane illumination is very effective, but it needs two
perpendicular lenses and can not be used for multiwell plates or to
observe the liver of a living, adult mouse.

In this work I present an approach that makes use of modern display
and camera technology. We only modify the microscope's illumination
path, the space around objective lens and specimen remains as
accessible as in any conventional wide-field microscope.



\section{Phototoxicity in life sciences and the model organism
  \celegans}
\label{sec:intro-phototoxicity}
The partner in our project who is responsible for decisions related to
life sciences and biology is Institut Pasteur (Paris, FR). They work
on infectious diseases. 

In order to motivate the importance of phototoxicity, I would like to
portray an elegant drug screening experiment which I have seen on one
of my visits in Paris: An automatic microscope continuously images a
cell culture in multiwell plates. These cells carry a pathogen. The
pathogen, the nuclei of the cultured cells and the membranes of the
cells are each stained with a different fluorophore. The cells in each
well of the plates are exposed to a different chemical.

A chemical is considered a hit and will be investigated during further
trials, when the time lapse images show that the culture cells stay
healthy and the number of pathogens decrease. As neither people nor
animals come to harm, this screening experiment is an impeccable
method to systematically understand and hopefully heal certain
diseases. However, this experiment doesn't work very well, if the
excitation light --- and not the drug --- kills the pathogens. The
effect of phototoxicity should therefore be minimized.


Now one would hardly develop a microscope and directly test it with
dangerous pathogens. As part of our collaboration, the Institut
Pasteur therefore developed a safe biological test system that is
relatively easy to maintain \citep{Stiernagle2006} and allows to test
the phototoxicity of various microscopes \citep{Tinevez2012}.




The basis of the system is the embryo of the organism \celegans. These
are small invertebrates. The adult form is approximately \unit[1]{mm}
long.  Their anatomy and development are comparatively simple and have
been well characterized \citep{Sulston1977,Durbin1987}.

\jpginput{12cm}{celegans-devel}{Phototoxic effects while imaging the
  embryonal development of three \celegans\ embryos (strain
  AZ212, histone-2B tagged with eGFP) with different excitation
  intensities. The embryo with lowest excitation dosage (left)
  develops fastest. The embryo with the highest dosage (right) ceases
  development and nearly all fluorophores are bleached after the
  experiment. Images by J.-Y. Tinevez (Institut Pasteur, Paris, FR).}
  

We use embryos of a genetically modified strain\footnote{Our strain
  has WormBase ID AZ212 \citep{Praitis2001}.} that expresses eGFP
tagged histones (enhanced green fluorescent protein, excitation
maximum \unit[488]{nm}, emission maximum \unit[509]{nm}). Histones are
incorporated into the chromatin during cell divisions, i.e.\ the
nuclei of our worms fluoresce green.  The mother worm passes a
sufficient amount of these proteins into the cytoplasm of the
embryo. In the beginning of its development the embryo entirely relies
on this reserve of histones. Only in a much later stage --- certainly
not during the first few hours, that we observe --- it will form its
own histones.

\figref{fig:celegans-devel} compares time-lapse experiments on three \cma{embryo example} 
different \celegans\ embryos with varying
excitation intensities.

The lineage tree of two developing \celegans\ embryos is the \cma{reproducible development}
same.  With all other factors being equal, particularly if the
temperature is constant at $\unit[21\pm
1]{\degreeCelsius}$, two different embryos will develop at the
same speed from egg to fertile adult in three and a half days.


At the beginning of the experiment, embryos are removed from their
mothers at an identical stage, before any cellular divisions have
occured. Then a $z-$stack of the egg with 41 slices and one micron
$z-$sampling is obtained every two minutes.

The columns in \figref{fig:celegans-devel} depict three different embryos
whose development was imaged according to this protocol for two hours
and 38 minutes with different excitation powers.

The figure displays the maximum intensity projections of the
$z-$stacks.  In order to make the cell nuclei visible in all images, I
normalized the data to the same range. As can be guessed from the
photon shot noise, the upper left image contains the least number of
fluorescence photons, and the upper right the most.

An analysis of the time-lapse data show that one hour into the
experiment the embryo with the highest excitation dose (right) has
stopped developing and its fluorophores are strongly bleached.  Some
cells even turned apoptotic and went into programmed cell death.

After two hours and 38 minutes the experiment was stopped and the
embryo which was exposed to the lowest dose (left) has developed the
largest number of cells. The middle embryo ceased developing while the
right embryo died even earlier and nearly all its fluorophores are
bleached at the end of the experiment.

In \figref{fig:worm-integration-time} I reproduce quantitative data
from \cite{Tinevez2012}. Each data point in this graph corresponds to
a two hour time-lapse imaging experiment of a \celegans\ embryo in a
wide-field microscope. From a very low excitation up to a certain
threshold dose the development isn't affected by the light and
approximately 50 cells develop during the two hours.

For a dose above the threshold the development is slowed due to
phototoxicity and the number of cells at the end of the experiment
decreases.

\gnuplotinput{worm-integration-time}{Longer exposure times are less
  phototoxic. Each data point corresponds to one embryo that developed
  under a particular excitation dose for two hours. The solid lines
  are sigmoidal fits to the data. Also indicated are the two
  phototoxicity thresholds given by the inflection point of the
  sigmoid and their $95\%$ confidence intervals. This data was
  provided by J.-Y. Tinevez (Institut Pasteur, Paris, FR) and is also
  published in \cite{Tinevez2012}.}  

The orange data points in the diagram correspond to a per slice
integration time $\tau$ of \unit[100]{ms} and for the green data
the integration time is five times higher.

\nomenclature{$\Omega$}{Excitation dose in $\joule/(\centi\meter^2\textrm{stack})$} 
\nomenclature{$\Phi_e$}{Radiant flux of excitation light in watts} 
The dose $\Omega$ on the $x-$axis is calculated as
\begin{align}
\Omega = \frac{\Phi_e n \tau}{A},
\end{align}
with integration time $\tau$, area $A$ of the illuminated field, the
number of slices $n=41$ and radiant flux $\Phi_e$ of the excitation
light, as measured in the pupil.

Naively one would assume that it shouldn't make any difference if the
excitation light dose is administered with \unit[100]{ms} or
\unit[500]{ms} exposures but these data show that a longer exposure
time and low intensity are less phototoxic.

These results agree with an earlier study in tobacco plants
\citep{Dixit2003}. They investigate cell death a few days after
illumination and find that there is a threshold dose below which no
phototoxicity can be detected, and that this threshold decreases with
light intensity. Dixit and Cyr show that the damage is caused by
reactive oxygen species and they explain the shift of the
phototoxicity threshold by the limited capacity of the cells'
scavenging system for those radicals. They also predict the existence
of redox-sensitive checkpoints in the mitotic division cycle.

%\citep{Sancar2004}

In summary this section describes how to measure phototoxicity with
biological specimen.  The next section gives an overview of the
underlying photophysics and the rest of this work describes our
attempt to build a microscope with reduced phototoxic footprint.



\section{Photophysical principles of phototoxicity}
\begin{summary}
  Here I give a short overview of fluorescence of molecules in order
  to introduce the terms photobleaching and phototoxicity.
\end{summary}
A fluorophore is a molecule that can absorb and subsequently emit
light. During the absorption of a photon the molecular orbital
transitions from the electronic ground state $S_0$ to an excited state
$S_1$. The lifetime of the excited state $S_1$ is in the order of a
few nanoseconds.
\begin{figure}[!hbt]
  \centering
  \svginput{.8}{flu-level}
  \caption{The Jablonski energy level diagram of an illustrative
    fluorescent molecule. The boxes depict orbitals, up and down
    arrows symbolize the spin of the outer electrons. Fat horizontal
    lines represent electronic states. Thinner lines indicate
    vibro-rotational states. Various processes are shown with their
    typical time scales. VR = vibro-rotational relaxation, ISC =
    intersystem crossing, IC = internal conversion \cite[inspired
    from][]{Haken2006}.}
  \label{fig:flu-level}
\end{figure}
A Jablonski diagram, as depicted in \figref{fig:flu-level}, summarizes
information \cma{energy levels} about the energy levels of a molecule
and possible transition processes.

%  If the photon has an even higher
% energy, the electron will go into the second excited singlet state
% $S_2$.

The majority of known stable and bright fluorophores absorb and emit
in the wavelength range between \unit[300]{nm} and \unit[700]{nm}.
Photons at the high energy end of this range can excite molecules into
higher energy levels $S_n, (n>1)$ than the first excited state; these
states are unstable and hardly return to the ground state $S_0$. On
the other side of the spectrum: a molecule that absorbs in the
near-infrared ($\unit[>700]{nm}$) has a low-lying excited singlet
state $S_1$ and therefore potentially increased reactivity and a high
probability for a non-radiative transfer back into the ground state
$S_0$ \citep{Sauer2011}.


The term \emph{Stokes' shift} describes the frequency shift between
the absorbed and emitted photon; the energy difference is lost as heat
to the fluorophore molecule and surrounding solvent.  For the
practical implementation of fluorescence microscopes this is
significant, as it enables to separate excitation and emission light
with a dichroic beam splitter.

A fluorescence photon is emitted into a random direction. We use this
in the next section to describe image formation in the fluorescence
microscope.


The triplet states $T_n$ play an important role in photobleaching.
Pure electronic absorption of one photon has no effect on the spin of
an electron and therefore the transition from singlet states $S_n$
into the triplet state $T_n$ shouldn't occur. However, interaction
with the nuclei can mediate this spin transition. Therefore, in
fluorophores this transition has a small probability, resulting in
long lifetimes of the triplet state $T_1$.

\cite{Deschenes2002} show that excitation of higher triplet states
$T_n$ is the predominant reactive process for photobleaching in
vacuum. In particular they measured that one rhodamine~6G molecule
\emph{in vacuum} can emit more than \num{1e9} photons before it
bleaches, if the excitation intensity is low enough
$(\sim\unit[1]{\si{\watt/\cm^2}})$ to prevent decay over triplet
states.

In normal atmosphere the prolonged lifetime of the triplet state $T_1$
makes it highly likely for the fluorophore to react with molecular
oxygen $\O_2$. Oxygen is abundant and has a triplet ground state
${}^3\Sigma$ with two unpaired electrons of parallel spin in its
$\pi^*-$orbitals (see \figref{fig:oxygen}).

  \citep{Bernas2004}

\begin{figure}[!hbt]
  \centering
  \svginput{1}{oxygen}
  \caption{{\bf left:} Schematic that depicts how the orbitals of the
    oxygen molecule are formed from the atomic orbitals. {\bf right:}
    Molecular oxygen has the lowest energy in its triplet state
    ${}^3\Sigma$ where the spins of the two outer $\pi^*-$electrons
    are parallel. Inspired from \citet{Linde2011a}.}
  \label{fig:oxygen}
\end{figure}

If a ground-state oxygen molecule comes into physical contact with a
$T_1$ fluorophore, the energy of the latter can be transferred by an
electron exchange energy transfer mechanism in which the orbitals
directly interact with each other \citetext{\citealp[p.~438]{Haken2006} and
  \citealp{Linde2011a}}.

During this reaction, which is also known as triplet--triplet
annihilation, two forms of singlet oxygen form in competition: The
lower energy state ${}^1\Delta$ and the short-lived, higher energy
state ${}^1\Sigma$ that immediately ($T_{1/2}\sim\unit[10^{-9}]{s}$)
sends out a \unit[1268]{nm} photon and decays into ${}^1\Delta$.

The resulting singlet oxygen ${}^1\Delta$ is very reactive. In a
typical specimen it diffuses only a few tens of nanometres until it
reacts with another molecule.

(FIXME 2000 greenbaum measures oxygen production, bernas 2004 anoxia gfp)

Nowadays many methods are known to reduce photobleaching: Substitute
oxygen with noble gases or remove it enzymatically
\citep[p.~89]{Sauer2011}, depopulate the triplet state by adding
reducing as well as oxidizing agents to the solvent
\citep{Vogelsang2008} or couple a triplet quencher directly to the
fluorophore \citep[p.~19]{Sauer2011}. For fixed samples it helps to
change the solvent or polymer.
 
In living specimen these techniques may reduce photobleaching, but
they can also have a detrimental effect on the biological system
itself. Removing oxygen will quite certainly have a negative
effect. In order to reduce phototoxicity it makes sense to think about
the light management in the microscope.


\section{Conventional microscopes}
\begin{summary}
  Most of the fluorescence microscopes that are in common use today do
  not excite fluorophores of the specimen in an optimal way. In
  this section I outline how these microscopes work and explain how
  out-of-focus blur severely limits the performance of the wide-field
  microscope.
\end{summary}


A microscope, is a device that collects light coming from one plane  \cma{lateral image}
and forms a magnified image on a
camera. \figref{fig:widefield-microscope}~b) shows a schematic
representation of the detection path of a wide-field microscope.

The main components are an objective lens with focal length $f$ and a \cma{telecentric arrangement}
tube lens TL1 with focal length $f_\textrm{TL}>f$. Sample, lenses and
camera are arranged in double-telecentric configuration, i.e.\ the
sample is located in the front focal plane of the objective, the tube
lens is at distance $f_\textrm{TL}$ behind the pupil and the camera is
in the focal plane behind the tube lens.


\nomenclature{$\beta$}{Transversal magnification of an objective
  $\beta=f_\mathrm{TL}/f$, for Zeiss lenses the magnification $\beta$
  is written on the objective and the focal length of the tube lens is
  defined as $f_\textrm{TL}=\unit[164.5]{mm}$}


Light from the sample is collimated by the objective lens and
\cma{lateral magnification} re-imaged by the tube lens. The lateral
magnification $\beta$ is given by the ratio of the focal lengths of
the two lenses:
\begin{align}
  \beta=\frac{\overline{O'P'}}{\overline{OP}}=\frac{f_\mathrm{TL}}{f}.
\end{align}
Note that in \figref{fig:widefield-microscope}~b) I represent the
\cma{necessary corrections} objective lens as a single element.  This
is a simplification.

In the paraxial limit ray-tracing calculations for a thick lens or
even several consecutive lens elements can be simplified by bending
the ray only at one place --- at the principal plane.

\nomenclature{marginal ray}{Axial ray through the periphery of the
  entrance aperture}

\nomenclature{chief ray}{Ray from the periphery of the field through
  the center of the entrance aperture}

\nomenclature{entrance aperture}{Projection of the limiting aperture
  of the optical system into object space}


Microscope objectives must collect light from a large aperture in
order to produce a high resolution image. This is a fact I will
support shortly using the wave-optical model. Unfortunately the large
ray angles in the objective prevent its simplified description using
principal planes, but an analysis using the eikonal theory shows that
an optical system that fulfills the Abbe Sine condition allows perfect
imaging even for widespread ray bundles.
\begin{align}
  \beta = \frac{n \sin\alpha}{n' \sin\alpha'} \qquad \textrm{(Abbe Sine condition)}
\end{align}
This condition ensures that the focal length, a quantity which is
usually defined only for paraxial rays, is equal for all angles.  This
in turn means that such a lens carries out a Fourier transform from
the front to the back focal plane with linear scaling. Note that a
lens with a non-linear distortion in the back focal plane will fail to
produce an image that is similar to the object.

It turns out that ray bending in a high-aperture lens system that
fulfills the Abbe Sine condition can be simplified to a one bend at a
single surface, quite similar to the utilization of principal planes
in paraxial optics. For a high-aperture system this surface is no
longer a plane.  Instead it is a sphere with radius $n f$ and called
\emph{aplanatic sphere}. I depict this surface as two circle segments
with bold red strokes on the lenses in
\figref{fig:widefield-microscope}~b).

In addition to the Abbe Sine condition microscope lenses are also
corrected for spherical aberration and linear coma \citep{Gross2005}.
Then the coma rays are symmetric around the chief ray, the wavefront
and point spread function are approximately invariant for small field
sizes (in first order).  This ensures that the imaging conditions are
invariant for small regions of the field plane and allows to express
image formation with linear systems theory.

\subsection{Wave-optical theory for image formation}
In the following I want to describe how the image on the camera       \cma{wave optics}
forms. For this we have to use wave theory because close to the image
rays intersect, invalidating ray-optical predictions. As both
theories are very much related, we can give a useful interpretation of
the aplanatic surface for wave optics.

The underlying Maxwell equations and the wave equation are linear and   \cma{plane waves}
we can represent propagating solutions (evanescent solutions are
neglected) of the wave equation as a superposition of the elementary
solution --- the monochromatic, plane waves described by wave vector
$\k$:
\begin{align}
  u(\r,t)=u\,\exp(i(\k\r-\omega t)),\quad \r=(r_x,r_y,r_z),\
  \k=(k_x,k_y,k_z),\ |\k|=2\pi n/\lambda_0,
\end{align}
The accurate treatment of high-aperture optics would in fact require a
vectorial calculation of the image for a fluorophore with a particular
dipole orientation.  Subsequently these images should be averaged to
account for random fluorophore orientations, but as I don't need
quantitative expressions I limit myself to the simpler scalar problem
which provides a qualitatively similar result.

% \begin{figure}[!hbt]
%   \centering
%   \svginput{1}{sine-condition}
%   \caption{dasfklj}
%   \label{fig:sine-condition}
% \end{figure}



\begin{figure}[!hbt]
  \centering
  \svginput{1}{widefield-microscope}
  \caption{{\bf a)} Transmitted segment of the three-dimensional
    frequency spectrum is highlighted in red on the Ewald sphere. {\bf
      b)} Schematic of the detection path of a modern microscope. The
    sample is in the front focal plane of the objective. The detection
    tube lens TL1 forms a magnified image on the camera. The aplanatic
    spheres for objective and tube lens are indicated in
    \textcolor{red}{red}. {\bf c)} Parallel laser epifluorescence
    excitation. The excitation tube lens TL2 focuses a laser into the
    pupil of the objective. The beam is reflected by a dichroic beam
    splitter (BS) towards the objective. An extended area in the
    specimen is illuminated. Fluorescence light returns through the
    objective, is transmitted through BS and forms an image on the
    camera. }
  \label{fig:widefield-microscope}
\end{figure}

\nomenclature{$u(\r)$}{Scalar field as a function of spatial
  coordinates} 

\nomenclature{$\widetilde u(\vnu)$}{Fourier transform
  of scalar field as a function of spatial frequencies}

\nomenclature{$\r=(r_x,r_y,r_z)^T$}{Three-dimensional spatial
  coordinate} 

\nomenclature{$\r_t=(r_x,r_y)^T$}{Transversal two-dimensional spatial
  coordinate}

\nomenclature{$\vnu=(\nu_x,\nu_y,\nu_z)^T$}{Three-dimensional spatial
  frequency}

\nomenclature{$\vnu_t=(\nu_x,\nu_y)^T$}{Transversal two-dimensional
  spatial frequency}

Assuming excited fluorophores in the sample give rise to a                 \cma{Ewald sphere}
monochromatic electromagnetic field --- again, I simplify the problem
by omitting the complication that fluorophores emit photons in a
wavelength range --- then using the spatial frequency vector
$\vnu=\k/(2\pi)$ we can expand the three-dimensional, stationary field
amplitude distribution $u(\r)$ into its spatial frequency spectrum
$\widetilde u(\vnu)$:
\begin{align}
  u(\r)=\int_{-\infty}^{\infty}\int_{-\infty}^{\infty}\int_{-\infty}^{\infty}
  \widetilde u(\vnu) \exp(2\pi i \r\vnu)\ \textrm{d}^3 \vnu
\end{align}
Since we have assumed a monochromatic field and the length $|\vnu|$ of
the spatial frequency vector is the inverse of the wavelength, the
support of this spectrum $u(\vnu)$ is limited to the surface of a
sphere of radius $n/\lambda_0$:
\begin{align}
  \supp \widetilde u(\vnu) &= \{\vnu \in \mathbb{R}^3: |\vnu|=n/\lambda_0\}.
\end{align}
This sphere is the transfer function of free space, and is also called
Ewald sphere.  \nomenclature{Ewald sphere}{Transfer function of free
  space} Scaling the Ewald sphere with $f\lambda_0$ gives the
aplanatic surface of the lens. 

According to \cite{McCutchen1964} the transfer function $\widetilde
h(\vnu)$ of the lens is defined by complex values on the Ewald sphere. 
aperture:
\begin{align}
  \widetilde h(\vnu)&=P(\vnu_t) \exp\left(\frac{2\pi i}{\lambda} 
    W(\vnu_t)\right)
  \delta\left(|\vnu|-\frac{n}{\lambda_0}\right),
\end{align}
with the Dirac delta function $\delta$, transversal spatial frequency
vector $\vnu_t=(\nu_x,\nu_y)^T$, and the real valued pupil function
$P(\vnu_t)$ and wavefront error $W(\vnu_t)$. McCutchen calls
$\widetilde h(\vnu)$ the generalized aperture.

For this discussion I set $W(\vnu_t)=1$, i.e.\ there are no wavefront
aberration and the lens is diffraction limited. Furthermore I use a
uniform cylinder as pupil function $P(\vnu_t)$, in order to limit the
size of the calotte or cap of the Ewald sphere that is defined by the
acceptance angle $\alpha$ of the objective\footnote{Note that this
  expression is only valid for $\alpha\in[0,\pi/2]$. An expression for
  $\widetilde h(\vnu)$ encompassing the full range $[0,\pi]$ for
  $\alpha$ must contain two functions of each $P$ and $W$, in
  dependence on whether the spatial frequency vector $\vnu$ is
  directed in or against the direction of the optical axis. This is
  necessary to express the transfer function of a 4Pi microscope.}:

% \begin{align}
%   \widetilde h(\vnu) =
%   \begin{cases}
%     P_-(\vnu_t) \exp\left(\frac{2\pi i}{\lambda} 
%     W_-(\vnu_t)\right)
%   \delta\left(|\vnu|-\frac{n}{\lambda_0}\right) & \nu_z<0
%  \\
% P_+(\vnu_t) \exp\left(\frac{2\pi i}{\lambda} 
%     W_+(\vnu_t)\right)
%   \delta\left(|\vnu|-\frac{n}{\lambda_0}\right) & \nu_z\ge 0
%   \end{cases}
% \end{align}

\begin{align}
  P(\vnu_t) &=
  H\left(|\vnu_t|-\frac{n\sin(\alpha)}{\lambda_0}\right), \quad \textrm{with}\ 
  H(x)=
  \begin{cases} 
    1 & x\ge 0 \\
    0 & x<0 
  \end{cases}
\end{align}
where $H(x)$ is the step function. In general $P(\vnu_t)$ can assume
values between 0 and 1 in order to account for apodization due to
Fresnel reflection or natural vignetting. I ignore these effects here.

Multiplication of the angular frequency spectrum $\widetilde u(\vnu)$
with the generalized aperture $\widetilde h(\vnu)$ gives the angular
frequency spectrum of the amplitude in the image:
\begin{align}
  \widetilde u'(\vnu) = \widetilde u(\vnu)\cdot \widetilde h(\vnu).
\end{align}
According to the convolution theorem this multiplication of the
spectra corresponds to a convolution of the field distribution $u(\r)$
and an amplitude point spread function $h(\r)=\mathcal{F}(\widetilde
h(\vnu))$ that describes the imaging of the objective lens:
\begin{align}
  u'(\r) = u(\r) \otimes h(\r) =
  \int_{-\infty}^{\infty}\int_{-\infty}^{\infty}\int_{-\infty}^{\infty}
  u(\r')\ h(\r-\r')\ \textrm{d}^3\r'.
\end{align}


with the
three-dimensional frequency spectrum  of the
sample as shown in \figref{fig:widefield-microscope}~a).




transversal spatial frequencies $\vnu_t=(\nu_x,\nu_y)$

Die Pupille begrenzt die durchgelassenen Strahlbuendel und wirkt damit
als low-pass filter auf das Ortsfrequenzspektrum. Falls das System
Aberrationen aufweist, koennte man diese hier mit der Wellenfront
$W(\vnu_t)$ einfuegn. Durch multiplikation mit der Funktion $\widetilde h$ 

diffraction limited W=1




Faltungstheorem
\begin{align}
  \mathcal{F}[\widetilde h(\vnu_t)\widetilde u(\vnu_t)]
\end{align}


angularly band-limited


\begin{align}
  A &= \frac{f}{d} = \frac{1}{2\tan\alpha}\\
  h_r(\vnu_t,\nu_z)&=\frac{1}{2\pi\nu_t A}\sqrt{1-\left(\frac{2A\nu_z}{\nu_t}\right)^2}
\end{align}

The Ewald sphere allows an intuitive calculation of the transfer
function of a microscope. The low-pass 

The pupil works as a filte




no absorption or diffraction in sample
random fluorophores
recording successive slices, telecentricity, real microscope 
main results: image formation linear in intensity, three dimensionally shift-invariant

missing cone \cite{Streibl1984}


- note that $\nu_z$ can be expressed in terms of the other components,
  replacing in the exponential inside the integral acoordingly gives
  with $= U(\nu_x,\nu_y,\nu_z)/cos(\alpha)$

$$ u(x,y,z)=\int\int U_\textrm{2D}(\nu_x,\nu_y)|_{z=0}  exp(2\pi i (x \nu_x+y\nu_y+z\sqrt{(n/\lambda)^2-\nu_x^2-\nu_y^2})) d \nu_x d \nu_y$$

- the aperture angle $\alpha$ defines the maximum transversal freq of
   the spectrum of a 3d scalar point response function

- the 3d freq spectrum is given by a segment of the ewald sphere
  (which mccutchen calls generalized aperture)

- based on this, one can estimate the transversal distribution of the
  psf I(r,z=0) and the axial psf I(r=0,z)

- first order born approximation:

    - light is deflected only by a single interaction 

    - diffraction is linear in frequency space: diffracted spectrum $U_s$ is
      given by the convolution of the incident spectrum $U_i$ with object
      spectrum f

$$U_s(\nu_x,\nu_y,\nu_z) \sim f(\nu_x,\nu_y,\nu_z) \otimes U_i(\nu_x,\nu_y,\nu_z)$$

    - if $U_i$ is a planar wave, then the scattered wave is just the
      object spectrum shifted by the frequency of the incoming wave

    -single moment transfer: only those frequencies $\vec\gamma$ are
     transferred forr which the laue equation is satisfied

$$\nu_s-\nu_i=\vec\gamma$$





a double telecentric system



the entrance pupil is the the image of the limiting aperture into 

aperture stop limits the direction cosines passing from object space
to image space through the optical system

in a certain distance from the image plane a spherical wave is assumed,
the so-called Gaussian reference sphere


\begin{align}
  \beta = \frac{\nu}{\nu'}=\frac{n\sin\alpha}{n'\sin\alpha'}
\end{align} % FIXME steht da nun n oder nicht? IAT 3

the aplanatic surface and ewald sphere are related just by a factor
$f\lambda$

for application of linear systems theory it is necessary that the
imaging conditions are invariant at least over small regions of the
field plane (isoplanatic condition)

The field in the pupil is
\begin{align}
u(\nu)=F\left[U(x)\right]
\end{align}


The field behind the pupil aperture is
\begin{align}
u'(\nu)=h(\nu) u(\nu)
\end{align}

the field in the image plane is obtained by a repeated Fourier
transform with a corresponding scaling of the pupil coordinates
linear mapping between object and image space coordinates $x'=\beta x$

scaling of the spectra $\nu'=$
\begin{align}
U'(x')=\mathcal{F}(h(\nu) u(\nu)) = H(x) \otimes U(x) = \int U(x'') H(x-x'') \textrm{d}x''
\end{align}

the camera can only detect the intensity
\begin{align}
I'(x')=|U'(x')|^2=U'(x')U'^*(x')
\end{align}

the marginal ray starts from an outer field point in the object and
passes through the center of the entrance pupil, which here is in
this double telecentric system is in axial infinity 

in a system, natural vignetting (projection along chief ray in object
and image) should be taken into account with energy apodization
factors

etendue
geometrical flux



The uncoloured beam in \figref{fig:widefield-microscope}~a) represents
rays that start from the intersection $O$ of the optical axis and the
front focal plane of the objective. The objective collects the rays
and collimates them into a beam that is parallel to the optical
axis. After traversing the tube length $f+f_\mathrm{TL}$, the rays are
focused by the detection tube lens TL1 on the intersection $O'$ of its
focal plane and the optical axis. 

The blue beam corresponds to rays that start from an off-axis point
$P$ in the front focal plane of the objective. Behind the objective
the blue beam is a parallel beam. However, the beam is tilted relative
to the optical axis. The tube lens TL1 focuses the blue beam into a
spot at $P'$ on its focal plane.

The objective fulfils the Abbe sine condition -- it is aplanatic. The
microscope forms stigmatic\footnote{An imaging system collects some of
  the rays, that leave an object and directs them towards the
  image. If all rays that leave an object point converge in the
  conjugate image point, then we call the image point stigmatic.}
images of points from the front focal plane in the plane perpendicular
to the optical axis, containing $O'$ and $P'$. The plane with the
images is called intermediate image plane. It is magnified by the
factor $M$:


In our microscope we use an objective with magnification $M=63$. The
focal length of the tube lens is for most Zeiss
microscopes. Therefore the focal length of our objective is
$f=\unit[2.61]{mm}$.

Let's assume we have an opaque sample with just two small
($\diameter<\unit[120]{nm}$) holes with $\unit[2]{\mu m}$ distance
between them.  We put this object into the front focal plane of the
objective and position a camera on $O'$. When illuminating the mirror
from the side opposite to the objective, the camera will show two
spots with $\unit[126]{\mu m}$ distance.

\nomenclature{NA}{Numerical aperture $\textrm{NA}=n\sin\alpha$, with
  refractive index $n$ of immersion medium and acceptance half-angle
  $\alpha$ of the lens}

Note that \figref{fig:widefield-microscope} depicts a \emph{thin-lens
  model of a high numerical aperture objective} that fulfils Abbe's
sine condition. A real objective contains in the order of ten coated
lenses of different glass and crystalline materials. Their curvatures,
positions and materials were all carefully chosen, taking into account
manufacturing tolerances and wavelengths, so that the microscope
behaves as the thin-lens model predicts. Diffraction at the periphery
of the pupil in the back focal plane dictates the resolution, one can
achieve inside of the sample.


It is quite possible that heating to \unit[37]{${}^\circ$C} will ruin
such a high-precision instrument. A related source of aberrations
(departure of design performance) is the refractive index inside of
the specimen. In Appendix~\ref{sec:ray-aberration} we describe a more
complicated model that can predict the effect of embedding the sample
in water (instead of immersion oil with the same refractive index as
the glass).

\subsection{Widefield epifluorescence microscope}
Fluorescence photons are emitted in all directions, independent of the
original illumination direction. Therefore it is possible and
convenient to use the objective for excitation as well as
detection. This mode of microscopy is called epifluorescence (Greek:
$\varepsilon\pi\iota$; on, above).  In this configuration usually only
a small percentage of the excitation light returns due to diffraction
or reflection. This simplifies the separation of fluorescence light
from excitation light.  Furthermore parts of opaque specimen can be
imaged and it is beneficial that the illumination needs to be aligned
only once.

\nomenclature{BFP}{Back focal plane}

The red beam in \figref{fig:widefield-microscope}~c) is a parallel
laser. The excitation tube lens TL2 focuses the beam into the back
focal plane (BFP) of the objective. The beam is reflected at a
dichroic beam splitter (BS). This is a glass plate that has been
coated with dielectric layers. The refractive index, thickness and
sequence of the layers are designed so that the excitation light is
reflected towards the objective. Excitation light, that is scattered
or reflected in the sample and returns through the objective is
reflected towards the light source. However, lower energy fluorescence
light returning from the objective is transmitted towards the
camera. Behind the objective the beam is parallel and illuminates the
specimen. The field of view is the demagnified diameter of the laser
beam before TL2.
\subsubsection*{Non-uniformity due to coherent interference}
Note that tiny dirt particles and coherent interference in laser beams
can produce unwanted non-uniformities in the illumination. As a remedy
the spatial coherence of the laser is sometimes reduced.  Incoherent
light emitting diodes, mercury or xenon arc lamps are often used
instead of lasers. In the latter case a band pass filter selects the
useful part of the spectrum of the excitation lamp upstream of the
dichroic beam splitter.

\subsubsection{Out-of-focus blur}
However, independent of the choice of the light source, the wide field
microscope in epifluorescence configuration exposes many layers of the
sample. This leads to fluorescence of out-of-focus fluorophores.

There are two reasons, why out-of-focus fluorophores give blurred
images. Not even an ideal imaging system -- a device that forms
%% FIXME refer to maxwell or born/wolf
stigmatic images of all the points in one volume in another volume --
would form sharp images on the camera plane. After all, the camera is
just a plane and the object under observation is three dimensional.

Furthermore a microscope is far from being an ideal imaging system. In
a microscope it is not possible to obtain a sharp image of a different
slice of the object by changing the axial position of the camera
behind the tube lens TL1 \citep{Botcherby2007,Botcherby2008a}.
\subsubsection*{Deconvolution}
When a stack of several slices of an object is obtained, it is
possible to suppress the blurred part of each image in all the
others. These algorithms (deconvolution) can improve the perceived
quality of images in some stacks. However, there are two fundamental
problems:

First the \emph{missing cone problem} prevents focusing on a
homogeneous fluorescent plane. Physics dictates that there is always a
gap in the transfer function of the objective when the fluorescence
process is linear and the objective collects only photons from one
half space (see \figref{fig:missing-cone}). Not all spatial
frequencies within the transfer function attenuate with defocus
\citep{Neil1997}.

\begin{figure}[!hbt]
  \centering
  \svginput{1}{missing-cone}
  \caption{Schematic depicting $k_xk_z-$cross sections of the support
    of optical transfer function (see Appendix~\ref{sec:app_hilo}) for
    microscope objectives with different collection angles. {\bf
      left:} Objectives, that only collect light that is directed into
    one half space, have the missing cone problem. There, low spatial
    frequencies do not attenuate with defocus. {\bf right:} Theoretical
    objective with larger collection angle and no missing cone.}
  \label{fig:missing-cone}
\end{figure}

Second, even with ideal detectors there is photon shot noise in the
image. In deconvolution algorithms the image of one slice is improved
by subtracting blurred versions of the other slices. When the blurred
intensity is large, its shot noise is high as well. Subtraction only
increases noise and a faint in-focus image can be severely
deteriorated by the noise of the out-of-focus light.
\subsection{Confocal microscope}


{\bf c)} Confocal
    microscope. A pinhole PH2 is imaged as a diffraction limited spot
    into the specimen. Returning fluorescence light is only detected
    when it passes through an aligned pinhole PH1. This configuration
    rejects light that doesn't originate from the front focal plane
    (green) of the objective.

One way of addressing both problems of the wide field microscope is
depicted in \figref{fig:widefield-microscope}~c). In the confocal
microscope the whole field of view isn't illuminated simultaneously.
The excitation tube lens TL2 collimates the light coming from a
pinhole PH2 and illuminates the full back focal plane of the
objective. In the front focal plane of the objective the red beam then
converges to illuminate the smallest possible single spot. The spot
size is defined by diffraction at periphery of the BFP. However,
out-of-focus fluorophores are still being excited by the hour-glass
shaped illumination.

The eponymous idea of the confocal microscope is to replace the camera
with a pinhole PH1. This pinhole, if aligned carefully to the position
of the image of the focused laser spot, has no influence on the light
detected from in-focus fluorophores. However, an out-of-focus
fluorophore that is defocused by $\Delta z$ towards the objective will
lead to a diverging beam (colorless) at the tube lens and will be
imaged into a point behind the focal plane of the tube lens. The
pinhole only transmits a part of the circle of confusion. Hence
defocused fluorophores contribute less to the sensor signal.

An image of the in-focus specimen is obtained by scanning the pinholes
PH1 and PH2 over the field of view and measuring intensity at each
position individually. The optical removal of out-of-focus light
prevents degradation of the signal by its shot noise and improves the
point-spread function of the objective. The missing cone problem is
fixed and the resolution improved by a factor of two. Note however,
that information about out-of-focus fluorophores is lost which would
be obtained in a wide field microscope with deconvolution. Therefore a
wide field microscope will give better results when a lot is known
about the sample structure and this knowledge is fed into the
deconvolution. E.g.\ the localization of sparse beads of specific size
will be better in a wide field microscope.

The confocal microscope was invented in 1955 \todo{check patent
  citation} \citep{Minsky1961,Minsky1988} to reduce the influence of
scattering effects in neuron samples stained by Golgi's method. This
invention preceded the laser and was unfortunately not put into
practical use for biology until three decades later \citep{Amos1987}.
\subsection{Phototoxicity in conventional microscopes}
When imaging living specimen we should distinguish between useful and
unnecessary excitation. Taking into account the detection capabilities
of objective lenses we should maximize the ratio of in-focus to
out-of-focus fluorescence. The epifluorescent wide field and confocal
microscope surely do not represent an optimum in this regard.

The following chapter \ref{sec:approaches} will introduce other
microscopy techniques that are more considerate of where to deposit
excitation power within the specimen.
\subsection{2-photon laser scanning fluorescence microscopy}
\label{sec:2-photon}
If the laser intensity in the focal spot of a confocal microscope is
sufficiently high, then two infrared photons can be absorbed within
\unit[$\sim 5$]{fs} and excite the same electronic state.

In this regime, the fluorescence emission increases quadratically with
laser intensity. This non-linearity confines the excitation volume to
the vincinity of the focal plane \citep{Denk1990}. Fluorophores
outside of this region are not excited. Therefore this method produces
sectioned images by default and there is no need for a detection
pinhole.

As an additional benefit infrared light is scattered less than visible
light of half the wavelength. This increases penetration depth and
image quality. Photodamage outside of the focal volume is unlikely and
phototoxicity is much lower, compared to the single-photon confocal
microscope, when $z-$stacks are acquired.

However, the phototoxicity within the focal volume is higher and
techniques like ultramicroskopy (section
\ref{sec:light-sheet-microscopy}) with single-photon excitation are
preferable, when low overall phototoxicity is a requirement.

\section{Image detectors in wide field microscopy}
\label{sec:ccd-intro}
\begin{summary}
  Here we describe CCD\nomenclature{CCD}{Charge-coupled devices}
  sensors and their characteristics.
\end{summary}
Charge-coupled devices are semiconductor devices that contain a 2D
grid of capacitors, formed by at least three groups of electrodes
(phases). Cycling the voltage on these electrodes allows to push
charges, which has been accumulated under the capacitors (registers)
into their neighbours. They turned out to be the ideal tool to move
charges, produced by photon absorption in light sensitive diodes,
across the substrate into read out logic.

Forty years of development lead to imaging devices with remarkable
charge transfer efficiency, high quantum efficiency (up to 95\% with
back illumination) and very low dark currents. Until ten years ago the
performance of CCD imagers in the low light regime was limited by the
noise of the read out amplifier (a few electrons per pixel
rms\footnote{root mean square} \todo{rms}).

Since the millennium we have electron multiplying CCD (EMCCD)
\nomenclature{EMCCD}{Electron multiplying charge-coupled devices}
technology, which allows comparably good performance at low photon
numbers \citep{Mackay,Robbins2003} and moderate read out speeds (tens
of MHz). EM-CCDs contain a row of additional registers in front of the
read out circuit. There, one of the three phases is clocked with a
much higher voltage (up to \unit[40]{V}) then is needed purely for
charge transfer ($\unit[\sim6]{V}$). The large electric fields cause
charge carriers to be accelerated to sufficiently high velocities, so
that additional carriers are generated by impact ionization. The
charge multiplication chance per transfer is small ($\sim1\%$) but by
using several hundred registers a substantial gain in the number of
charges can be achieved. In microscopy we usually work with gains of
up to 300. Higher gains are possible but limit the dynamic range.

The charge amplification helps to push the read noise from
$\sim\unit[40]{electrons\ rms}$ to significantly below
$\unit[1]{electron\ rms}$ --- in effect creating a sensor limited only
by the photon noise. However, the multiplicative nature of the gain
leads to a perceived reduction in the quantum efficiency of the sensor
(excess noise factor), i.e. an image with $\unit[100]{photons/pixel}$
without gain will look like the same image at only
$\unit[50]{photons/pixel}$ with EM-gain (see Appendix
\ref{sec:ccd-meas}).


pixel in
ccd ist passiv
cmos ist aktiv

column parallel readout sony exmor

exmor r additionally back illuminated (only works for small sensors)

%%% Local Variables: 
%%% mode: latex
%%% TeX-master: "kielhorn_memi"
%%% End: 


\chapter{Methods of controlling illumination patterns}
\label{sec:illum-patterns}
\section{programmable array}
\cite{Jovin2011}

\chapter{The concept of spatio-angular microscopy}
\label{sec:concept}
\begin{summary}
  Here we introduce our spatio-angular microscope. First we motivate
  the concept of its illumination system using exemplary fluorophore
  distributions, that occur in typical specimen.

  Then we describe some decisions we faced during the initial design
  phase concerning the arrangement of optical components. Furthermore,
  we position our method within known approaches of light control for
  microscopy. Of all published techniques for excitation illumination
  control, the light field microscope \ref{levoy} comes closest to our
  approach.  We explain differences between both techniques and
  discuss their respective pros and cons.  We (FIXME verschieben)
  discuss the peculiarities and limitations of the hardware components
  only in later chapters (\ref{sec:dev1}, \ref{sec:mma}).  Initially,
  the details would be detrimental to clarity.

  It turns out, that the effective use of the spatio-angular
  microscope, requires more knowledge about the specimen than a
  conventional or a SPIM microscope (\ref{spim}). Ideally the
  distribution of refractive index and fluorophores within the
  specimen should be known. If these parameters were known perfectly,
  there wouldn't be any (FIXME) necessity for an image in the first
  place. However, while imaging a known specimen, predictions (FIXME
  gute) of these parameters can often be made. The higher the
  precision of these predictions, the greater the reduction in
  phototoxicity will be.

  The computer-based selection of appropriate illumination masks
  requires the prediction (FIXME), or at least an estimate (FIXME
  understanding), of the three-dimensional distribution of light within
  the specimen.

  In the last part of this chapter, we describe how we practically
  implement the computational control loop in our spatio-angular
  microscope. Here we touch topics of image processing and we also
  draw parallels to treatment planning for radiotherapie of tumors.
\end{summary}
\section{Motivation}

  - Um die grundlegende Idee hinter dem Spatio-Angularen Mikroskop zu
    verstehen, betrachten wir zunaechst die Lichtverteilung im Objekt
    bei einem herkoemmlichen Mikroskop: Abbildung fig:hourglass-all-a
    zeigt schematisch die Seitenansicht von Objektivlinse, Objekt und
    dem Strahlenverlauf des Anregungslichtes in einem konfokalen
    Mikroskop. Ein paralleles Lichtbuendel mit kreisfoermigem
    Querschnitt (in der Darstellung nicht sichtbar) trifft auf die
    Objektivlinse. Die Linse fokussiert das Licht in ihrer Brennebene.

  - Zwischen Linse und Brennebene bilden die Lichtstrahlen einen
    konvergenten Kreiskegel. Angenommen, wir haben eine schwach
    absorbierende Probe, die Energie des Lichtes entlang der
    kreisfoermigen Querschnitte innerhalb des Kegels bleibt dann
    konstant. Die Intensitaet innerhalb des Kegels ist proportional
    zur Dichte der Lichtstrahlen in jedem kreisfoermigen Querschnitt
    und steigt demnach quadratisch an\footnote{Das strahlenoptische
    Modell gilt in grossen Teilen der Darstellung, jedoch nicht
    ueberall.  Das Gesetz von Malus-Lupin besagt, dass die
    Beschreibung mit Lichtstrahlen oder Wellenfronten equivalent sind,
    solange sich Strahlen nicht ueberschneiden (Kaustik) oder (FIXME
    formeln) oder ein starker Intensitaetsgradient auftritt. Demnach
    gilt das strahlenoptische Modell fast ueberall im Kegel, bis auf
    einen Bereich mit einem Abstand von wenigen Wellenlaengen zum Rand
    und im Fokus selbst. Die wellenoptische Behandlung dieser Bereiche
    ist zwar moeglich, rechentechnisch aber erheblich
    aufwaendiger. Deshalb beschraenken wir uns in unserem Prototypen
    und dieser Arbeit ausschliesslich auf das strahlentheoretische
    Modell}.

  - Der fluoreszente Bead (1) im Fokus wuerde demnach deutlich
    staerker angeregt werden, als der Bead (2) ausserhalb der
    Fokusebene. Im konfokalem Fluoreszensmikroskop wird das
    Fluoreszenslicht beider Beads vom Objektiv und
    Detektionstubuslinse in die Zwischenbildebene abgebildet werden.
    Das Bild des in-focus Beads (1) ist dabei scharf, von ihm
    ausgehendes Fluoreszenslicht wird auf einer moeglichst kleinen
    Flaeche konzentriert -- genau auf dem Zentrum des
    Detektionspinholes.  Der out-of-focus Bead (2) erzeugt hingegen
    nur ein unscharfes Bild, sein Licht wird ueber eine grosse Flaeche
    verteilt. Zum detektierten Signal des konfokalen Mikroskops traegt
    zwar nur ein verschwindend geringer Anteil des vom Out-of-fokus
    Beads emittierten Lichts bei, mit Blick auf die Phototoxizitaet
    des Systems kann man jedoch sagen, dass es besser waere, die
    Anregung des out-of-fokus Beads von vornherein zu unterbinden.


\begin{figure}[!hbt]
  \centering
  \svginput{.43}{hourglass-all}
  \caption{{\bf (a)} Two fluorescent beads are illuminated by all
    angles that an objective can deliver. The sharp image of the
    in-focus bead is deteriorated by blurry fluorescence of the
    out-of-focus bead. {\bf (b)} Angular control allows selective
    illumination of the in-focus bead and results in a better image on
    the camera. {\bf (c)} Angular control is insufficient, when an
    extended in-focus area is illuminated. {\bf (d)} However,
    simultaneous spatial and angular control allows sequential
    excitation of the in-focus beads while excluding the out-of-focus
    bead.}
  \label{fig:hourglass-all}
\end{figure}



\begin{figure}[!hbt]
  \centering
  \svginput{1.5}{memi-simple}
  \caption{Simplified schematic of the illumination system in our
    spatio-angular microscope. A homogeneous extended light source
    illuminates from the left. It is imaged by $L_1$ and $L_2$ into
    the intermediate image $F'$. Then the tubelens $L_3$ and the
    objective $L_4$ form an image of $F'$ in the sample plane $F$. The
    first spatial light modulator SLM1 is in the plane P', which is
    conjugate to the pupil (BFP) P of the objective. Using SLM1 we can
    control illumination angles in the sample. SLM2 is directly imaged
    into the sample and allows spatial illumination
    control.} 
  \label{fig:memi-simple}
\end{figure}

\section{A protocol for spatio-angular illumination control}
\section{Finding optimal illuminationOptimization using a raytracer}


\chapter{Device 1: prototype for spatio-angular illumination}
\begin{summary}
   - Im vorhergehende Kapitel haben wir das dem spatio-angularen
     Mikroskop zugrundliegende Konzept dargestellt. Hier gehen wir auf
     zusaetzliche Details ein, die fuer die praktische Implementierung
     wichtig sind. Unter anderem die Eigenschaften der beiden
     verwendeten Displays, elektronische Synchronisation der
     verschiedenen Komponenten und einem Algorithmus, um das             % /Hier mehr spezifische Probleme/
     Koordinatensystem der Kamerapixel und der Pixel des focal plane
     SLM ineinander zu transformieren.

   - Das pupil plane SLM wurde durch unseren Partner Fraunhofer IPMS
     waehrend des Projekts neu entwickelt.  Daher widmen wir uns diesem   % /MMA kommt spaeter extra/
     Subsystem im Kapitel (FIXME) naeher.
\end{summary}
\section{Description of the optical components}
So far we have only shown the beam path for transmissive displays (in
\figref{fig:memi-simple}). Such SLM only have a very low transmission
in practice. Therefore we use reflective displays in our prototype.

In \figref{fig:memi-real} I adjusted the beam path accordingly. This
schematic also depicts the optics we use to adapt light from the laser
to fill the etendue of our system. The light source enters the system
from the bottom left. The optic components are color corrected and
have anti-reflex coating for wavelengths in the range from
\unit[400]{nm} to \unit[700]{nm}.

The system successively illuminates the pupil plane SLM---a grayscale
micromirror array developed by our project partner Fraunhofer IPMS
Dresden---and the focal plane SLM, a commercial binary liquid crystal
on silicon display.
 
I gathered some of the following details from the documents that were
created during the development of our prototype and are classified as
confidential. I have summarized the key decisions here and the
relevant project partners have agreed to the publication (FIXME not
finished).


\begin{figure}[!htbp]
  \centering
  \svginput{2}{memi-real}
  \caption{Schematic of the light path through our microscope. Laser
    light enters from the lower left, is scrambled and homogenized to
    illuminate the pupil plane SLM in P'' and the focal plane SLM in
    F'. $F$ is the field plane in the sample and its primed versions
    are conjugated planes. $P$ is the pupil of the objective. $B_0$
    and $B_1$ are adjustable circular apertures. PBS is a polarizing
    beam splitter. DBS is a dichromatic beam splitter.  The red boxes
    deliminate subsystems of the illumination system: {\bf A:} light
    scrambling and homogenization, {\bf B:} Fourier-optical filter to
    provide intensity modulating pupil plane SLM. {\bf C:}
    Polarization based intensity modulator as focal plane SLM. {\bf
      D:} Wide-field fluorescence microscope with detection
    path. (FIXME finish diagram, don't use B twice)}
  \label{fig:memi-real}
\end{figure}

\subsection{Ensuring homogeneous illumination}
A quantitative evaluation of our experiments (FIXME ref sec:results)
with different illumination patterns is simplified when both pupil
plane SLM and focal plane SLM are uniformly illuminated.

We use either a laser\footnote{Lasever LSR473H, diode-pumped solid
  state laser, output power 600mW, $\lambda=\unit[473]{nm}$} or an
light emitting diode (LED) \nomenclature{LED}{light emitting diode} as
the light source in our experiments. Below we discuss optical measures
that attain homogeneity of the illumination of both displays.

The LED\footnote{Huey Jann HPB8-48KBD, wavelength
  $\unit[(463\pm1)]{nm}$, brightness \unit[35]{lm}, view angle
  $120{}^\circ$, FIXME TODO: Flaeche messen} we use has a large active
area. Due to etendue mismatch a relatively large amount of its
produced light will never reach the sample. But it is easy to achieve
a homogeneous illumination. Moreover, the LED can be quickly switched
on and off electronically \footnote{The DPSS Laser doesn't allow fast
  direct electronic switching at full power. We have to use an
  acousto-optic modulator connected with the additional expense of its
  optical alignment (FIXME siehe spaetere ref section).}.

Unlike an LED, a laser delivers light of considerably higher spectral
radiance ($\unit[]{W/(sr\, m^2 m)}$). Thus it is in principle possible
to use the laser as a highly efficient light source for our
system. Unfortunately, the high spectral and spatial coherence of a
laser often lead to high-contrast fluctuations of the irradiance and
we have to compensate for this by time averaging.

When using the Laser, we send its parallel Gaussian beam into a
bundle\footnote{Fiber bundle with circular cross-section (Loptek,
  Berlin, DE), \unit[1.1]{mm} diameter and \unit[2]{m} length. The
  beam broadening is $3{}^\circ$ and increases, when the bundle is
  bent \citep{D8.4}.}  of randomly distributed fibers. This randomizes
the light distribution at the bundle output and also broadens the
illumination angles.

A relay system (A1) images the circular output of the fiber bundle
onto the entrance of a light pipe. This relay system contains a
rotating microlens array\footnote{Array of cross-oriented cylindrical
  lenses on both sides with a pitch of \unit[0.5]{mm} resulting in an
  effective focal length of \unit[6.9]{mm} (LIMO, Dortmund, DE).}. It
is driven by a motor with the axis of rotation being diplaced from the
optical axis. This time-varying element allows to reduce speckle.

Both, the fiber bundle and microlens array, increase the etendue of
the laser illumination to the optimum value, which is given by one of
our SLM as discussed below in \ref{sec:etendue} (FIXME ref). 

The light  pipe is  a hollow  mirror-integrator tunnel  with quadratic
cross-section and depicted in \figref{fig:integrator-rod}. The mixing
effect of the  tunnel can be understood by  considering the irradiance
in the plane of the tunnel output as it would occur without tunnel.

Drawing the outline of the square cross-section into this irradiance
map selects the light that directly reaches this plane.  Surrounding
this outline with the four squares that touch its edges selects the
light that will reach the output plane after one reflection. The
irradiance maps from neighbouring squares are mirrored and added to
the direct illumination. Depending on the numerical aperture of the
input light, more reflections may occur --- resulting in the addition
of irradiance from next-nearest-neighbours and so forth.

This improves the uniformity of the light distribution in the output
plane without altering the numerical aperture of the light.  The more
subregions are superimposed, the better will be the uniformity.
Assuming $N$ subregions were overlaid and their contributions were
statistically independent, then according to the central limit theorem
the standard deviation of the irradiance is proportional to
$1/\sqrt{N}$ \citep{Koshel2012}.

However, we also align the source distribution to be rotationally
symmetric about the optical axis and obtain an even more uniform
output than this prediction because positive and negative slopes from
different subregions compensate in the superposition (also
\cite{Koshel2012}).

In our system the side length 

ueberaus
prohibitively 

Eine Relais-Optik (A1
   und A2 in Fig 4.1) vergroessert
   den Tunnelausgang des Tunnels auf $\unit[4\times4]{mm^2}$
   in die Ebene F'''.

\footnote{Ein Tunnel mit $\unit[4\times4]{mm^2}$
   Querschnitt beduerfte nicht dieser Optik, dann waeren die Winkel
   der Strahlen im System jedoch noch kleiner und der Tunnel muesste
   unhandlich lange werden.} 

   \jpginput{8cm}{integrator-rod}{Hollow mirror-integrator tunnel with
     a quadratic cross section of \unit[2.5]{mm}
     side length and \unit[250]{mm} length.}



 - Zu den zwei Relais-Systemen hat der Optikdesigner kommentiert
   (FIXME ref D8.9), dass diese nicht fuer eine perfekte Abbildung,
   sondern fuer einen guten Transport der homogenenen Lichtverteilung    % /Interessantes zu Relais-Systemen an Tunnelenden/
   optimiert wurden. Beim System A1 am Tunneleingang werden drei Elemente
   (FIXME oder 2?, und wo ist das Mikrolinsenarray) eingesetzt, um das
   Licht vom runden Faserende in den quadratischen Tunneleingang zu
   transportieren. Am anderen Ende (A2 Fig 4.1) transportieren fuenf Elemente das
   Licht vom Tunnelausgang in die Ebene F''' mit der
   Beleuchtungsapertur.

 - Waehrend der Konzeption wurde auch eine auf zwei Mikrolinsenarrays    % /Nicht benutzt alternative/
   (fly's eye condensor) basierende Optik fuer die Homogenisierung des Lasers in
   Betracht gezogen (FIXME ref D8.2). In-Visions Planung zufolge,
   waere dieser jedoch schwieriger zu justieren als der Tunnel und
   zudem nicht fuer den vollen Wellenlaengenbereich von 400 bis 700nm
   verwendbar gewesen.

  - Um eine homogene Ausleuchtung mit dem Tunnel zu erreichen sind       % /Erfahrungen/
    folgende Punkte wichtig (FIXME ref D8.5):

   - Das Buendelende sollte den Tunneleingang deutlich ueberdecken. Es
     muss vermieden werden, dass die Tunnelecken dunkler als die Mitte
     des Tunnels sind. Ein inhomogen ausgeleuchteter Buendeleingang
     fuehrt zu inhomogener Beleuchtung des pupil plane SLM.

   - Das Ende des Faserbuendels muss in vier Achsen justiert werden
     koennen (Zentrierung von Position und Winkel).

   - Die Brennweite der Mikrolinsen sollte kuerzer gewaehlt werden,
     als die Rechnung vorhersagt. Damit kann unweigerlich auftretendes
     Mikrochipping der zementierten Glasspiegel kompensiert werden.

\subsection{ Fourier-optischer Filter zur Kontrasterzeugung am pupil plane SLM}
  - Der micro-mirror array, den wir als pupil plane SLM einsetzen,        % /MMA torsion spiegel/
    besteht aus Torsionsspiegeln, die die Phase des Lichts modulieren
    (fuer eine genauere Beschreibung siehe spaeteres Kapitel               
    FIXME). Um damit eine Intensitaetsmodulation zu bewirken, nutzen
    wir den in Fig 4.2 B gezeigten Fourier filter. 

  - Die Linse L1 hat zwei Aufgaben: Zum einen bildet sie die Feldmaske   % /Schlierenoptiklinse/
    B0 in den Feldstopp B1 ab. Zum anderen wird die Ebene P'' mit dem
    SLM nach unendlich abgebildet.

  - Bei ungekippten Spiegeln, wird somit F''' nach F'' abgebildet und    % /MMA Kontrasterzeugung/
    gleichzeitig gibt es ein scharfes Bild von P'' nach P'. Beide
    Ebenen F'' und P' sind dann homogen ausgeleuchtet.

  - Werden die Spiegel auf der linken Haelfte in P'' gekippt, dann
    lenken sie das Licht entlang der gestrichelten Linie (in Fig 4.1)
    ab. Dieses Licht wird von der Apertur B1 absorbiert und steht dann
    nicht in P' zur verfuegung. D.h. die rechte Seite in P' ist
    dunkel. Der gesamte radiant flux ($\unit[]{W}$) durch die Apertur in
    F'' nimmt ab, die irradiance ($\unit[]{W/m^2}$) ueber die Apertur
    bleibt aber homogen.

  - Im realen System besteht die Linse L1 aus 4 Elementen. Aufgrund
    der Symmetrie weist sie keinen axialen Farbfehler auf. Es bleibt
    jedoch ein kleiner lateraler Farbfehler (FIXME genauer ergruenden
    was das bedeutet).
 

\subsection{ Relais-System zwischen pupil plane und focal plane SLM}
  - Die Linsen L2 und L3 bilden ein doppelt telezentrisches             % /Relais-System/
    Relais-System mit Vergroesserung 2 und bilden F'' auf der Ebene
    des focal plane SLM in F' ab. Gleichzeitig bildet dieses
    Relais-System den pupil plane SLM von P'' nach unendlich ab.
 
  - Prinzipiell koennte man auch den focal plane SLM in F'' an Stelle
    der Apertur B1 platzieren. In unserem Prototypen haben wir uns
    jedoch fuer dieses zusaetzliche Relais-System entschieden, um den
    Kontrast beider SLM voneinander zu entkoppeln.

   - TODO warum haben wir das relay system? 
     - vermutlich weil wir den mma kontrast vom lcos entkoppeln wollen
     - es ist natuerlich fuer sammelnde system, dass axial color sich
       aufaddiert und nicht kompensiert wird


\subsection{ Polarisationsbasierte Kontrasterzeugung am focal plane SLM}
  - Der von uns verwendete focal plane SLM ist ein liquid crystal on
    silicon Geraet, dass die Polarisation des reflektierten Lichts
    entweder um 90 grad dreht oder konstant laesst.
 
  - Ein Polarisationsstrahlteiler erzeugt daraus einen binaeren
    Intensitaetskontrast (siehe Fig 4.1 C).

  - Wir haben uns fuer einen wire-grid Polarizer (Moxtek PBF02C, Orem,
    UT, US) entschieden, weil die Platte weniger Rueckreflexe
    verursacht als ein Strahlteilerwuerfel.

  - Die s-Polarisation des eingehenden Lichts wird in Richtung des SLM
    reflektiert. Aktive Pixel des SLM rotieren die Polarisation des
    Lichts um 90 Grad und passiert dann den Strahlteiler als
    p-Polarisation in Tranmission in richtung Mikroskop. Dort befindet
    sich ein zusaetzlicher Cleanup-Analysator im Strahlengang.
 
  - Es waere auch denkbar, SLM und Strahlteiler anders anzuordnen, so
    dass das vom SLM kommende Licht in das Mikroskop
    \emph{reflektiert} wird. In diesem Fall verschlechtert jedoch eine
    ungewollte Oberflaechendurchbiegung des Strahlteilers die
    Abbildungsqualitaet vom focal plane SLM. Deshalb nutzen wir den
    Strahlteiler in Tranmission.

  - Die duenne Platte (<2mm) des Strahlteilers macht das System leicht
    asymmetrisch und fuehrt damit hauptsaechlich zu Astigmatismus und
    lateral color (ref D8.9 FIXME), das Optikdesign bleibt aber
    beugungsbegrenzt.


\jpginput{14cm}{setup-photo-blueprint}{The wide field epi-fluorescence
  microscope with attached illumination head. The positions of the two
  spatial light modulators (Micro mirror array (MMA) and liquid
  crystal on silicon display (LCoS)) are indicated. Drawing by Josef
  Wenisch (In-Vision, Austria).}


\jpginput{14cm}{memi-setup-only-lenses}{only lenses7}

\subsection{ Variables Teleskop als Tubuslinse}
  - Die groesse der Pupille von Mikroskopobjektiven haengt von deren
    Bildfeld und numerischer Apertur ab. Die letzt Linse
    TL${}_\textrm{ill}$ in unserem Beleuchtungssystem ist daher so
    konzipiert, dass sie P'' mit variabler Vergroesserung nach P
    abbildet. 

  - Die Linse besteht aus drei beweglichen Gruppen und kann somit
    garantieren, dass der pupil plane SLM bei Vergroesserungsaenderung
    stationaer auf der pupil plane des Objektivs abgebildet bleibt und
    gleichfalls der focal plane SLM immer im unendlichen abgebildet
    bleibt (FIXME gibt es ein paper mit begruendung?).





\begin{figure}[!htbp]
   \centering
   \svginput{2}{memi-sketch}
   \caption{Schematic of the lenses in the MEMI system and their focal
     lengths. The focal length $f_\textrm{TL}$ of the tube lens can be
     varied. This allows to scale the second intermediate image
     $r''_\textrm{MMA}$ of the micro mirror array to fit the back
     focal plane of different objectives. Dimensions in mm.}
   \label{fig:memi-sketch}
 \end{figure}




% \imagw{14cm}{mma}{{\bf left:} Scanning electron microscope image of
%   the micro-mirror array (MMA).  The pixel pitch of the device is
%   \unit[0.016]{mm}. The hinges for the tilt movement and the
%   electrodes are clearly visible. {\bf middle:} Optical reflective
%   microscope image of the MMA. {\bf right:} exaggerated rendering of
%   how a 8x8 checker board pattern would be displayed on the
%   device. Electron and optical micrograph by Fraunhofer IPMS Dresden
%   (Germany)}

% \begin{figure}[!hbt]
%   \centering
%   \includegraphics[width=7cm]{mma-plain}
%   \includegraphics[width=7cm]{mma-ill}
%   \caption{{\bf left:} Micro mirror array chip during installation of
%     the optics. {\bf right:}~Illuminated micro mirror array in the
%     aligned system.}
%   \label{fig:mma-closeup}
% \end{figure}

\chapter{optimization of the spatio-angular illumination patterns}
\label{sec:optimization}
\chapter{mma as an intensity modulator}
\label{sec:mma}
\include{device2}
\chapter{experimental results with spatio-angular microscope (device 1)}
\label{sec:results}
\chapter{discussion}
\label{sec:discussion}
- zuerst habe ich dvi lcos mit mma verbaut, das hat leider nur
  gelegntlich funktioniert

- urspruenglich war geplant folgendes system einzusetzen: Biological
  applications of an LCoS-based programmable array microscope (PAM)

- dann habe ich usb lcos eingesetzt, damit geht es immer, ist aber
  langsamer und deutlich weniger nuetzlich zum experimentieren

- ausserdem ist die ettendue des beleuchtungssystems arg
  eingeschraenkt mit einem 63x objektiv (NA=1.47) wird nur ein feld
  mit 40 um durchmesser beleuchtet

- deshalb untersuchten wir einen anderen weg zur kontrasterzeugung und
  lernten dabei dass ein interferometrischer ansatz sehr wohl geeignet
  ist, die ettendue zu erhoehen

  - einschraenkungen in der realisierbaren optik (freier durchmesser
    der nomarski prismen) fuehrte zu nicht ganz ueberzeugenden bildern

  - ein piston mma wuerde zu deutlich besseren ergebnisse fuehren

- ein weiterer ansatz fuer spatio-angular beleuchtung wurde mit einer
  holographischen methode verfolgt

  - dabei lernten wir dass die qualitaet des verwendeten
    phasenmodulators zu wuenschen uebrig laesst

  - einfacherer ansatz mit nur einem display, erfordert daher weniger
    optik und elektronik

  - loest jedoch nicht das problem geringer ettendue (die moegliche
    ettendue muss ich mir genauer ueberlegen, sie haengt mit der
    anzahl der pixel des displays und den grating konstanten zusammen,
    die dargestellt werden koennen, da das system off-axis betrieben
    werden muss, wird die ettendue geviertelt)

- Im Nachhinein muss man sagen, dass es Zielfuehrender gewesen
    waere, und unsere Aufgabe erheblich vereinfacht haette, wenn wir
    beide SLM vom gleichen Typ verwendet haetten. Es handelte sich
    aber um einen Prototypen und er war in den ersten Jahr des
    Projektes noch nicht verfuegbar. Das Projekt wird von Pasteur und
    Fraunhofer, diesmal unter Verwendung zweier ihrere SLM,
    weitergefuehrt. 

  - Leider wird dieser Ansatz unsereserachtens nicht das wesentliche
    Problem der kleinen Ettendue bereinigen und der neue Prototyp wird
    noch immer nicht die interessantesten Experimente erlauben. Es ist
    ganz einfach so, dass es einfacher waere, ein biologisch
    Relevantes Experiment zu designen, wenn das Beleuchtungssystem
    auch die volle Ettendue heutiger Mikroskopobjektive ausschoepft.

- kameras sind zur zeit an einem wendepunkt. vermutlich wuerde man
  heutzutage eine sCMOS benutzen, dann sollte man aber auf die
  triggereigenschaften achten

- arduino war nuetzlich um die elektronische triggerung ohne grossen
  aufwand umzusetzen (der hauptaufwand war oft nicht die
  zeitsteuerung, sondern eine ordentliche galvanische entkopplung der
  displays, die ist auch wichtig)

  - da unterscheiden sich die hersteller ohnehin sehr stark, bei dem
    dvi display war es erforderlich, testpunkte vom board abzugreifen
    und ueber adum zu entkoppeln, bei neueren varianten des usb boards
    kann man mittlerweile einfach einen stecker anstecken

- man kann relativ viel aufwand bei der rekonstruktion von optisch
  geschnittenen bildern betreiben, fuer das reale problem ist die
  vermeidung von artefakten dann oft doch nicht so wichtig (z.b. beads
  oder nuklei lokalisieren)

- transmission ist nicht ausreichend um wuermer zu untersuchen  

- vergleiche die folgenden displays:

  - holoeye (erwaehne triggerversuche, kalibrationsmessungen von
    uebertragungsfunktion und interpixel cross talk, hamamatsu)

  - forthdd (frage sie vielleicht, ob sie mir im nachhinein doch noch
    information geben)

  - ti dmd (sehr gute dokumentation, sehr viele funktionen; gut waere,
    wenn ich ein programm auf dem lokalen arm prozessor laufen lassen
    koennte, was die vollen 4000fps aus runlength (oder irgendwie
    komprimierten) daten vom usb aus erzeugen koennte

    - deflection angle defines f/\# number of projection lens and
      therefore etendue, for good contrast f/\# shouldn't be smaller
      than f/2.8

  - mma (naja)

- hilo ist nicht unbedingt notwendig, ziemlich kompliziert und brauch
  fudge factor

  - high density fluorophore labelling for the shorter wavelength
  fluorophore (better signal noise)

\chapter{outlook}
\label{sec:outlook}
- den algorithmus zur beleuchtungsoptimierung kann man noch deutlich
  verbessern

  - gleichzeitige beleuchtung mehrerer nuklei

  - andere objektstrukturen (z.b. zylinder, axone)

    - 2010 hermann cuntz: One Rule to Grow Them All: A General Theory
      of Neuronal Branching and Its Practical Application

      - modell wie neuronen wachsen um axon oder dendritendichte
        vorherzusagen

  - voxels05\_final

- eine genaue analyse einiger probleme mit wellenoptischer partiell
  kohaerenter theorie steht noch aus und waere interessant (nach
  wichtigkeit)

  - partiell kohaerente simulation des mma im schlierenoptischen system

    - sind graulevel vorteilhaft?

    - wuerde ein mma, bei dem alle spiegel in dieselbe richtung kippen
      die ettendue verdoppeln?

  - partiell kohaerente simulation des mma im shearing
    interferometrischen system

    - was ist die maximale ettendue eines wollaston prismas?

  - holographie methode mit extended source

  - Denkbar waere auch ein scannendes konfokales Mikroskop, dass an
    die Beleuchtungswinkel an jedem Punkt kontrolliert (siehe
    fig:hourglass-all-b).  Bisher wurden in der Literatur nur Systeme
    beschrieben, die die Phase des Beleuchtungslicht in der Pupille
    aendern (FIXME ref). Eine Adaption dieser Systeme zu einem
    spatio-angularen ist naheliegend und ich schlage vor, derartige
    Systeme auch untersucht werden sollten. Die Kombination von CLEM,
    einem Ringdetektor (vielleicht mit UZI) koennte die Bildgebung im
    Inneren lebender Organe (z.B. Gehirn) verbessern.


%\include{app_term}
\bibliographystyle{abbrvnat}
\bibliography{literature}
%\bibliography{../All}
\end{document}


%%% Local Variables: 
%%% mode: latex
%%% TeX-master: t
%%% End: 
