% ~/from-hp2-notebook/0331/lens
% there is also code
\chapter{Raytracing for spatio-angular microscopy}
\label{sec:raytrace}
\renewcommand{\i}{\nvect i}
\begin{summary}
  Imaging with our microscope requires that during the observation of
  the sample new light patterns are created continuously and as fast
  as possible. We can not easily solve this problem by using a
  commercial software and I therefore decided to implement a simple
  raytracer.

  This chapter documents the basic concepts. Note that some
  approximations, which are usually used in optical design (paraxial,
  only non-skew rays) are not applicable here, because rays are to be
  pursued in all possible angles through diverse fluorophore
  distributions in samples.

FIXME translate

Ich beginne mit einfachen geometrischen Formeln um die Schnittpunkte
eines Strahls mit einer Ebene oder Kugel zu bestimmen und um die
Brechung an einer Oberflaeche auszurechnen.

Dann erklaere ich die Brechung an einer duennen paraxialen Linse und
zeige eine Modifikation der Formeln fuer Hochaperturlinsen. Damit
zeige ich wie man vorwaerts und rueckwaerts durch ein
Mikroskopobjektiv rechnen kann, ohne dessen genaue Designparameter und
Glassorten zu kennen.

Furthermore, I consider the refraction at the coverslip--medium
interface for non-index matched media. This enables the calculation of
illumination patterns for highly inclined and laminated optical sheet
microscopy, as introduced in section \ref{sec:hilo}.

FIXME translate: Zum Schluss diskutiere ich noch eine geometrische
Betrachtung, die im speziellen Fall von in Kugeln verteilten
Fluorophoren dabei Hilft, die Anzahl der fuer eine Simulation durch
das System zu tracenden Strahlen erheblich zu verkleinern.

{\bf Note:} The formulas in frames are implemented in the computer
code that is published on
\url{https://github.com/plops/mma/tree/master/lens}.
\end{summary}
\section{Geometrische Grundlagen}
\subsection{Intersection of a ray and a plane}
Let a ray start at a point $\s$ with direction $\hd$.  A plane
(defined by a point $\c$ and the normal $\n$) intersects this ray if
normal and ray direction are not perpendicular: $\n\,\hd\not=0$. The
distance between the plane and the origin is $h=\c\n$. I define the
plane equation in Hesse normal form:
\begin{align}
  \r\n=h
\end{align}
I replace the coordinate $\r$ with the ray equation and solve for
the parameter $\tau$:
\begin{align}
  (\s+\tau\hd)\n&=h\\
  \s\n+\tau\hd\n&=h\\
  \tau&=\boxed{\frac{h-\s\n}{\hd\n}}
\end{align}
 \begin{figure}[!hbt]
   \centering
   \svginput{1}{plane-intersection}
   \caption{Schematic of plane-ray intersection.}
 \end{figure}
\subsection{Intersection of a ray and a sphere}
Let a ray start at a point $\s$ with direction $\hd$.  Let a sphere be
centred in $\c$ with radius $R$. Their two equations
\begin{align}
  (\r-\c)^2&=R^2\\
  \r&=\s+\tau\hd
\end{align}
define the intersection points. Substitution of $\r$ results in a
quadratic equation for $\tau$:
\begin{align}
  (\s+\tau\hd-\c)^2&=R^2\\
  \l&:=\boxed{\s-\c}\\
  l^2+2\tau\l\hd+\tau^2-R^2&=0\\
  \underbrace{1}_a\tau^2+\underbrace{2\l\hd}_b\tau+\underbrace{l^2-R^2}_c&=0
\end{align}
\subsubsection{Solving the quadratic equation}
In order to prevent numerical errors the following solution should be
used \citep{Press1997}:
\begin{align}
  \Delta&:=\boxed{b^2-4ac}\\
  q&:=\boxed{-\frac{b+\sqrt{\Delta}\sign b}{2}}\\
  \tau&=\boxed{
  \begin{cases}
    q/a &\,\textrm{when}\,\abs{q}\approx 0\\ 
    c/q &\,\textrm{when}\,\abs{a}\approx 0\\
    (q/a, c/q) &\,\textrm{else}
  \end{cases}}
\end{align}
If the discriminant $\Delta$ is negative the ray misses the sphere and
there is no solution. If the discriminant is zero the ray touches the
periphery and there is only one solution. A positive discriminant
corresponds to two solutions.
\subsection{Refraction at planar surface}
Now I describe the refraction at a planar surface\footnote{I use the
  same notation as \cite{McClain1993}.}. The wavelength of the light
in vacuum defines the length of the wave vector $\k_0$. The lengths of
the incident and transmitted wave vectors $\k_1$ and $\k_2$ are given
by the refractive index in their respective half space:
\begin{align}
  k_0&=2\pi/\lambda\\
  k_1&=n_1 k_0\\
  k_2&=n_2 k_0.
\end{align}
I choose the normal $\n$ to be directed into the half-space of the
incident wave (see \figref{fig:refraction-plane}) and define the
transversal and normal component of the wave vectors:
\begin{align}
  \k_{1n}&=(\k_1\n)\n\\ 
  \k_{1t}&=\k_1 - \k_{1n}.
\end{align}
The two component vectors are orthogonal and during refraction the
transversal component of the wave vector is invariant:
\begin{align}
  k_2^2&=k_{2n}^2 + k_{2t}^2\\
  \k_{2t}&=\k_{1t}.
\end{align}
Using the two equations from above one can calculate the length of the
normal component of the transmitted wave vector $\k_2$:
\begin{align}
  k_2^2&=k_{2n}^2 + (\k_1 - \k_{1n})^2\\
  k_{2n}^2&=k_2^2-(\k_1-(\k_1\n)\n)^2\\
  &= k_2^2-(k_1^2-2(\k_1\n)^2+(\k_1\n)^2)\\
  &= k_2^2-k_1^2+(\k_1\n)^2.
\end{align}
Finally one can express the full transmitted wave vector $\k_2$ using
only known quantities:
\begin{align}
  \k_2&=\k_{1t}-\sqrt{k_2^2-k_1^2+(\k_1\n)^2}\n\\
  &=\k_1-(\k_1\n)\n-\sqrt{k_2^2-k_1^2+(\k_1\n)^2}\n.
\end{align}
I divide by $k_2$ with $\k_2/k_2=\t$ and $\k_1/k_2=\eta\,\i$ in order
to introduce unit direction vectors $\i$ and $\t$ for incident and
outgoing light. The relative index change across the interface is
$\eta=n_1/n_2$.
\begin{figure}
  \centering
  \svginput{1}{refraction}
  \caption{Refraction at an interface transforms the incident wave
    vector $\k_1$ into the outgoing wave vector $\k_2$.}
  \label{fig:refraction-plane}
\end{figure}
\begin{align}
  \t&=\eta\i-\eta(\i\n)\n-\sqrt{1-\eta^2+\eta^2(\i\n)^2}\n\\
  &=\boxed{\eta\i-\left(\eta\i\n+\sqrt{1-\eta^2(1-(\i\n)^2)}\right)\n}
\end{align}
When the radical under the square root is negative a reflection occurs
instead (TIRF).

 FIXME translate: Bemerke das Totale Interne Reflexion in meinem
 Anwendungsfall prinzipiell nur mit dem Verlust des Strahls
 einhergeht, denn er traegt nicht mehr zur Samplebeleuchtung bei. Die
 exakte des reflektierten Strahls Richtung ist in diesem Fall
 irrelevant. Ich schreibe sie der Vollstaendigkeit halber aber hier
 mit auf.


 Then the tangential component is invariant and normal
component inverts the sign:
 \begin{align}
   \k_2&=\k_{1t}-\k_{1n}\\
   &=\k_1 - 2\k_{1n}\\
   &=\k_1-2(\k_1\n)\n\\
   \t_\textrm{TIR}&=\boxed{\i-2(\i\n)\n}
 \end{align}
\section{Refraction on lenses}
FIXME translate:

\cma{Gueltigkeitsbereich des Modells} Eine ideale Linse ist
infinitesimal duenn und definiert durch ihre Fokuslaenge, also dem
Abstand zwischen der Linse und der Ebene, in der parallel einfallende
Strahlen sich schneiden. Fuer eine ideale Linse ist die Fokuslaenge
unabhaengig davon, unter welchem Winkel Strahlen einfallen. In der
Praxis gilt das Modell der duennen Linse nur fuer sehr lange
Fokuslaengen beispielsweise eine 3mm dicke Linse und 200mm
Fokuslaenge, dann jedoch auch nur fuer Strahlen die unter einem sehr
kleinen Winkel zur optischen Achse verlaufen (paraxiale Strahlen). 

\cma{Hauptebenen} Fuer eine dicke Linse kann man den Strahlverlauf auf
eine duenne Linse zurueckfuehren, wenn man die zwei Hauptebenenen der
dicken Linse bestimmt und den Strahl dazwischen einfach axial
verschiebt \cite{Smith2000}. Die Hauptebene einer dicken Linse
befindet sich auf dem Schnittpunkt eines parallel zur optischen Achse
einfallenden Strahls $\i$ und des austretenden Strahls $\r$. Wie auch
die Fokuslaenge ist auch die Hauptebene eine Linseneigenschaft, die
fuer reale Linsen nur im paraxialen Fall gilt. An einer dicken Linse
gibt es zwei Hauptebenen, fuer die zwei moeglichen Richtungen von
denen man die parallelen Strahlen hindurchsenden kann. Der Abstand
zwischen einer Hauptebene und dem dazugehoerigen Fokuspunkt (dem
Schnitt von $\r$ mit der optischen Achse) ist identisch und die
Fokuslaenge $f$.

Wie bereits auf Seite \pageref{aplanatic} erwaehnt, kann man
Mikroskopobjektive als eine Linse auffassen, bei der die Fokuslaenge
unabhaengig vom Einfallswinkel ist. In diesem Fall ist die Hauptebene
keine Ebene mehr sondern wird zu einer Kugeloberflaeche deformiert. In
der Sektion \ref{sec:high-aperture-lens} zeige ich wie man die Formeln
vom Modell der duennen Linse entsprechend fuer eine aplanatische Linse
mit Immersion anpassen kann.


\subsection{Refraction on paraxial thin lens}
FIXME translate:
Zuerst beschreibe ich die Brechung an einer duenne Linse.

The incident beam with direction $\i$ hits the lens at the point
$\vrho$. A line parallel to $\i$ through the centre of the lens
defines the point on the focal plane, which will be intersected by the
transmitted ray $\r$ as well.


\begin{figure}[hbtp]
  \centering
  \svginput{1}{lens-fwd}
  \caption{Construction of a ray on a thin lens. The incident beam
    with direction $\i$ (from right) hits the lens at the point
    $\vrho$. FIXME translate: Diese Darstellung ist einer Abbildung in
    \citep{Hwang2008} nachempfunden.}
\end{figure}


The triangle $ABC$ is similar to triangle $FOA$. All three angles are
identical because each of the lines are parallel:
$\overline{CB} \parallel \overline{OA} \parallel \vrho$,
$\overline{FA} \parallel \overline{CA}$ and $\overline{AB} \parallel
\overline{OF} \parallel \i$. The side $\overline{OF}$ is hypotenuse of
a right angled triangle. Its adjacent with respect to the angle
$\theta$ has length $f$. Therefore one can deduce the length
$\abs{\overline{OF}}=f/\cos\theta$.



Between the two similar triangles, the following relation holds and
can be used to calculate the length $\abs{\overline{BC}}$:
\begin{align}
  \frac{\abs{\overline{BC}}}{\abs{\overline{BA}}}&=
  \frac{\abs{\overline{OA}}}{\abs{\overline{OF}}}\\
  \frac{\abs{\overline{CB}}}{1}&=
  \frac{\rho}{f/\cos(\theta)}.
\end{align}
Given its length, the vector $\vv{CB}$ can now be calculated, because
its direction is known to be along $\vrho$. With this vector and $\i$
one can now obtain the (arbitrarily scaled) transmitted vector
$\r'$. I could normalize it but it turns out to be useful for the high
NA immersion lens to deliver the vector $\r$, that ends in the focal
plane.  The procedure from above is condensed in the following
equations:
\begin{align}
  \vrho&=(x_0,y_0,0)^T=\rho (\cos\phi,\sin\phi,0)^T\\
  \phi&=\arctan(y_0/x_0)\\
  \cos\theta&=\boxed{\i\,\hz}\\
  \r'&=\i- \frac{\cos\theta}{f}\vrho\\
  \r&=\boxed{\frac{f}{\cos\theta} \i -\vrho}
\end{align}
with the axial unit vector $\hz=(0,0,1)^T$.
\subsection{Refraction through high aperture objective (illumination)}
\label{sec:high-aperture-lens}
Now I augment the results of the calculation from the previous chapter
to treat an aplanatic immersion objective \citep{Hwang2008}.
\begin{figure}[!hbt]
  \centering
  \svginput{1}{obj-fwd}
  \caption{Construction of a ray on an high numerical aperture oil
    immersion objective. As opposed to a thin air lens the objective's
    focal length needs to be corrected by the focus difference vector
    $\a$ to accommodate for the immersion and one must take into
    account spherical principal surface (aplanatic surface) .}
\end{figure}
I account for the immersion medium by shifting the focal plane in
sample space to $nf$ using the focus difference vector $\a$, i.e.\ for
an immersion lens with $n=1.52$ the focus moves further away from the
principal plane.
\begin{align}
  \a &= \boxed{f (n-1) \hz} \\
  R &= \boxed{nf}
\end{align}
In order to account for the curvature of the aplanatic surface, the
origin of the transmitted ray is axially shifted by a $\rho$ dependent
sag $\s$ from the principal plane onto the aplanatic surface:
\begin{align}
  \s &= \left(R - \sqrt{R^2-\rho^2}\right) \hz
\end{align}
The final ray exiting the objective has the direction $\r_0$:
\begin{align}
  \r_0 &= \boxed{\r + \a - \s}.
\end{align}

FIXME translate: Alle fuer unsere Bildgebung in Frage kommenden
Mikroskopobjektive sind als aplanatisches System ausgelegt und nur am
Rand des Feldes werden Abweichungen auftreten, Dieses Modell ist daher
sehr gut geeignet um unsere Objektive waehrend der
Beleuchtungsoptimierung zu repraesentieren.

In the paper \citet{Hwang2008} the authors demonstrate the viability
of this model by comparing its results with a full raytrace through a
$100\times\,1.41$ objective. Focus displacement errors are less than
\unit[130]{nm} for a field of $\unit[86.4]{\mu m}$ radius.

Das ist vollkommen ausreichend fuer unsere Anwendung.  Man koennte
meinen, es waere gut die genauen Objektivparameter zu kennen (also
Glaeser sowie die Curvature und die Scheitelpositionen der
Linsenoberflaechen). Diese Angaben werden jedoch von keinem Hersteller
veroeffentlicht. Ausserdem spielen in Mikroskopobjektiven die
Zentrierfehler eine herausragende Rolle, so dass das Design an sich
vermutlich kaum eine bessere Aussage als dieses Modell
zulaesst. Moeglicherweise koennte man unser spatio-angulares Mikroskop
vermutlich sogar dazu einsetzen, um die Performance eines
Mikroskopobjektivs zu vermessen.

\subsection{Reverse path through oil objective (detection)}
Now I consider the oil objective in the reverse direction (see
\figref{fig:obj-ref-full}). Given a ray that starts within the sample
I will determine the ray in the pupil.

FIXME translate: Dafuer habe ich mir nacheinander zwei Ansaetze
ueberlegt. Der erste und einfachere nutzt aus, dass ein perfektes
Mikroskopobjektiv die Strahlwinkel aus dem Sample in linearer Weise in
Positionen auf der Pupille umwandelt. Dieser Ansatz reicht fuer die
Berechnung von pupil plane SLM patterns aus, wenn sich das Sample in
index-matched embedding medium befindet.

Im zweiten Ansatz berechne ich zusaetzlich die Winkel der aus der
Pupille austretenden Strahlen. F\"ur ein perfekt aplanatisches
Objektiv bringt das zwar keine Vorteile, die Formeln lassen sich dann
aber aendern, um auch Aberrationen zu beruecksichtigen.

\subsubsection{Easy case: back focal plane positions only}
If the intersections of rays in the back focal plane are sufficient, a
full raytrace is not necessary. This is the case when imaging beads in
index matched embedding medium and we want to calculate a pattern for
the pupil plane SLM (MMA). Instead it is sufficient to ignore ray
origins and just consider their directions in the sample.

A unit ray direction $\i=(x,y,z)^T$ in sample space is transformed
into a position $\r_b=(x',y')^T$ in the back focal plane of the
objective. The azimuthal angle $\phi$ isn't changed when going through
the objective. The polar angle $\theta$ defines how far off axis the
back focal plane is hit.
\begin{align}
  \phi'&=\phi=\arctan(y/x)\\
  \theta&=\arcsin(\sqrt{x'^2+y'^2})\\
  r_b&=nf\sin\theta\\
  \r_b&=r_b(\cos\phi',\sin\phi')^T
\end{align}
 \begin{figure}[!hbt]
   \centering
   \svginput{1}{obj-rev}
   \caption{Schematic for tracing a ray direction $\i$ from sample
     space into the back focal plane. The bigger the angle between
     $\i$ and the optical axis, the further outside the ray will pass
     through the back focal plane.}
 \end{figure}
 \subsubsection{Full raytrace through oil objective in detection
   direction}
\label{sec:objective-raytrace-detection}
Now I discuss the general case and calulate both, the origins and the
directions of rays emerging from the back focal plane. This is
necessary in order to trace light bundles from the specimen into the
plane of the camera (or focal plane SLM). In the next section I will
further modify these formulas to incorporate aberrations due to
non-index matched embedding medium.

The position of the objective is defined by its principal point $\c$
and the normal $\n$ (directed along optical axis towards sample
space). The incident ray is defined by its starting point $\p$ and the
direction $\i$. First I calculate the centre of the aplanatic sphere
$\vect g$:
\begin{align}
  \vect g &= \c + nf\, \n.
\end{align}
Then I obtain the position $\p'$ by intersecting the incident ray and
the plane perpendicular to the optical axis through $\vect{g}$.  The
focus difference vector is defined by its length and the optical
axis. It can be used to calculate an intermediate point $\p''$.
\begin{align}
  \a &= -f\, (n-1)\,\n \\
  \p'' &= \p' + a.
\end{align}
The point $\p''$ has been shifted, so that an aplanatic air lens would
image it exactly as the oil objective would image $\p'$. One can use
$\p''$ to find the direction $\t$ of the transmitted ray. It is just
the normalized difference vector $\vect m$ to the principal point $\c$.
\begin{align}
  \vect m &= \c - \p'' \\
  \t &= \vect m / \abs{\vect m}.
\end{align}
As a last step I calculate the starting point $\e'$ of the transmitted
ray by intersecting the incident ray with the aplanatic sphere (in
point $\e$) and axially shifting this point onto the principal plane.

FIXME translate: Bemerkung: Um die Richtigkeit der Implementierung der
Formeln am aplanatischen Objektiv zu ueberpruefen koennen die
voneinander unabhaengigen Algorithmen dieser Sektion (in
Detektionsrichtung) mit Sektion \ref{sec:high-aperture-lens} (in
Beleuchtungsrichtung) miteinander verglichen werden.

\begin{figure}[!hbt]
  \centering
  \svginput{1}{obj-rev-full}
  \caption{Construction to find the transmitted ray through an oil
    immersion objective from a point within the sample.}
  \label{fig:obj-ref-full}
\end{figure}
\subsection{Treatment of aberration (detection)}
\label{sec:ray-aberration}
Now I will extend the formulas of the previous section to include
aberrations due to a non-matched embedding medium $n_e\not=n$.

I consider a ray originating in point $\p$ with direction $\i$ within
an embedding medium of index $n_e$. I determine the intersection $\f$
of the ray with the coverslip--embedding interface and refract to
obtain $\i'$. Then I calculate the time $t$ a photon takes, to travel
from $\p$ to the interface $\f$:
\begin{align}
  t = \abs{\f - \p} \frac{n_e}{c}
\end{align}
and extend the path of the photon backward along the direction $\i'$
(corrected for the refraction at the coverslip--embedding surface) by
the distance $tc/n$. This results in the corrected position $\p'$ that
indicates where the photon would have originated if the embedding
would have been index matched.  Now I can apply the equations from the
previous section on the ray defined by $\p'$ and $\i'$ to obtain the
transmitted ray in the pupil.

 \begin{figure}[!hbt]
   \centering
   \svginput{1}{obj-rev-full-emb}
   \caption{Construction of an oil immersion objective with a
     non-index matched embedding medium.}
 \end{figure}
\section{Sphere projection}
\label{sec:sphere-projection}
While the previous sections have described a fairly general raytracer,
this section is very technical and relates to our specific problem to
represent the sphere model of a fluorophore distribution with as few
rays as possible.

FIXME translate:

Abbildung fig:touch-cone~D) zeigt eine Darstellung der
Interaktionslaenge von Strahlen mit out-of-focus Nuklei, die auf allen
Punkten der Pupille, die durch den Zielpunkt $\c$ gehen. Die
Berechnung eines solchen Bildes erfordert die Berechnung einer
Vielzahl an Strahlen (mindestens 50x50) durch das Objektiv. Um den
Rechenaufwand zu verringern, kehre ich die Strahlrichtung um. Ich
ermittle Strahlen von der Periphery jedes out-of-fokus Nukleus durch
den Zielpunkt $\c$ und bestimme die entsprechenden ``Schattenmasken''
auf der Pupille in fig:touch-cone~E). Durch diesen Ansatz kann man mit
deutlich weniger Strahlen (es reichen 7 Strahlen pro out-of-fokus
Nukleus) hinreichend gute Masken ermitteln.

Jetzt erklaere ich anhand der in fig:touch-cone~A) dargestellten
Geometrie, wie ich die Punkte auf auf der Peripherie der Nuklei
selektiere, um diese Berechnung zu ermoeglichen.

The tangents of an out-of-focus sphere
{\color[rgb]{0.06666667,0.50196078,0}$S^\s_r$}
centred at $\s$ with radius $r$ that pass through the target $\c$ form
a double cone (assuming $\c$ is outside of $S^\s_r$. The tangents
touch the surface of the sphere $S^\s_r$ at the circular intersection
{\color[rgb]{0.66666667,0,0}{$C$}} with the sphere
{\color[rgb]{0,0.26666667,0.66666667}$S^\c_r$}
centred at $\c$ with radius $R=\abs{\c-\s}$. Radius $R$ is the
distance from the target to the centre of the out-of-focus sphere.
\begin{figure}[!hbt]
  \centering
  \svginput{1}{touch-cone}
  \caption{{\bf (A)} Schematic of how an out-of-focus nucleus and a
    target point $c$ (not necessarily in the centre of a target
    nucleus) define a cone of tangential rays. {\bf (B)} FIXME
    translate: Darstellung mitsamt Objektiv um das Problem zu
    veranschaulichen. {\bf (C)} Beispielverteilung mit sechs
    fluoreszenten Kugeln die fuer D und E benutzt wurde. {\bf (D)}
    Darstellung der Pupille mit exakt ermittelter Interaktionslaenge
    der Strahlen durch $\c$ mit out-of-focus Nuklei. {\bf (E)} Mit der
    in dieser Sektion vorgestellten Approximation der
    Interaktionslaenge.}
  \label{fig:touch-cone}
\end{figure}
In order to find a point $\e$ where a tangent touches the out-of-focus
sphere, it is sufficient to solve the following equation in a 2D
coordinate system with the origin in the centre $\s$ of the
out-of-focus sphere:
\begin{align}
  (x-R)^2+y^2&=R^2\\
  x^2+y^2=r^2
\end{align}
There are two solutions:
\begin{align}
  x_1&=\frac{r^2}{2R}\label{eqn:x1}\\ 
  y_{1/2}&=\pm\frac{r}{2R}\sqrt{4R^2-r^2} \label{eqn:y1}
\end{align}
In the case $R<r$ the out-of-focus nucleus is intersecting the target,
obviating the reason to do the projection in the first place.

I construct two vectors $\hx$ and $\hy$ that span the coordinate
system, in order to transform the solution from 2D into 3D. The
(unnormalized) direction $\x$ of the x-axis of this coordinate system
is given by the difference vector of the target $\c$ and the nucleus
centre $\s$. The direction $\y$ must be perpendicular to $\x$ and is
obtained by calculating the cross product with another vector $\q$.
I ensure that $\q$ and $\x$ are not colinear. The vectors $\q$ and
$\x$ are colinear, when the absolute value of their scalar product
equals the square of the length $\abs{\q\x}=\x^2$.
\begin{align}
  \x&=\c-\s\\
  \q&=\begin{cases}
    (0,0,1)^T & \textrm{when}\ \abs{x_z}<\frac{2}{3}\abs{\x}\\
    (0,1,0)^T & \textrm{else}
  \end{cases}\\
  \y&=\x\times\q \\
  \hx&=\x/\abs{\x}\\
  \hy&=\y/\abs{\y}
\end{align}
Now I can sample the intersection circle $C$ in order to create
viable starting points $\e$ for tangential rays.  Let $M_\phi^\hc$ be
a rotation matrix that rotates a vector by angle $\phi$ around an axis
$\hc$. A point $\e$ on the circle is then defined using one solution
from equations \ref{eqn:x1} and \ref{eqn:y1}. The ray direction $\f$
is then easily obtained:
\begin{align}
  \e(\phi)&=\s+x_1\hx+y_1M_\phi^\hx\,\hy\\
  \f(\phi)&=\c-\e.
\end{align}
Tracing a sufficient number of rays (e.g.\ 7) with direction $\f$ for
different angles $\phi$ to the back focal plane gives the projection
of the intersection circle $C$. Note that this projection in general
is not a circle anymore.

For practical reasons it is useful to project vector $\x$ as well. It
can be used as the centre of the (distorted) shape on the back focal
plane to rasterize it as a fan of triangles.

\section{Conclusion}
FIXME translate:

In diesem Kapitel habe ich einen Ueberblick ueber den von mir
entwickelten Raytracer gegeben. Dieser ist speziell auf mein Problem
fuer die Abbildung mit einem aplanatischen Objektiv zugeschnitten.
Die Berechnung wurde optimiert, so dass ohne weiteres gute
Beleuchtungsmuster in Echtzeit ermittelt werden koennen. 

Weiterhin habe ich die Formeln erweitert, um auch die Aberration zu
beruecksichtigen, die auftritt wenn ein Sample nicht index-matched
eingebettet wurde. Dies verringert zwar die bei der Detektion
moeglichen Aufloesung schon bei geringen Eindringtiefen erheblich,
ermoeglicht aber eine interessante Art der Beleuchtung mit einem
highly inclined and laminated optical sheet (see section
\ref{sec:hilo}). In diesem Fall wird durch ein Fenster am Rand der
pupille beleuchtet, so dass die Strahlen die Grenzflaeche zum Medium
nahezu im kritischen Winkel der Totalreflexion erreichen und sehr
steil durch das Medium gehen. Um die korrekte Position im Feld
auszuleuchten, muss das auf dem focal plane SLM angezeigte Fenster
entsprechend verschoben werden um die auftretende (vorallem
sphaerische) Aberration zu kompensieren.

Note that ray optics are not an sufficient approximation, when
intensity features in the scale of the wavelength are to be
investigated. Small features would mean that only a few pixels of the
focal plane SLM would be enabled. This would mean that information of
the pupil plane SLM pattern is heavily filtered and no simultaneous
tight angular control would be possible.

%FIXME maybe compare to ./cyberpower-store/0314/zeiss-patents/20080106795-correction-ring.pdf 
%or US7268953-63x.pdf

%%% Local Variables: 
%%% mode: latex
%%% TeX-master: "kielhorn_memi"
%%% End: 


