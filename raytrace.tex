% ~/from-hp2-notebook/0331/lens
% there is also code
\chapter{Raytracing for spatio-angular microscopy}
\label{sec:raytrace}
\renewcommand{\i}{\nvect i}
\begin{summary}
  Imaging with our microscope requires a continuous update of the
  patterns for the spatial light modulators during operation. The
  generation of those patterns should be as fast as possible. We can
  not easily solve this problem by using commercial software and I
  therefore decided to implement a simple raytracer.

  This chapter documents the basic concepts. Note that some
  approximations, which are usually used in optical design (paraxial,
  only non-skew rays) are not applicable here, because rays are to be
  pursued in all possible angles through diverse fluorophore
  distributions in the specimen.

  I begin by introducing simple geometric formulas to determine the
  points of intersection between a ray and a plane or a sphere. I also
  describe how to calculate refraction at a planar surface.

  Then I explain the refraction at a thin, paraxial lens and show a
  modification of the formulas for high aperture lenses. This allows
  to trace rays through a microscope objective in two directions
  (denoted as detection or illumination direction) without knowing the
  exact design parameters and glass types.

  Furthermore, I consider the refraction at the ``cover slip--medium''
  interface for non-index matched media. This enables the calculation
  of illumination patterns for highly inclined and laminated optical
  sheet microscopy, as introduced in section \ref{sec:hilo}.

  I follow up with a rather technical discussion of a geometric
  problem that helps to significantly speed up the raytracing
  calculations for the specific case of a sample that can be
  represented as a three-dimensional distribution of fluorescent
  spheres.

  {\bf Note:} The formulas that are emphasized by surrounding frames
  are implemented in the computer code that is published on
  \url{https://github.com/plops/mma/tree/master/lens}.
\end{summary}
\section{Basic geometric algorithms}
\subsection{Intersection of a ray and a plane}
 \begin{figure}[!hbt]
   \centering
   \svginput{1}{plane-intersection}
   \caption{Schematic of plane-ray intersection.}
\end{figure}
Let a ray start at a point $\s$ with direction $\hd$.  A plane
(defined by a point $\c$ and the normal $\n$) intersects this ray if
its normal and the ray's direction are not perpendicular:
$\n\,\hd\not=0$. The distance between the plane and the origin is
$h=\c\,\n$. I define the plane equation in Hesse normal form:
\begin{align}
  \r\n=h
\end{align}
I replace the coordinate $\r$ with the ray equation and solve for the
parameter $\tau$.
\begin{align}
  (\s+\tau\,\hd)\,\n&=h\\
  \s\n+\tau\,\hd\,\n&=h\\
  \tau&=\boxed{\frac{h-\s\,\n}{\hd\,\n}}
\end{align}
The point of intersection is on the ray at $\s+\tau\,\hd$.
\subsection{Intersection of a ray and a sphere}
Let a ray start at a point $\s$ with direction $\hd$.  Let a sphere be
centred in $\c$ with radius $R$. Their two equations
\begin{align}
  (\r-\c)^2&=R^2\\
  \r&=\s+\tau\hd
\end{align}
define the intersection points. Substitution of $\r$ results in a
quadratic equation for $\tau$:
\begin{align}
  (\s+\tau\hd-\c)^2&=R^2\\
  \l&:=\boxed{\s-\c}\\
  l^2+2\tau\l\hd+\tau^2-R^2&=0\\
  \underbrace{1}_a\tau^2+\underbrace{2\l\hd}_b\tau+\underbrace{l^2-R^2}_c&=0
\end{align}
\subsubsection{Solving the quadratic equation}
In order to prevent numerical errors the following solution should be
used \citep{Press1997}:
\begin{align}
  \Delta&:=\boxed{b^2-4ac}\\
  q&:=\boxed{-\frac{b+\sqrt{\Delta}\sign b}{2}}\\
  \tau&=\boxed{
  \begin{cases}
    q/a &\,\textrm{when}\,\abs{q}\approx 0\\ 
    c/q &\,\textrm{when}\,\abs{a}\approx 0\\
    (q/a, c/q) &\,\textrm{else}
  \end{cases}}
\end{align}
If the discriminant $\Delta$ is negative the ray misses the sphere and
there is no solution. If the discriminant is zero the ray touches the
periphery and there is only one solution. A positive discriminant
corresponds to two solutions.
\subsection{Refraction at planar surface}
Now I describe the refraction at a planar surface\footnote{I use the
  same notation as \cite{McClain1993}.}. The wavelength of the light
in vacuum defines the length of the wave vector $\k_0$. The lengths of
the incident and transmitted wave vectors $\k_1$ and $\k_2$ are
obtained by muliplication with the refractive index in their
respective half space:
\begin{align}
  k_0&=2\pi/\lambda\\
  k_1&=n_1 k_0\\
  k_2&=n_2 k_0.
\end{align}
\begin{figure}
  \centering
  \svginput{1}{refraction}
  \caption{Refraction at an interface transforms the incident wave
    vector $\k_1$ into the outgoing wave vector $\k_2$.}
  \label{fig:refraction-plane}
\end{figure}
I choose the normal $\n$ to be directed into the half-space of the
incident wave (see \figref{fig:refraction-plane}) and define the
transversal and normal component of the wave vectors to be:
\begin{align}
  \k_{1n}&=(\k_1\n)\n\\ 
  \k_{1t}&=\k_1 - \k_{1n}.
\end{align}
These two vectors are orthogonal and during refraction the transversal
component of the wave vector is invariant:
\begin{align}
  k_2^2&=k_{2n}^2 + k_{2t}^2\\
  \k_{2t}&=\k_{1t}.
\end{align}
Using the two equations from above, one can calculate the length of
the normal component of the transmitted wave vector $\k_2$:
\begin{align}
  k_2^2&=k_{2n}^2 + (\k_1 - \k_{1n})^2\\
  k_{2n}^2&=k_2^2-(\k_1-(\k_1\,\n)\,\n)^2\\
  &= k_2^2-(k_1^2-2(\k_1\,\n)^2+(\k_1\,\n)^2)\\
  &= k_2^2-k_1^2+(\k_1\,\n)^2.
\end{align}
Finally, one can express the full transmitted wave vector $\k_2$ using
only known quantities:
\begin{align}
  \k_2&=\k_{1t}-\sqrt{k_2^2-k_1^2+(\k_1\,\n)^2}\n\\
  &=\k_1-(\k_1\n)\n-\sqrt{k_2^2-k_1^2+(\k_1\,\n)^2}\n. \label{eq:k2}
\end{align}
I divide by $k_2$ with $\k_2/k_2=\t$ and $\k_1/k_2=\eta\,\i$ in order
to introduce unit direction vectors $\i$ and $\t$ for incident and
outgoing light. The relative index change across the interface is
$\eta=n_1/n_2$. With these substitions equation (\ref{eq:k2}) becomes:
\begin{align}
  \t&=\eta\,\i-\eta\,(\i\,\n)\,\n-\sqrt{1-\eta^2+\eta^2\,(\i\,\n)^2}\,\n\\
  &=\boxed{\eta\,\i-\left(\eta\,\i\,\n+\sqrt{1-\eta^2(1-(\i\,\n)^2)}\right)\n}
\end{align}
When the expression under the square root is negative a reflection
occurs instead (TIRF). Note that in my application total internal
reflection just corresponds to a loss of the beam, for it no longer
contributes to sample illumination. The exact direction of the
reflected beam is not relevant in this case but I write them here for
completeness' sake.

In the case of a reflection the tangential component is invariant and
normal component inverts the sign:
 \begin{align}
   \k_2&=\k_{1t}-\k_{1n}\\
   &=\k_1 - 2\k_{1n}\\
   &=\k_1-2(\k_1\,\n)\,\n\\
   \t_\textrm{TIR}&=\boxed{\i-2(\i\,\n)\n}
 \end{align}
\section{Refraction through lenses}

An \cma{validity of thin lens model} ideal lens is infinitesimally
thin and defined by its focal length, i.e.\ the distance between the
lens and the plane through the point in which all the rays of a
focused parallel bundle converge. For an ideal lens the focal length
is independent of the incidence angle but in practice, the model of
the thin lens is only valid for lenses of long focal length, e.g.\ a
\unit[3]{mm} thick lens with \unit[200]{mm} focal length and for
paraxial rays that subtend very small angles from the optical axis.

Refraction by a \cma{principal planes} thick lens can be reduced to
the problem of the thin lens by finding the two principal planes of
the thick lens and shifting the ray between them axially
\cite{Smith2000}. The principal plane of a thick lens is located on
the intersection between an incident beam $\i$,0 that is parallel to
the optical axis, and the transmitted beam $\r$. Just as the focal
length, the principal planes are a property of lenses that are only
defined in the paraxial limit. There are always two principal planes,
one for each of the two possible illumination directions. The
distances between each principal plane and its corresponding focus
point (the intersection of $\r$ with the optical axis) are identical,
and define the focal length.

As already mentioned in section \ref{aplanatic} on page
\pageref{aplanatic} a microscope objective is a lens which is
corrected to have a constant focal length for rays of widely varying
incidence angle. In this case, the principle surface is no longer a
plane but is deformed into a spherical surface. After introducing the
formulas for the thin lens in the next section, I show in section
\ref{sec:high-aperture-lens} how to carry over those formulas to a
model that describes an aplanatic lens with immersion.

\subsection{Refraction through a paraxial thin lens}
First I describe the refraction by a thin lens: The incident beam with
direction $\i$ hits the lens at the point $\vrho$. A line parallel to
$\i$ through the centre $O$ of the lens defines the point on the focal
plane, which will be intersected by the transmitted ray $\r$ as well.


\begin{figure}[hbtp]
  \centering
  \svginput{1}{lens-fwd}
  \caption{Construction of a ray that is refracted through a thin
    lens. The incident beam with direction $\i$ (from right) hits the
    lens at the point $\vrho$. This diagram is inspired from a figure
    in \cite{Hwang2008}.}
\end{figure}


The red triangle~1 with the points $ABC$ is similar to green
triangle~2 with points $FOA$. All three angles are identical because
each of the lines are parallel: $\overline{CB} \parallel
\overline{OA} \parallel \vrho$, $\overline{FA} \parallel
\overline{CA}$ and $\overline{AB} \parallel \overline{OF} \parallel
\i$. The side $\overline{OF}$ is hypotenuse of the yellow right angled
triangle 3. Its adjacent with respect to the angle $\theta$ has length
$f$. Therefore one can deduce the length
$\abs{\overline{OF}}=f/\cos\theta$.



Between the two similar triangles, the following relation holds and
can be used to calculate the length $\abs{\overline{BC}}$:
\begin{align}
  \frac{\abs{\overline{BC}}}{\abs{\overline{BA}}}&=
  \frac{\abs{\overline{OA}}}{\abs{\overline{OF}}}\\
  \frac{\abs{\overline{CB}}}{1}&=
  \frac{\rho}{f/\cos(\theta)}.
\end{align}
Given its length, the vector $\vv{CB}$ can now be calculated, because
its direction is known to be along $\vrho$. With this vector and $\i$
one can now obtain the (arbitrarily scaled) transmitted vector
$\r'$. I could normalize this result but it turns out to be more
useful for formulas of the high aperture immersion lens to calculate
the vector $\r$, that ends in the focal plane.  The procedure from
above is condensed in the following equations:
\begin{align}
  \vrho&=(x_0,y_0,0)^T=\rho (\cos\phi,\sin\phi,0)^T\\
  \phi&=\arctan(y_0/x_0)\\
  \cos\theta&=\boxed{\i\,\hz}\\
  \r'&=\i- \frac{\cos\theta}{f}\vrho\\
  \r&=\boxed{\frac{f}{\cos\theta} \i -\vrho}
\end{align}
with the axial unit vector $\hz=(0,0,1)^T$.
\subsection{Refraction through high aperture objective (illumination)}
\label{sec:high-aperture-lens}
Now I augment the results of the calculation from the previous section
to treat an aplanatic immersion objective \citep{Hwang2008}.
\begin{figure}[!hbt]
  \centering
  \svginput{1}{obj-fwd}
  \caption{Construction of a ray on an high numerical aperture oil
    immersion objective. As opposed to a thin air lens the objective's
    focal length needs to be corrected by the focus difference vector
    $\a$ to accommodate for the immersion and one must take into
    account spherical principal surface (aplanatic surface) .}
\end{figure}
I account for the immersion medium by axially shifting the focal plane
in sample space to $nf$ using the difference vector $\a$, i.e.\ in an
immersion medium with $n=1.52$ the focus moves further away from the
principal plane. FIXME principal plane and nodal point?
\begin{align}
  \a &= \boxed{f (n-1) \hz} \\
  R &= \boxed{nf}
\end{align}
In order to account for the curvature of the aplanatic surface, the
origin of the transmitted ray is axially shifted by a $\rho-$dependent
sag $\s$ from the principal plane onto the aplanatic surface:
\begin{align}
  \s &= \left(R - \sqrt{R^2-\rho^2}\right) \hz
\end{align}
The final ray exiting the objective has the direction $\r_0$:
\begin{align}
  \r_0 &= \boxed{\r + \a - \s}.
\end{align}

All microscope lenses that come into consideration for use in our
system are designed as an aplanatic lens. The model described by above
formulas is therefore very well suited to represent the objectives
when we run our illumination optimization algorithm to find
illumination patterns for the two SLM in our spatio-angular
microscope..

In the paper \citet{Hwang2008} the authors demonstrate the viability
of this model by comparing its results with a full raytrace through a
$100\times$ objective with $NA=1.4$. There, focus displacement errors
are less than \unit[130]{nm} for a field of $\unit[86.4]{\mu m}$
radius. This is perfectly adequate for our application.

One might think it would be better to know the exact objective
parameters, i.e.\ glasses, curvatures and vertex positions of lens
surfaces. These details are, however, to my knowledge not published by
any manufacturer. In addition alignment of the components plays a
prominent role in building high performance objectives. Therefore just
the design parameters alone probably do not provide a better model of
a microscope objective. They would have to be augmented with
performance measurements of the individual objective,
e.g. point-spread functions in different regions of the field.

\subsection{Reverse path through oil objective (detection)}
Now I consider an oil immersion objective in the detection direction,
tracing rays from the sample into the pupil.

For that I present two approaches. The first and simpler one utilizes
the fact that a perfect microscope lens converts ray angle in the
sample in a linear manner into positions on the pupil. This approach
is sufficient when calculating pupil plane SLM patterns for samples in
an index matched embedding medium.

In the second approach I additionally calculate the angle in which
rays emerge from the pupil. For a perfectly aplanatic lens this would
hardly be an advantage but the formulas will be modified to take into
account aberrations.
\subsubsection{Easy case: back focal plane positions only}
If the points of intersection of rays with the back focal plane are
sufficient, a full raytrace is not necessary. This is the case with
aberration-free imaging, i.e.\ when the sample is embedded in an index
matched medium and we want to calculate a pattern for the pupil plane
SLM. Then it is possible to ignore the starting points of rays in the
specimen and just work with their directions.

A unit ray direction $\i=(x,y,z)^T$ in sample space is transformed
into a position $\r_b=(x',y')^T$ in the back focal plane of the
objective. The azimuthal angle $\phi$ isn't changed when going through
the objective. The polar angle $\theta$ defines how far off axis the
back focal plane is hit.
\begin{align}
  \phi'&=\phi=\arctan(y/x)\\
  \theta&=\arcsin(\sqrt{x'^2+y'^2})\\
  \r_b&=r_b\,(\cos\phi',\sin\phi')^T,\quad\textrm{with}\   r_b=nf\sin\theta
\end{align}
 \begin{figure}[!hbt]
   \centering
   \svginput{1}{obj-rev}
   \caption{Schematic for tracing a ray direction $\i$ from sample
     space into the back focal plane. The bigger the angle between
     $\i$ and the optical axis, the further outside the ray will pass
     through the back focal plane.}
 \end{figure}
 \subsubsection{Full raytrace through oil objective in detection
   direction}
\label{sec:objective-raytrace-detection}
Now I discuss the general case and calulate both, the origins and the
directions of rays emerging from the back focal plane. This is
necessary in order to trace light bundles from the specimen into the
plane of the camera (or focal plane SLM). In the next section I will
further modify these formulas to incorporate aberrations due to
non-index matched embedding medium.

The position of the objective is defined by its principal point $\c$
and the normal $\n$ (directed along optical axis towards sample
space). The incident ray is defined by its starting point $\p$ and the
direction $\i$. First I calculate the centre of the aplanatic sphere
$\vect g$ (see \figref{fig:obj:rev-full}).
\begin{align}
  \vect g &= \c + nf\, \n.
\end{align}
\begin{figure}[!htbp]
  \centering
  \svginput{1}{obj-rev-full}
  \caption{Construction to find the transmitted ray through an oil
    immersion objective from a point within the sample.}
  \label{fig:obj-rev-full}
\end{figure}
Then I obtain the position $\p'$ by intersecting the incident ray and
the plane perpendicular to the optical axis through the centre
$\vect{g}$ of the aplanatic sphere.  The focus difference vector is
defined by its length and the optical axis. It can be used to
calculate an intermediate point $\p''$.
\begin{align}
  \a &= -f\, (n-1)\,\n \\
  \p'' &= \p' + a.
\end{align}
The point $\p''$ has been shifted, so that an aplanatic air lens would
image it exactly as the oil objective would image $\p'$. One can use
$\p''$ to find the direction $\t$ of the transmitted ray. It is just
the normalized difference vector $\vect m$ to the principal point $\c$.
\begin{align}
  \vect m &= \c - \p'' \\
  \t &= \vect m / \abs{\vect m}.
\end{align}
As a last step I calculate the starting point $\e'$ of the transmitted
ray by intersecting the incident ray with the aplanatic sphere (in
point $\e$) and axially shifting this point onto the principal plane.

Note: In order to verify the correctness of these formulas or their
implementation it is possible to compare the algorithms of this
section (for tracing in detection direction) and section
\ref{sec:high-aperture-lens} (for illumination direction).
\subsection{Treatment of aberration (detection)}
\label{sec:ray-aberration}
Now I will extend the formulas of the previous section to include
aberrations due to a non-matched embedding medium $n_e\not=n$.

I consider a ray originating in point $\p$ with direction $\i$ within
an embedding medium of index $n_e$. I determine the intersection $\f$
of the ray with the ``cover slip--embedding'' interface and refract to
obtain $\i'$. Then I calculate the time $t$ a photon takes, to travel
from $\p$ to the interface $\f$:
\begin{align}
  t = \abs{\f - \p} \frac{n_e}{c}
\end{align}
and extend the path of the photon backward along the direction $\i'$
(corrected for the refraction at the ``cover slip--embedding'' surface) by
the distance $tc/n$. This results in the corrected position $\p'$ that
indicates where the photon would have originated if the embedding
medium were index matched.  Now I can apply the equations from the
previous section on the ray defined by $\p'$ and $\i'$ to obtain the
transmitted ray in the pupil.

 \begin{figure}[!hbt]
   \centering
   \svginput{1}{obj-rev-full-emb}
   \caption{Construction of an oil immersion objective with a
     non-index matched embedding medium.}
 \end{figure}
\section{Sphere projection}
\label{sec:sphere-projection}
While the previous sections have described a fairly general raytracer,
this section is very technical and relates to our specific problem to
represent a fluorophore distribution as a model of spheres and
simulate it with as few rays as possible.

\figref{fig:touch-cone}~D) shows a representations of the length of
the section between out-of-focus nuclei and rays starting from all
points in the pupil and going through the target point $\c$. Creating
such an image in the illumination direction requires to trace a lot of
rays (at least $50\times 50$). In order to reduce the computational
effort, I reverse the calculation direction and trace rays starting
from the periphery of out-of-focus nuclei through the target point
$\c$ in order to determine appropriate ``shadow masks'' in the pupil
plane (as depicted in \figref{fig:touch-cone}~E)). Already with six
rays per nucleus, this approach can determine very  good masks.

Now I explain how to select good points on the periphery of
out-of-focus nuclei in order to allow this calculation. I utilize the
geometry in \figref{fig:touch-cone}~A).

The tangents of an out-of-focus sphere
{\color[rgb]{0.06666667,0.50196078,0}$S^\s_r$} centred at $\s$ with
radius $r$ that pass through the target $\c$ form a double cone
(assuming $\c$ is outside of $S^\s_r$). The tangents touch the surface
of the sphere $S^\s_r$ in the circle
{\color[rgb]{0.66666667,0,0}{$C$}}. We will find a parametric
expression for the points on the circle $C$ by intersecting the sphere
$S^\s_r$ and the sphere {\color[rgb]{0.28235294,0.24313725,0.21568627}$S^\c_R$}
centred at $\c$ with radius $R=\abs{\c-\s}$ which is the distance from
the target to the centre of the out-of-focus sphere.
\begin{figure}[!htbp]
  \centering
  \svginput{1}{touch-cone}
  \caption{{\bf (A)} Schematic of how an out-of-focus nucleus and a
    target point $c$ (not necessarily in the centre of a target
    nucleus) define a cone of tangential rays. {\bf (B)} Illustration
    including the objective.  {\bf (C)} Sample distribution with six
    fluorescent spheres as used for D and E.  {\bf (D)} Diagram of the
    pupil with precisely calculated intersection length with
    out-of-focus nuclei of each ray starting in the pupil and passing
    through the target point $\c$. {\bf (E)} Same diagram as D but
    using the performance improvement as described in this section.}
  \label{fig:touch-cone}
\end{figure}

In order to find a point $\e$ where a tangent touches the out-of-focus
sphere, it is sufficient to solve the following equation in a
two-dimensional coordinate system with the origin in the centre $\s$
of the out-of-focus sphere:
\begin{align}
  (x-R)^2+y^2&=R^2\\
  x^2+y^2=r^2
\end{align}
There are two solutions:
\begin{align}
  x_1&=\frac{r^2}{2R}\label{eqn:x1}\\ 
  y_{1/2}&=\pm\frac{r}{2R}\sqrt{4R^2-r^2} \label{eqn:y1}
\end{align}
In the case $R\le r$ the out-of-focus nucleus is very close to the
target, obviating the reason to do the projection in the first
place. In the more useful case of $R>r$ there are two solutions but
either one of them is sufficient to define the circle $C$.

I construct two vectors $\hx$ and $\hy$ that span the coordinate
system, in order to transform the solution from 2D into 3D. The
direction of $\hx$ is given by the difference vector between target
$\c$ and nucleus centre $\s$. The direction $\hy$ must be
perpendicular to $\x$ and is obtained by calculating the cross product
with another vector $\q$.  I ensure that $\q$ and $\x$ are not
colinear. The vectors $\q$ and $\x$ are colinear, when the absolute
value of their scalar product equals the square of the length
$\abs{\q\x}=\x^2$.
\begin{align}
  \x&=\c-\s\\
  \q&=\begin{cases}
    (0,0,1)^T & \textrm{when}\ \abs{x_z}<\frac{2}{3}\abs{\x}\\
    (0,1,0)^T & \textrm{else}
  \end{cases}\\
  \y&=\x\times\q \\
  \hx&=\x/\abs{\x}\\
  \hy&=\y/\abs{\y}
\end{align}
Now I can sample the intersection circle $C$ in order to create
viable starting points $\e$ for tangential rays.  Let $M_\phi^\hc$ be
a rotation matrix that rotates a vector by angle $\phi$ around an axis
$\hc$. A point $\e$ on the circle is then defined using one solution
from equations \ref{eqn:x1} and \ref{eqn:y1}. The ray direction $\f$
is then easily obtained:
\begin{align}
  \e(\phi)&=\s+x_1\hx+y_1M_\phi^\hx\,\hy\\
  \f(\phi)&=\c-\e.
\end{align}
Tracing a sufficient number of rays (e.g.\ 7) with direction $\f$ for
different angles $\phi$ to the back focal plane gives the projection
of the intersection circle $C$. Note that this projection in general
is not a circle anymore.

For practical reasons I project the vector $\hx$ as well. I use it as
a centre to rasterize the shape in the pupil plane as a fan of
triangles.

\section{Conclusion}
In this chapter I have given an overview on the raytracer that I use
as a component in the illumination optimization algorithm for the
spatio-angular microscope. This software is tailored to the problem of
imaging with an aplanatic lens. I optimized the calculations so that
illumination patterns can be determined in real time, while the device
operates.

I described an algorithm that can account for aberration that occurs
when a sample is not embedded in index matched medium. On the one hand
this has a negative impact on the resolution of the detected images
already for small penetration depths ($\sim\unit[10]{\mu}$) but it
enables the interesting approach of highly inclined and laminated
optical sheet microscopy (see section \ref{sec:hilo}). In this case, a
window on the edge of the pupil is illuminated so that rays approach
the ``cover slip--medium'' surface close to the critical angle of total
reflection --- and after refraction they will traverse the medium in a
very steep angle. To illuminate the proper position in the field, the
window that is displayed on the focal plane SLM must be moved in order
to compensate for any, but mainly spherical, aberrations.

Note that ray optics are not a sufficient approximation, when
intensity features in the scale of the wavelength are to be
investigated. Small features would mean that only a few pixels of the
focal plane SLM would be enabled. This would mean that information of
the pupil plane SLM pattern is heavily filtered and no simultaneous
tight angular control would be possible. Therefore, algorithms that
are based on code in this chapter must generate patterns with big
feature sizes. Features on the pupil plane SLM should be larger than
several percent of the pupil diameter.

%FIXME maybe compare to ./cyberpower-store/0314/zeiss-patents/20080106795-correction-ring.pdf 
%or US7268953-63x.pdf

%%% Local Variables: 
%%% mode: latex
%%% TeX-master: "kielhorn_memi"
%%% End: 


