\chapter{The concept of spatio-angular microscopy}
\label{sec:concept}
\begin{summary}
  Here we introduce our spatio-angular microscope. First we motivate
  the concept of its illumination system using exemplary fluorophore
  distributions, that occur in typical specimen.

  Then we describe some decisions we faced during the initial design
  phase concerning the arrangement of optical components. Furthermore,
  we position our method among known approaches of light control for
  microscopy. Of all published techniques for excitation illumination
  control, the light field microscope \ref{levoy} comes closest to our
  approach.  We explain differences between both techniques and
  discuss their respective pros and cons.  We address the
  peculiarities and limitations of the hardware components in later
  chapters (\ref{sec:dev1}, \ref{sec:mma}).  Initially, the details
  would be detrimental to clarity.

  It turns out, that the effective use of the spatio-angular
  microscope, requires more knowledge about the specimen than a
  conventional or a SPIM microscope (\ref{spim}). Ideally the
  distribution of refractive index and fluorophores in the specimen
  should be known. If these parameters were precisely known, there
  would be no need for an image in the first place. However, while
  imaging a known specimen, sufficiently good predictions of these
  parameters can often be made. The higher the accuracy of these
  prognoses, the greater the reduction in phototoxicity will be.

  The computer-based selection of appropriate illumination masks
  requires the prediction, or at least an approximate estimation, of
  the three-dimensional distribution of light in the specimen.

  In the last part of this chapter, we describe how we practically
  implement the computational control loop in our spatio-angular
  microscope. Here we touch topics of image processing and we also
  draw parallels to treatment planning for radiotherapie of tumors.
\end{summary}
\section{Motivation}
In order to introduce the basic idea, underlying our spatio-angular
microscope, first we consider the distribution of excitation light in
the object of a conventional fluorescence microscope:
\figref{fig:hourglass-all}~a) schematically illustrates the side view
of objective lens, object and beam path of the excitation light in a
confocal microscope. A parallel beam with a circular cross-section
(this cross-section is not shown in the illustration) passes through
the lens. The lens focuses the light in its focal plane.

Between lens and focal plane the light rays form a convergent circular
cone. If refractive index variations in the object are negligible, the
light distribution below the plane of focus forms a cone as well, due
to symmetry.  Assuming a non- or weakly absorbing specimen, the energy
of the light in the circular cross-sections of the cone remains
constant. The intensity inside of the cone is proportional to the
density of light rays in each circular cross-section and therefore
increases quadratically \footnote{The ray-model is valid in large
  parts of \figref{fig:hourglass-all}~a), but not everywhere. The Law
  of Malus--Lupin states that rays and wavefronts are equivalent as
  long as rays do not intersect (caustic) or (FIXME formulas?) a
  strong intensity gradient occurs. Thus the ray-model is valid almost
  everywhere in the cone, except for a region with a distance of a few
  wavelengths to the edge and the focus itself. While the wave-optical
  treatment of these areas is possible, it is computationally much
  more expensive than ray tracing. Wave-optical effects either lead to
  blurring in a length scale of a few wavelengths or intensity
  fluctuations due to interference. If necessary, heuristics could be
  employed, to find upper bounds for the radiance from ray tracing
  results. For this reason we exclusively employ the ray-model in this
  work.}.


The fluorescent bead (1) in the focus would therefore be excited
significantly more than the bead (2) outside the focal plane. In the
confocal fluorescence microscope, the fluorescence light of the two
beads is imaged by the objective lens and the detection tube lens into
the intermediate image plane.

The image of the in-focus bead (1) is sharp, i.e.\ its emanating
fluorescence light is concentrated on an area as small as possible and
positioned exactly on the detection pinhole. Conversely, the image of
the out-of-focus bead (2) is blurred. Its fluorescence light is
distributed over a large area.

While only a tiny proportion of the light emitted by the out-of-focus
bead contributes to the detection signal of the confocal microscope,
and therefore hardly affects the image quality, with respect to
overall phototoxicity of the full confocal system, it would be better
to prevent the excitation of the out-of-focus bead in the first place.

\begin{figure}[!hbt]
  \centering
  \svginput{.43}{hourglass-all}
  \caption{{\bf (a)} Two fluorescent beads are illuminated by all
    angles that the objective can (FIXME ich mag deliver nicht,
    accept?) deliver. The sharp image of the in-focus bead is
    deteriorated by blurry fluorescence of the out-of-focus bead. {\bf
      (b)} Angular control allows selective illumination of the
    in-focus bead and results in a better image on the camera. {\bf
      (c)} Angular control is insufficient, when an extended in-focus
    area is illuminated. {\bf (d)} However, simultaneous spatial and
    angular control allows sequential excitation of the in-focus beads
    while excluding the out-of-focus bead.}
  \label{fig:hourglass-all}
\end{figure}

The scheme in \figref{fig:hourglass-all}~b) demonstrates how
the light cone would have to be manipulated so that the out-of-focus
bead is not illuminated. The expected fluorescence image in the
intermediate image plane then contains only information from the
in-focus bead (3).

Viewed from the in-focus bead (3) the change in illumination
corresponds to a restriction of the light angles. Such control can be
exerted well through a mask in the other focal plane of the objective
lens (also denoted pupil plane or back focal plane (BFP)).

Thus we have shown that it is useful and possible to equip a confocal
microscope with angular control. However, in our project we set out to
to build a wide field microscope, in order to benefit from the speed
and quantum efficiency of modern cameras.  Nonetheless, a confocal
microscope with angular illumination control seems to have potential
and this subject is worthy further investigation.  See ref
sec:conclusion on page pageref sec:conclusion for a discussion of a
combination of techniques that would lead to a promising system.

We now turn to the task of bringing angular control to the wide field
microscope. \figref{fig:hourglass-all}~c) shows a configuration of the
specimen with two in-focus beads (5) and (6), and one out-of-focus
bead (7).  If both (5) and (6) should be illuminated simultaneously,
i.e.\ the entire field is to be exposed by an extended light source,
then the out-of-focus bead (7) will always be exposed. The angular
illumination control is ineffective for this pattern.

Only by a sequential, selective illumination of the in-focus beads (8)
and (9), as shown in \figref{fig:hourglass-all}~d), angular control
regains its function. For this reason a wide field system with angular
control using a mask in the pupil requires an additional mask in the
field.  Therefore, we call our method spatio-angular
microscopy. "Spatial" refers to the illumination control in the field
and "angular" refers to the control in the pupil plane.


\begin{figure}[!hbt]
  \centering
  \svginput{1.5}{memi-simple}
  \caption{Simplified schematic of the illumination system in our
    spatio-angular microscope. A homogeneous extended light source
    illuminates from the left. It is imaged by $L_1$ and $L_2$ into
    the intermediate image $F'$. Then the tubelens $L_3$ and the
    objective $L_4$ form an image of $F'$ in the sample plane $F$. We
    use two spatial light modulators. The pupil plane SLM (PP-SLM) in
    P' and the focal plane SLM (FP-SLM) in F'.}
  \label{fig:memi-simple}
\end{figure}

\figref{fig:memi-simple} shows the optical path through our prototype
in a greatly simplified form.  From the left side, an extended light
source illuminates the system. A sequence of telecentric lenses $L_1$,
$L_2$, $L_3$ and the objective lens $L_4$ image the light source from
F'' into the front focal plane (indicated by F, for field). The
etendue $G=\frac{\pi}{4}(D_\textrm{field}\textrm{NA})^2$---also called
information capacity, light gathering capacity or space-bandwidth
product; its value is related to the number of point spread functions
that can be resolved in the field (FIXME check definitions))---of the
light source must be large enough, so that both the pupil P as well as
the field F are fully illuminated.

In each of the two planes P' and F' we place a spatial light modulator
that allows to control the intensity of the transmitted light.

Looking at the scheme in \figref{fig:memi-simple}, one could argue
that we could save a lens, if we placed the pupil plane SLM into P
instead of P'. There are three reasons why this is neither possible,
nor beneficial: First, the pupil of modern high-performance objective
lenses is not accessible\footnote{This is due to historical reasons.
  The length of objective lenses was defined, such that no refucusing
  would be necessary, when changing the objectives. In the case of
  Zeiss lenses, their length is \unit[45]{mm}. Nowadays, it would be
  easy to compensate different objective lengths using the ubiquitous
  motorized focus control.  Rather than continuing to be restricted by
  an antiquated standard, Manufacturers should produce longer
  high-performance lenses with accessible pupil plane.}  Second, the
detection path for fluorescent light should contain as few optical
components as possible and we can definitely not afford it to be
blocked by a SLM.  Thirdly, the two masks induce non-linear and
therefore difficult to predict filtering of spatial frequencies. An
analysis requires consideration of partial spatial coherence.

Considering the fact that the image of the focal plane SLM is most
important to us, we decided to place it downstream of the pupil plane
SLM. So possibly the image in the pupil is disturbed, but we can
always produce a very fine, high-contrast structures in the sample.

To be able to achieve high resolution in the field, is the main
difference between our approach and Levoys light field microscope.  In
the light field microscope the density of the microlenses noticeably
limits the resolution. Although the light field microscope allows to
control the angle of incidence in all field positions independently,
but this requires a single high-resolution SLM with a comparatively
low refresh rate. We use two small SLM, which can achieve
\unit[1]{kHz} frame rate. Structured illumination with high resolution
patterns allows us to circumvent the missing cone problem of the
widefield microscope. We use them to compute optical sections. Later
we will show that depth discrimination improves with higher resolution
patterns.
\section{A protocol for spatio-angular illumination control}
\section{Finding optimal illuminationOptimization using a raytracer}
