\chapter{The concept of spatio-angular microscopy}
\label{sec:concept}
\begin{summary}
  Here we introduce our spatio-angular microscope. First we motivate
  the concept of its illumination system using exemplary fluorophore
  distributions, that occur in typical specimen.

  Then we describe some decisions we faced during the initial design
  phase concerning the arrangement of optical components. Furthermore,
  we position our method within known approaches of light control for
  microscopy. Of all published techniques for excitation illumination
  control, the light field microscope \ref{levoy} comes closest to our
  approach.  We explain differences between both techniques and
  discuss their respective pros and cons.  We (FIXME verschieben)
  discuss the peculiarities and limitations of the hardware components
  only in later chapters (\ref{sec:dev1}, \ref{sec:mma}).  Initially,
  the details would be detrimental to clarity.

  It turns out, that the effective use of the spatio-angular
  microscope, requires more knowledge about the specimen than a
  conventional or a SPIM microscope (\ref{spim}). Ideally the
  distribution of refractive index and fluorophores within the
  specimen should be known. If these parameters were known perfectly,
  there wouldn't be any (FIXME) necessity for an image in the first
  place. However, while imaging a known specimen, predictions (FIXME
  gute) of these parameters can often be made. The higher the
  precision of these predictions, the greater the reduction in
  phototoxicity will be.

  The computer-based selection of appropriate illumination masks
  requires the prediction (FIXME), or at least an estimate (FIXME
  understanding), of the three-dimensional distribution of light within
  the specimen.

  In the last part of this chapter, we describe how we practically
  implement the computational control loop in our spatio-angular
  microscope. Here we touch topics of image processing and we also
  draw parallels to treatment planning for radiotherapie of tumors.
\end{summary}
\section{Motivation}

  - Um die grundlegende Idee hinter dem Spatio-Angularen Mikroskop zu
    verstehen, betrachten wir zunaechst die Lichtverteilung im Objekt
    bei einem herkoemmlichen Mikroskop: Abbildung fig:hourglass-all-a
    zeigt schematisch die Seitenansicht von Objektivlinse, Objekt und
    dem Strahlenverlauf des Anregungslichtes in einem konfokalen
    Mikroskop. Ein paralleles Lichtbuendel mit kreisfoermigem
    Querschnitt (in der Darstellung nicht sichtbar) trifft auf die
    Objektivlinse. Die Linse fokussiert das Licht in ihrer Brennebene.

  - Zwischen Linse und Brennebene bilden die Lichtstrahlen einen
    konvergenten Kreiskegel. Angenommen, wir haben eine schwach
    absorbierende Probe, die Energie des Lichtes entlang der
    kreisfoermigen Querschnitte innerhalb des Kegels bleibt dann
    konstant. Die Intensitaet innerhalb des Kegels ist proportional
    zur Dichte der Lichtstrahlen in jedem kreisfoermigen Querschnitt
    und steigt demnach quadratisch an\footnote{Das strahlenoptische
    Modell gilt in grossen Teilen der Darstellung, jedoch nicht
    ueberall.  Das Gesetz von Malus-Lupin besagt, dass die
    Beschreibung mit Lichtstrahlen oder Wellenfronten equivalent sind,
    solange sich Strahlen nicht ueberschneiden (Kaustik) oder (FIXME
    formeln) oder ein starker Intensitaetsgradient auftritt. Demnach
    gilt das strahlenoptische Modell fast ueberall im Kegel, bis auf
    einen Bereich mit einem Abstand von wenigen Wellenlaengen zum Rand
    und im Fokus selbst. Die wellenoptische Behandlung dieser Bereiche
    ist zwar moeglich, rechentechnisch aber erheblich
    aufwaendiger. Deshalb beschraenken wir uns in unserem Prototypen
    und dieser Arbeit ausschliesslich auf das strahlentheoretische
    Modell}.

  - Der fluoreszente Bead (1) im Fokus wuerde demnach deutlich
    staerker angeregt werden, als der Bead (2) ausserhalb der
    Fokusebene. Im konfokalem Fluoreszensmikroskop wird das
    Fluoreszenslicht beider Beads vom Objektiv und
    Detektionstubuslinse in die Zwischenbildebene abgebildet werden.
    Das Bild des in-focus Beads (1) ist dabei scharf, von ihm
    ausgehendes Fluoreszenslicht wird auf einer moeglichst kleinen
    Flaeche konzentriert -- genau auf dem Zentrum des
    Detektionspinholes.  Der out-of-focus Bead (2) erzeugt hingegen
    nur ein unscharfes Bild, sein Licht wird ueber eine grosse Flaeche
    verteilt. Zum detektierten Signal des konfokalen Mikroskops traegt
    zwar nur ein verschwindend geringer Anteil des vom Out-of-fokus
    Beads emittierten Lichts bei, mit Blick auf die Phototoxizitaet
    des Systems kann man jedoch sagen, dass es besser waere, die
    Anregung des out-of-fokus Beads von vornherein zu unterbinden.


\begin{figure}[!hbt]
  \centering
  \svginput{.43}{hourglass-all}
  \caption{{\bf (a)} Two fluorescent beads are illuminated by all
    angles that an objective can deliver. The sharp image of the
    in-focus bead is deteriorated by blurry fluorescence of the
    out-of-focus bead. {\bf (b)} Angular control allows selective
    illumination of the in-focus bead and results in a better image on
    the camera. {\bf (c)} Angular control is insufficient, when an
    extended in-focus area is illuminated. {\bf (d)} However,
    simultaneous spatial and angular control allows sequential
    excitation of the in-focus beads while excluding the out-of-focus
    bead.}
  \label{fig:hourglass-all}
\end{figure}



\begin{figure}[!hbt]
  \centering
  \svginput{1.5}{memi-simple}
  \caption{Simplified schematic of the illumination system in our
    spatio-angular microscope. A homogeneous extended light source
    illuminates from the left. It is imaged by $L_1$ and $L_2$ into
    the intermediate image $F'$. Then the tubelens $L_3$ and the
    objective $L_4$ form an image of $F'$ in the sample plane $F$. The
    first spatial light modulator SLM1 is in the plane P', which is
    conjugate to the pupil (BFP) P of the objective. Using SLM1 we can
    control illumination angles in the sample. SLM2 is directly imaged
    into the sample and allows spatial illumination
    control.} 
  \label{fig:memi-simple}
\end{figure}

\section{A protocol for spatio-angular illumination control}
\section{Finding optimal illuminationOptimization using a raytracer}
