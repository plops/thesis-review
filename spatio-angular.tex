\chapter{The concept of spatio-angular microscopy}
\label{sec:concept}
\begin{summary}
  Here we introduce our spatio-angular microscope. First we motivate
  the concept of this illumination system using 
  exemplary fluorophore distributions, that occur
  in typical specimen.

  Then we describe some decisions we faced during the initial design
  phase concerning the arrangement of optical components. Of all published
  techniques for excitation illumination control, the light field
  microscope \ref{levoy} comes closest to our approach.
  We explain differences between both techniques and discuss their
  respective pros and cons. 
  We discuss the peculiarities and limitations of the hardware components
  only in later chapters (\ref{sec:dev1}, \ref{sec:mma}).
  Initially, the details would be detrimental to clarity.

  It turns out, that effective use of the
  spatio-angular microscope, which results in a
  reduction of phototoxicity, requires more knowledge (distribution of refractive index and fluorophores)
  about the specimen than a conventional or a SPIM microscope
  (\ref{spim}). 
  The computer-based selection of suitable illumination masks requires the
  prediction, or at least an understanding, of the three-dimensional
  distribution of light within the specimen.
  In the last part of this
  chapter, we describe how we practically implement our spatio-angular
  microscope. Here we touch topics of image processing and we also draw
  parallels to treatment planning for radiotherapie of tumors.
\end{summary}
\section{Motivation}
\section{A protocol for spatio-angular illumination control}
\section{Finding optimal illuminationOptimization using a raytracer}
