\chapter{The concept of spatio-angular microscopy}
\label{sec:concept}
\begin{summary}
  Here we introduce our spatio-angular microscope. First we motivate
  the concept of its illumination system using exemplary fluorophore
  distributions, that occur in typical specimen.

  Then we describe some decisions we faced during the initial design
  phase concerning the arrangement of optical components. Furthermore,
  we position our method within known approaches of light control for
  microscopy. Of all published techniques for excitation illumination
  control, the light field microscope \ref{levoy} comes closest to our
  approach.  We explain differences between both techniques and
  discuss their respective pros and cons.  We (FIXME verschieben)
  discuss the peculiarities and limitations of the hardware components
  only in later chapters (\ref{sec:dev1}, \ref{sec:mma}).  Initially,
  the details would be detrimental to clarity.

  It turns out, that the effective use of the spatio-angular
  microscope, requires more knowledge about the specimen than a
  conventional or a SPIM microscope (\ref{spim}). Ideally the
  distribution of refractive index and fluorophores within the
  specimen should be known. If these parameters were known perfectly,
  there wouldn't be any (FIXME) necessity for an image in the first
  place. However, while imaging a known specimen, predictions (FIXME
  gute) of these parameters can often be made. The higher the
  precision of these predictions, the greater the reduction in
  phototoxicity will be.

  The computer-based selection of appropriate illumination masks
  requires the prediction (FIXME), or at least an estimate (FIXME
  understanding), of the three-dimensional distribution of light within
  the specimen.

  In the last part of this chapter, we describe how we practically
  implement the computational control loop in our spatio-angular
  microscope. Here we touch topics of image processing and we also
  draw parallels to treatment planning for radiotherapie of tumors.
\end{summary}
\section{Motivation}
To understand the basic idea behind our spatio-angular microscope,
first we consider the distribution of light within the object of a
conventional microscope: Figure fig:hourglass-all-a schematically
illustrates the side view of objective lens, object and beam path of
the excitation light in a confocal microscope. A parallel beam with a
circular cross-section (not visible in the illustration) passes
through the lens. The lens focuses the light in its focal plane.

Between lens and focal plane the light rays form a convergent circular
cone. If refractive index variations in the object are negligible, the
light distribution below the plane of focus forms a cone as well, due
to symmetry.  Assuming a non- or weakly absorbing specimen the energy
of the light within the circular cross-sections of the cone remains
constant. The intensity within the cone is proportional to the density
of light rays in each circular cross-section and therefore increases
quadratically \footnote{ The ray-model is valid in large parts of
  Figure fig: hourglass-all-a, but not everywhere. The Law of Malus
  Lupin states that rays and wavefronts are equivalent as long as rays
  do not intersect (caustic) or (FIXME formulas) a strong intensity
  gradient occurs. Thus the ray-model is valid almost everywhere in
  the cone, except for a region with a distance of a few wavelengths
  to the edge and the focus itself. While the wave-optical treatment
  of these areas is possible, it is computationally much more
  expensive than ray tracing. For this reason we exclusively employ
  the ray optical model in this work.}.


The fluorescent bead (1) in the focus would therefore be excited
significantly more than the bead (2) outside the focal plane. In the
confocal fluorescence microscope, the fluorescent light of the two
beads is imaged by the objective lens and the detection tube lens into
the intermediate image plane.

The image of the in-focus bead (1) is sharp, its emanating fluorescent
light is concentrated on an area as small as possible and positioned
exactly on the detection pinhole.  On the other hand, the out-of-focus
bead (2) gives rise to a blurred image. Its fluorescence light is
distributed over a large area.

While only a tiny proportion of the light emitted by the out-of-focus
bead contributes to the detection signal of the confocal microscope,
but with respect to overall phototoxicity of the full confocal system,
it would be better to prevent the excitation of the out-of-focus bead
from the outset.

\begin{figure}[!hbt]
  \centering
  \svginput{.43}{hourglass-all}
  \caption{{\bf (a)} Two fluorescent beads are illuminated by all
    angles that an objective can deliver. The sharp image of the
    in-focus bead is deteriorated by blurry fluorescence of the
    out-of-focus bead. {\bf (b)} Angular control allows selective
    illumination of the in-focus bead and results in a better image on
    the camera. {\bf (c)} Angular control is insufficient, when an
    extended in-focus area is illuminated. {\bf (d)} However,
    simultaneous spatial and angular control allows sequential
    excitation of the in-focus beads while excluding the out-of-focus
    bead.}
  \label{fig:hourglass-all}
\end{figure}



\begin{figure}[!hbt]
  \centering
  \svginput{1.5}{memi-simple}
  \caption{Simplified schematic of the illumination system in our
    spatio-angular microscope. A homogeneous extended light source
    illuminates from the left. It is imaged by $L_1$ and $L_2$ into
    the intermediate image $F'$. Then the tubelens $L_3$ and the
    objective $L_4$ form an image of $F'$ in the sample plane $F$. The
    first spatial light modulator SLM1 is in the plane P', which is
    conjugate to the pupil (BFP) P of the objective. Using SLM1 we can
    control illumination angles in the sample. SLM2 is directly imaged
    into the sample and allows spatial illumination
    control.} 
  \label{fig:memi-simple}
\end{figure}

\section{A protocol for spatio-angular illumination control}
\section{Finding optimal illuminationOptimization using a raytracer}
