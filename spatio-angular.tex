\chapter{The concept of spatio-angular microscopy}
\label{sec:concept}
\begin{summary}
  Here we introduce our spatio-angular microscope. First we motivate
  the concept of this illumination system at hand (FIXME anhand eines
  beispiels zeigen) of exemplary fluorophore distributions, that occur
  in typical specimen.

  Then we describe some decisions we faced during the initial design
  phase regarding the order of optical components. Of all published
  techniques for excitation illumination control, the light field
  microscope \ref{levoy} resembles our approach to the highest degree
  (FIXME). We explain differences between both techniques and argue
  why our system (FIXME nachstehen, vorteile aufweisen).

  It turns out (FIXME herausstellen), that an effective use of the
  spatio-angular microscope, which (FIXME use=which?) results in a
  reduction of phototoxicity, requires more knowledge (FIXME quite,
  knowledge) about the specimen than conventional or a SPIM microscope
  (\ref{spim}) (distribution of refractive index and fluorophores). In
  order to computationally choose sensible (FIXME passend)
  illumination masks requires the prediction, or at least an
  understanding, of the three dimensional distribution of the
  illumination light within the specimen. in the last part of this
  chapter, we describe how we practically implement our spatio-angular
  microscope. we touch topics of image processing and we also show
  similarities (FIXME parallele ziehen) to medical tumor therapy with
  ionizing radiation.

\end{summary}
\section{Motivation}
\section{A protocol for spatio-angular illumination control}
\section{Finding optimal illuminationOptimization using a raytracer}
