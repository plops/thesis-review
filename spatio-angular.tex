\chapter{The concept of spatio-angular microscopy}
\label{sec:concept}
\begin{summary}
  Here we introduce our spatio-angular microscope. First we motivate
  the concept of its illumination system using exemplary fluorophore
  distributions, that occur in typical specimen.

  Then we describe some decisions we faced during the initial design
  phase concerning the arrangement of optical components. Furthermore,
  we position our method within known approaches of light control for
  microscopy. Of all published techniques for excitation illumination
  control, the light field microscope \ref{levoy} comes closest to our
  approach.  We explain differences between both techniques and
  discuss their respective pros and cons.  We (FIXME verschieben)
  discuss the peculiarities and limitations of the hardware components
  only in later chapters (\ref{sec:dev1}, \ref{sec:mma}).  Initially,
  the details would be detrimental to clarity.

  It turns out, that the effective use of the spatio-angular
  microscope, requires more knowledge about the specimen than a
  conventional or a SPIM microscope (\ref{spim}). Ideally the
  distribution of refractive index and fluorophores within the
  specimen should be known. If these parameters were known perfectly,
  there wouldn't be any (FIXME) necessity for an image in the first
  place. However, while imaging a known specimen, predictions (FIXME
  gute) of these parameters can often be made. The higher the
  precision of these predictions, the greater the reduction in
  phototoxicity will be.

  The computer-based selection of appropriate illumination masks
  requires the prediction (FIXME), or at least an estimate (FIXME
  understanding), of the three-dimensional distribution of light within
  the specimen.

  In the last part of this chapter, we describe how we practically
  implement the computational control loop in our spatio-angular
  microscope. Here we touch topics of image processing and we also
  draw parallels to treatment planning for radiotherapie of tumors.
\end{summary}
\section{Motivation}


\begin{figure}[!hbt]
  \centering
  \svginput{1.5}{memi-simple}
  \caption{Simplified schematic of the illumination system in our
    spatio-angular microscope. A homogeneous extended light source
    illuminates from the left. It is imaged by $L_1$ and $L_2$ into
    the intermediate image $F'$. Then the tubelens $L_3$ and the
    objective $L_4$ form an image of $F'$ in the sample plane $F$. The
    first spatial light modulator SLM1 is in the plane P', which is
    conjugate to the pupil (BFP) P of the objective. Using SLM1 we can
    control illumination angles in the sample. SLM2 is directly imaged
    into the sample and allows spatial illumination
    control.} 
  \label{fig:memi-simple}
\end{figure}
\section{A protocol for spatio-angular illumination control}
\section{Finding optimal illuminationOptimization using a raytracer}
